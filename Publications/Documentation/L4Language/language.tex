
%......................................................................
\subsection{Syntax}\label{sec:syntax}

\hypertarget{syn:Program}{}
\begin{rail}
  Program : (TopLevelElement +)
  ;
\end{rail}

\hypertarget{syn:TopLevelElement}{}
\begin{rail}
TopLevelElement : ClassDecl     
                | GlobalVarDecl 
                | GlobalVarDefn 
                | RuleOrFact    
                | Assertion     
                | Automaton     
                | System        
;
\end{rail}

\hypertarget{syn:ClassDecl}{}
\begin{rail}
ClassDecl : 'class' 'VAR' ('extends' 'VAR')? Fields
;                
\end{rail}

\hypertarget{syn:Fields}{}
\begin{rail}
Fields : ( '\{' (  'VAR' ':' Tp  +) '\}' )?
;
\end{rail}

\hypertarget{syn:VarDecl}{}
\begin{rail}
VarDecl : 'VAR' ':' Tp
;
\end{rail}

\hypertarget{syn:GlobalVarDecl}{}
\begin{rail}
GlobalVarDecl : 'decl' VarDecl
;
\end{rail}


\hypertarget{syn:GlobalVarDefn}{}
\begin{rail}
GlobalVarDefn : 'defn' VarDecl '=' Expr  
;
\end{rail}

\hypertarget{syn:ARName}{}
\begin{rail}
ARName : ( '<' 'VAR' '>' )?
;
\end{rail}

\hypertarget{syn:Rule}{}
\begin{rail}
Rule : 'rule' ARName  KVMap RuleVarDecls 'if' Expr 'then' Expr
;
\end{rail}

\hypertarget{syn:Fact}{}
\begin{rail}
Fact : 'fact' ARName  KVMap RuleVarDecls Expr
;
\end{rail}

% \hypertarget{syn:RuleVarDecls}{}
\begin{rail}
RuleVarDecls : 'for' ( VarDecl + ',' )
;
\end{rail}

\hypertarget{syn:Assertion}{}
\begin{rail}
Assertion : 'assert' ARName KVMap Expr
;
\end{rail}


\remms{TODO:}
\begin{itemize}
\item Grammar rule for \texttt{KVMap} to be defined
\item Grammar for \texttt{Tp} and \texttt{Expr}
\end{itemize}


A \nonterminalref{Program} is the main processing unit. It consists of a list
of top level elements that can be arranged in any order. A
\nonterminalref{TopLevelElement} can be a mapping, a list class declarations,
global variable declaration or definition, a rule, fact or assertion of a system.

A \nonterminalref{Mapping} maps identifiers of the program to GF
strings\remms{Inari: more details}.

Class declarations (\nonterminalref{ClassDecl}) come in several shapes. In
its simplest form, a class declaration introduces a new class and relates it
to a superclass, as in
\begin{lstlisting}
class Lawyer extends Person
\end{lstlisting}  

The \texttt{extends} clause can also be omitted, in which case the superclass
is implicitly assumed to be \texttt{Class} (see \secref{sec:typing} for the
built-in class hierarchy). Thus,
\begin{lstlisting}
class Person
\end{lstlisting}  
is equivalent to:
\begin{lstlisting}
class Person extends Class
\end{lstlisting}

New fields can be added to a class with field declarations
(\nonterminalref{Fields}). These are variable declarations
enclosed in braces; they can also be missing altogether (equivalent to empty
field declarations \texttt{\{\}}). For example, 
\begin{lstlisting}
class Person {
   name : String
   age : Integer }

class Lawyer extends Person {
   companyName : String }
\end{lstlisting}  
introduces two fields \texttt{name} and \texttt{age} to class \texttt{Person},
whereas \texttt{Lawyer} inherits \texttt{name} and \texttt{age} from
\texttt{Person} and in addition defines a third field \texttt{companyName}.

Global variable declarations (\nonterminalref{GlobalVarDecl}) introduce names
that are meant to have global visibility in the program.

A \nonterminalref{Rule} introduces local variables in a
\texttt{for} clause (which may be omitted if there are no local variables),
followed by a precondition (\texttt{if}) and a postcondition (\texttt{then}),
both assumed to be Boolean expressions. If there are several preconditions,
these have to be conjoined by \emph{and} to form a single formula. 
A \nonterminalref{Fact} is a notational variant of a  \nonterminalref{Rule}
that does not have a precondition (more technically, a fact is mapped to a
rule whose precondition is \texttt{True}).

An \nonterminalref{Assertion} is a Boolean expressions introduced by the keyword \texttt{assert}.


%......................................................................
\subsection{Typing}\label{sec:typing}



%......................................................................
\subsection{Pragmatics}\label{sec:pragmatics}



%%% Local Variables:
%%% mode: latex
%%% TeX-master: "main"
%%% End:
