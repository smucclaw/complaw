
%......................................................................
\subsection{Automata}\label{sec:automata}

Automata are an appropriate means for describing \emph{time-dependent} systems, as
opposed to rules and facts that provide a \emph{static} view.

In the present form, the L4 automata are inspired by the theory of Timed
Automata \cite{larsen1997uppaal} and in particular by the Uppaal
systen\footnote{\url{https://uppaal.org/}}. Our long-term aim is to go beyond the model
checking capabilities of Uppaal by a mapping of automata to SMT solvers. As
long as this feature is under development, we try to achieve maximal
compatibility so that Uppaal can be used together with L4, in two directions:

\begin{itemize}
\item The internal L4 data structures can be printed in the Uppaal
  \texttt{*.xta} format. An example is given in \figref{fig:system_uppaal}.
\item \texttt{*.xta} files written by Uppaal can be read by L4, with minor
  syntactic differences and restrictions. An example is given in \figref{fig:system_l4}.
\end{itemize}

These two methods will be revisited in the following.

\begin{figure}
  \begin{lstlisting}
chan x, y;
process Aut () {
    clock clock0, clock1;
    state
        loc0,
        loc1 { clock1 <= 4 },
        loc2 { clock1 <= 5 },
        loc3
        ;

    init loc0;
    trans
        loc1 -> loc3 { guard clock0 == 7 ;   },
        loc2 -> loc1 { guard clock1 >= 3 ;   },
        loc1 -> loc2 { guard clock1 >= 3 ;  assign clock1 = 0; },
        loc0 -> loc1 {    }
        ;
    }
system Aut; 
  \end{lstlisting}
  \caption{System specification in Uppaal format}\label{fig:system_uppaal}
\end{figure}

\begin{figure}
  \begin{lstlisting}
system AutSys {
    chan x, y;
    process Aut () {
        clock clock0, clock1;
        state
            loc0,
            loc1 { clock1 <= 4 },
            loc2 { clock1 <= 5 },
            loc3
            ;

        init loc0;
        trans
            loc1 -> loc3 { guard clock0 == 7 ;   },
            loc2 -> loc1 { guard clock1 >= 3 ;   },
            loc1 -> loc2 { guard clock1 >= 3 ;  assign clock1 = 0; },
            loc0 -> loc1 {    }
            ;
        }
    }
  \end{lstlisting}
  \caption{System specification in L4 format}\label{fig:system_l4}
\end{figure}



\subsubsection{Automata and Systems}

An \emph{automaton} is a graph commposed of nodes (states) and arcs
(transitions). States and transitions can be annotated with conditions
(temporal or not) and synchronization information. A \emph{system} is composed
of one or several automata and may contain declarations of channels (for
synchronization), clocks etc.


\subsubsection{From L4 to Uppaal}

The Uppaal \texttt{*.xta} format contains all the relevant structural
information for processing an automaton in Uppaal. However, no graphical
information is available, which typically leads to a poor layout when opening
an \texttt{*.xta} file.

An L4 program containing a system can be printed in Uppaal syntax with the
fake assertion
\begin{lstlisting}
  assert {printUp} True
\end{lstlisting}
which can be run as described in \secref{sec:running_solver}. The resulting
text can be written on a \texttt{*.xta} file and be processed by Uppaal. The
directive \texttt{printL4} produces the L4 format of automata.

Genuine proof obligations are currently not written to Uppaal (they would have
to be written to the corresponding \texttt{*.q} file).

\subsubsection{From Uppaal to L4}

A Uppaal \texttt{*.xta} file can be copied into L4 and then be manipulated as
any other L4 source. The L4 syntax and Uppaal syntax are sufficiently similar
so as to not require major changes, but the following manual postprocessing is
necessary:
\begin{itemize}
\item Automata and declarations belonging to one system have to be included
  into a system specification \texttt{system SysName \{ ... \}} as in
  \figref{fig:system_l4}. Several system specifications may coexist in an L4
  file.
\item Comments begin with a double slash (\texttt{//}) in Uppaal, which is not
  recognized by the L4 parser, and therefore have to be replaced by a sharp
  (\texttt{\#}) in L4.
\item Some expressions may be written differently, in particular the constants
  \texttt{True} and \texttt{False} which are written lowercase in Uppaal.
\end{itemize}

A notable difference is that Uppaal permits the declaration of automaton
\emph{templates} that can be instantiated. This is currently not possible in L4.


%......................................................................
\subsection{Rules}\label{sec:rules}


%......................................................................
\subsection{SMT Solver}\label{sec:smt_solver}


\subsubsection{Assertions}\label{sec:assertions}


\subsubsection{Running the solver}\label{sec:running_solver}



%......................................................................
\subsection{Expert Systems}\label{sec:epert_systems}





%%% Local Variables:
%%% mode: latex
%%% TeX-master: "main"
%%% End:
