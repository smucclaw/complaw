\def\corelfour{\emph{Core L4}\xspace}
\def\naturalfour{\emph{Natural L4}\xspace}

The L4 language family includes \naturalfour, a developer-facing
front-end language whose syntax is close to a controlled natural
language, and \corelfour, an intermediate representation language to
which \naturalfour reduces.

Given a ``legal program'' written in \naturalfour, the \texttt{natural4}
compiler parses the input, establishes an initial abstract syntax tree
(AST), and performs initial interpretation.

The \texttt{natural4} compiler then transpiles to certain language
back-ends which do not require further processing. For example,
Typescript classes are produced directly at this point from \naturalfour
class declarations.

The \texttt{natural4} compiler also transforms the input into an
intermediate representation (IR) for further processing, including a
comprehensive type inference and type checking pass. That IR is called
\corelfour, and is the subject of this document.

From the Core L4 IR, further transpilation targets are
generated, such as logic programming languages, formal verification
targets, and specification languages.

This document begins by presenting a concrete syntax for Core L4. This
syntax is conceptually important as it gives an approachable entry
point for understanding Core L4's semantics. In 2021--2022 the
transformation pathway from Natural L4 to Core L4 was explicit, via a
text file format. From 2023 onwards, this transformation was conducted
in memory, through direct conversion from one family of Haskell types
to another.

This document then discusses the underlying logics and semantics of Core L4.

The Natural L4 language is documented elsewhere. Legal engineers
writing L4 are not expected to go this far into the details of the L4
language family. This documentation is intended for language designers
and internals engineers working on the L4 compilers and interpreters
directly.

In the remainder of this document, ``L4'' refers to \corelfour.

\remms{to be continued}


%%% Local Variables:
%%% mode: latex
%%% TeX-master: "main"
%%% End:
