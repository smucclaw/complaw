\begin{itemize}
  \item Each natural l4 rule gives rise to a fragment of an automaton. 

  \item In the end, we stitch all these partial automata together.

  \item Diff rules may talk about the same automata, need to figure out how
  to sync and connect them together.
\end{itemize}

\subsection{Natural l4 to automata}
\begin{itemize}
  \item \S \texttt{ Rule name}
  
  this defines a state which is the entry point to the rest of the automaton
  representing the rule

  \item \texttt{Every/Party (entity label)}

  This indicates that the rule to be constructed is part of the automaton
  corresponding to the entity label.

  \begin{itemize}
    \item Q: Any semantic diff between every and party?
  \end{itemize}

  \item \texttt{subject constraint}

  not sure what this means

  \item \texttt{attribute constraint}

  not sure about this either

  \item \texttt{conditional constraint}

  ie if, when, unless

  These are guards on the action, represented as a transition
  
  Need to clarify: when = if, unless = if not

  \item \texttt{upon trigger}

  upon action a creates a transition labelled with the action a into
  this automaton.

  synchronize with other action a somewhere else in the network.

  \item How to handle clocks? What clock variables to use and when
  to reset?

  \item switch/case or match would be nice
  
  find some concrete example for motivation
\end{itemize}

As an example, consider the \S \, \texttt{Assessment} rule in the
\href{https://docs.google.com/spreadsheets/d/1leBCZhgDsn-Abg2H_OINGGv-8Gpf9mzuX1RR56v0Sss/edit#gid=1779650637}
{PDPA case}.
This gives rise to an automaton with a structure as shown below.

% Drawn with the help of https://madebyevan.com/fsm/
\begin{center}
  \begin{tikzpicture}[scale=0.2]
    \tikzstyle{every node}+=[inner sep=0pt]
    \draw [black] (39.2,-9.2) circle (3);
    \draw (39.2,-5) node {\texttt{\S \, Assessment}};
    \draw [black] (39.2,-25.9) circle (3);
    \draw (48.5,-25.2) node {
      $\substack{
        \\
        q
        \\
        x \leq 30
      }$
    };
    \draw [black] (25.1,-43.9) circle (3);
    \draw (25.1,-48.5) node {\texttt{\S \, Notification}};
    \draw [black] (54.3,-42.1) circle (3);
    \draw (54.3,-46.5) node {\texttt{R}};
    \draw [black] (39.2,-12.2) -- (39.2,-22.9);
    \fill [black] (39.2,-22.9) -- (39.7,-22.1) -- (38.7,-22.1);
    \draw (38.7,-17.55) node [left] {
      $\substack{
        \\
        \text{data breach occurs on or after ...}
        \\
        \text{becomeAware?}
        \\
        x := 0
      }$
    };
    \draw [black] (37.35,-28.26) -- (26.95,-41.54);
    \fill [black] (26.95,-41.54) -- (27.84,-41.22) -- (27.05,-40.6);
    \draw (31.58,-33.48) node [left] {Assess};
    \draw [black] (41.25,-28.09) -- (52.25,-39.91);
    \fill [black] (52.25,-39.91) -- (52.07,-38.98) -- (51.34,-39.66);
    \draw (54,-34) node [left] {$x = 30$};
  \end{tikzpicture}
\end{center}

Note here that:
\begin{itemize}
  \item For simplicity, we ignore attribute constraints and subject constraints
  like \texttt{WHO}, \texttt{WHICH}, etc.

  \item ``data breach occurs on or after ...'' is the conditional constraint,
  represented by a guard on the transition leading into the rest of the rule.

  \item ``becomeAware'' is a synchronization action, corresponding to the
  ``\texttt{UPON} becoming aware...'' part.

  \item $x$ is taken to be a fresh clock variable.

  \item $q$ is a fresh, anonymous state whose main purpose is to hold the clock
  invariant of $x \leq 30$.

  \item \texttt{\S \, Notification} represents the initial state of the
  automaton corresponding to the rule following the \texttt{HENCE} part.

  \item $R$ represents the initial state of the automaton corresponding to the
  rule following the \texttt{LEST} part.
\end{itemize}

The purpose of $x$ and $q$ is to model a time-out, following the
Time-out pattern described in
\cite[Definition 12]{sunjun_2008_timed_automata_patterns}.
The idea is that once the guard condition
(ie data breach occurs on or after ...) is satisfied,
and the organization becomes aware, so that a becomeAware! transition occurs,
we start a timer given by the fresh clock variable $x$ and enter state $q$.

The clock invariant $x \leq 30$ enforces that when the automaton is in state
$q$, we may only introduce a delay of up to 30 units of time before one of
the 2 outgoing transitions must be taken.
Taking 1 unit of time to be 1 day, this means that:
\begin{itemize}
  \item If the organization assesses within 30 days, the \texttt{Assess}
  transition is taken so that the automaton is now in the
  \texttt{\S \, Notification} state, which is the beginning of the
  corresponding rule. 

  \item If 30 days passes and the organization doesn't assess, the transition
  to the \texttt{R} state becomes enabled and is thus taken (because the
  automaton cannot stay in state $q$ anymore).
\end{itemize}

\subsection{Automata to natural l4}
This seems trickier.