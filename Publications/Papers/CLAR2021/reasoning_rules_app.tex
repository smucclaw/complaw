% ----------------------------------------------------------------------
\section{Reasoning with and about Rules}\label{sec:resasoning_with_rules_app}
% ----------------------------------------------------------------------


% ----------------------------------------------------------------------
\subsection{Facets of Defeasible Reasoning}\label{sec:facets}

Given a plethora of different notions of ``defeasibility'', we had to make a
choice as to which notions to support, and which semantics to give to them. We
will here concentrate on two concepts, which we call \emph{rule modifiers},
that limit the applicability of rules and make them ``defeasible''. They will
be presented in \secref{sec:intro_rule_modifiers}, and the rest of this section will be dedicated to giving
them a semantics in a classical first-order logic, by means of apparently different interpretations.
A discussion of a non-monotonic
logic, in the context of on Answer Set Programming, will be deferred to
(\secref{sec:defeasible_asp}). An orthogonal question is that of
arriving at conclusion in absence of complete information, which, via the
mechanism of negation as \remms{as/by??} failure, often prones the use of
non-monotonic logics. In \secref{sec:rule_inversion}, we will see how similar
effects can be achieved by means of \emph{rule inversion}.  We finish
with a comparison of the two strands of reasoning in \secref{sec:comparison}.\remms{This intro is a mess and has to be rewritten depending on how we organize the paper}


% ----------------------------------------------------------------------
\subsection{Introducing Rule Modifiers}\label{sec:intro_rule_modifiers}

We will concentrate on two rule modifiers that restrict the applicability of
rules and that frequently occur in law texts: \emph{subject to} and
\emph{despite}. To illustrate their use, we consider an excerpt of Singapore's
Professional Conduct Rules \S~34 \cite{professional_conduct_rules} (also see
\cite{morris21:_const_answer_set_progr_tool} for a more detailed treatment of
this case study):

\begin{description}
\item[(1)] A legal practitioner must not accept any executive appointment
  associated with any of the following businesses: 
  \begin{description}
  \item[(a)] any business which detracts from, is incompatible with, or
    derogates from the dignity of, the legal profession;
  \item[(b)] any business which materially interferes with the legal
    practitioner’s primary occupation of practising as a lawyer; (\dots)
  \end{description}
\item[(5)] Despite paragraph (1)(b), but subject to paragraph (1)(a) and (c)
  to (f), a locum solicitor may accept an executive appointment in a business
  entity which does not provide any legal services or law-related services, if
  all of the conditions set out in the Second Schedule are satisfied.
\end{description}

The two main notions developed in the Conduct Rules are which executive appointments a legal
practictioner \emph{may} or \emph{must not} accept under which
circumstances. As there is currently no direct support for deontic logics in
L4, these notions are defined as two predicates \texttt{MayAccept} and
\texttt{MustNotAccept}, with the intended meaning that these two notions are
contradictory, and this is indeed what will be provable after a complete
formalization.

Let us here concentrate on the modifiers \emph{despite} and \emph{subject
  to}. A synonym of ``despite'' that is often used in legal texts is
``notwithstanding'',  and a synonym of
``subject to'' is ``except as provided in'', see \cite{adams_contract_drafting_2004}.

The reading of rule (5) is the following:
\begin{itemize}
\item ``subject to paragraph (1)(a) and (c) to (f)'' means: rule (5) applies
  as far as (1)(a) and (c) to (f) is not established. Differently said, rules
  (1)(a) and (c) to (f) undercut or defeat rule (5).

  One way of explicitating the ``subject to'' clause would be to rewrite (5)
  to: ``Despite paragraph (1)(b), provided the business does not detract from,
  is incompatible with, or derogate from the dignity of, the legal profession;
  and provided that not [clauses (1)(c) to (f)]; then a locum
  sollicitor\footnote{in our class-based terminology, a subclass of legal
    practitioner} may accept an executive appointment.''

\item ``despite paragraph (1)(b)'' expresses that rule (5) overrides rule
  (1)(b). In a similar spirit as the ``subject to'' clause, this can be made
  explicit by introducing a proviso, however not locally in  rule (5), but
  remotely in rule (1)(b).

  One way of explicitating the ``despite'' clause of rule (5) would be to
  rewrite (1)(b) to: ``Provided (5) is not applicable, a legal practitioner
  must not accept any executive appointment associated with any business which
  materially interferes with the legal practitioner’s primary occupation of
  practising as a lawyer.''
\end{itemize}

The astute reader will have remarked that the treatment in both cases is
slightly different, and this is not related to the particular semantics of
\emph{subject to} and \emph{despite}: we can state defeasibility
\begin{itemize}
\item either in the form of (negated) preconditions of rules: ``rule $r_1$ is
  applicable if the preconditions of $r_2$ do not hold'';
\item or in the form of (negated) derivability of the postcondition of rules: ``rule $r_1$ is
  applicable if the postcondition of $r_2$ does not hold''.
\end{itemize}
We will subsequently come back to this difference\remms{where?}.

Before looking at a formalization, let us summarize this informal exposition
of defeasibility rule modifiers as follows:
\remms{Make the writing of the modifiers more homogenous: in italics or in quotes}
\begin{itemize}
\item ``$r_1$ subject to $r_2$'' and ``$r_1$ despite $r_2$'' are complementary
  ways of expressing that one rule may override the other rule. They have in
  common that $r_1$ and $r_2$ have contradicting conclusions. The conjunction
  of the conclusions can either be directly unsatisfiable (may accept vs.{}
  must not accept) or unsatisfiable \wrt{} an intended background theory
  (obtaining different maximal speeds is inconsistent when expecting
  \texttt{maxSp} to be functional in its fourth argument).
\item Both modifiers differ in that ``subject to'' modifies the rule to which
  it is attached, whereas ``despite'' has a remote effect.
\item They permit to structure a legal text, favoring conciseness and
  modularity: In the case of \emph{despite}, the overridden, typically more
  general rule need not be aware of the overriding, typically subordinate rules.
\item Even though these modifiers appear to be mechanisms on the meta-level in
  that they reasoning about rules, they can directly be reflected on the
  object-level.
\end{itemize}
Making the meaning of the modifiers explicit can therefore be understood as a
\emph{compilation} problem, which will be described in the following.

In L4, rule modifiers are introduced with the aid of \emph{rule annotations}, with a
list of rule names following the keywords \texttt{subjectTo} and
\texttt{despite}. We return to our running example and modify rule
\texttt{maxSpCarHighway} of \figref{fig:rules} with

\begin{lstlisting}
rule <maxSpCarHighway>
  {restrict: {despite: maxSpCarWorkday}}
# rest of rule unchanged
\end{lstlisting}

For the delight of the public of the country with the highest density of
sports cars, we also introduce a new rule:\remms{Problem with spaces in lstlisting}

\begin{lstlisting}
rule <maxSpSportsCar>
  {restrict: {subjectTo: maxSpCarWorkday, 
              despite: maxSpCarHighway}}
   for v: Vehicle, d: Day, r: Road
   if isSportsCar v && isHighway r
   then maxSp v d r 320
 \end{lstlisting}

In the following, we will examine the interaction of these rules.

%%% Local Variables:
%%% mode: latex
%%% TeX-master: "main"
%%% End:
