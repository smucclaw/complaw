\section{Introduction}\label{sec:introduction}

Computer-supported reasoning about law is a longstanding effort of researchers
from different disciplines such as jurisprudence, artificial intelligence, logic and
philosophy. What originally may have appeared as an academic playground is
now evolving into a realistic scenario, for different reasons. 

On the \emph{demand} side, there is a growing number of human-machine or
machine-machine interactions where compliance with legal norms is essential,
such as in sales, insurance, banking or digital rights management, to name but
a few. Innumerable ``smart contract'' languages attest of the interest to
automate these processes, even though many of them are rather dedicated
programming languages than formalisms intended to express and reason about
regulations.

On the \emph{supply} side, decisive advances have been made in fields such as
automated reasoning and language technologies, both for computerised domain
specific languages (DSLs) and natural languages. Even though a completely
automated processing of traditional law texts capturing the subtleties of
natural language is currently out of scope, one can expect to code a law text
in a DSL that is amenable to further processing.

This ``rules as code'' approach is the working hypothesis of our CCLAW
project\footnote{\url{https://cclaw.smu.edu.sg/}}: law texts are formalised in
a DSL called L4 that is sufficiently precise to avoid ambiguities of natural
languages and at the same time sufficiently close to a traditional law text
with its characteristic elements such as cross references, prioritisation of
rules and defeasible reasoning. Indeed, presenting these features is one of
the main topics of this paper. Once a law has been coded in L4, it can
be further processed: it can be converted to natural language \cite{inari} to
be as human readable as a traditional law and efficient executable code can be
extracted, for example to perform tax calculations (all this is not the topic
of the present paper). It can also be analysed, to find faults in the law
text on the meta level (such as consistency and completeness of a rule set),
but also on the object level, to decide individual cases.

\paragraph{Structure of the paper}




\paragraph{Related work}
ctd

\remms[inline]{TBD}

\begin{itemize}
\item For bibliographical references, see
  \footnote{\url{https://law.stanford.edu/publications/developing-a-legal-specification-protocol-technological-considerations-and-requirements/}}
\item \cite{sergot_kowalski_etal__british_nationality_acm_1986,kowalski_legislation_logic_programs_1995}
\item \cite{libal_steen_nai_suite_draft_reason_legal_texts_jurix_2019}
\item \cite{governatori_carnead_defeas_logic_icail_2011}
\end{itemize}



%%% Local Variables:
%%% mode: latex
%%% TeX-master: "main"
%%% End:
