\section{Introduction}\label{sec:introduction}

Computer-supported reasoning about law is a longstanding effort of researchers
from different disciplines such as jurisprudence, artificial intelligence, logic and
philosophy. What originally may have appeared as an academic playground is
now evolving into a realistic scenario, for various reasons. 

On the \emph{demand} side, there is a growing number of human-machine or
machine-machine interactions where compliance with legal norms or with a contract is essential,
such as in sales, insurance, banking and finance or digital rights management, to name but
a few. Innumerable ``smart contract'' languages attest of the interest to
automate these processes, even though many of them are rather dedicated
programming languages than formalisms intended to express and reason about
regulations.

On the \emph{supply} side, decisive advances have been made in fields such as
automated reasoning and language technologies, both for computerised domain
specific languages (DSLs) and natural languages. Even though a completely
automated processing of traditional law texts capturing the subtleties of
natural language is currently out of scope, one can expect to code a law text
in a DSL that is amenable to further processing.

This ``rules as code'' approach is the working hypothesis of our CCLAW
project\footnote{\url{https://cclaw.smu.edu.sg/}}: law texts are formalised in
a DSL called L4 that is sufficiently precise to avoid ambiguities of natural
languages and at the same time sufficiently close to a traditional law text
with its characteristic elements such as cross references, prioritisation of
rules and defeasible reasoning. Indeed, presenting these features is one of
the main topics of this paper. Once a law has been coded in L4, it can
be further processed: it can be converted to natural language \cite{listemnmaa2021cnl} to
be as human readable as a traditional law text, and efficient executable code can be
extracted, for example to perform tax calculations (all this is not the topic
of the present paper). It can also be analysed, to find faults in the law
text on the meta level (such as consistency and completeness of a rule set),
but also on the object level, to decide individual cases.

\paragraph{Overview of the paper}
The main emphasis of this paper is on the L4 DSL that is currently under
definition, which in particular features a formalism for transcribing rules
and reasoning support for verifying their properties. \secref{sec:l4_language} is dedicated to a description of the language
and the L4 system implementation currently under development. The rule
language will be dissected in \secref{sec:resasoning_with_rules}. We will in
particular describe mechanisms for prioritisation and defeasibility of rules
that are encoded via specific keywords in law texts. We then define a precise
semantics of these mechanisms, by a translation to logic. Classical, monotonic
logic, developed in \secref{sec:defeasible_classical}, has received surprisingly
little attention in this area, even though proof support in the form of
SAT/SMT solvers has made astounding progress in recent years. An alternative
approach, based on Answer Set Programming, is described in
\secref{sec:defeasible_asp}. We conclude in \secref{sec:conclusions}.


\paragraph{Related work}

There is a huge body of work both on computer-assisted legal reasoning and
(not necessarily related) defeasible reasoning. Sergot and Kowalski
\cite{sergot_kowalski_etal__british_nationality_acm_1986,kowalski_legislation_logic_programs_1995}
code the British Nationality Act in Prolog, exploiting Prolog's negation as
failure for default reasoning.



The Catala language \cite{merigoux_chataing_protzenko_cata_icfp_2021},
extensively used for coding tax law and resembling more a high-level
programming language than a reasoning formalism, includes default rules, which
are however not entirely disambiguated during compile time so that run time
exceptions can be raised.

An entirely different approach to tool support is taken with the LogiKEy \cite{benzmueller_etal_logikey_2020}
workbench that codes legal reasoning in the Isabelle interactive proof assistant, paving
the way for a very expressive formalism. In contrast, we have opted for a DSL
with fully automated proofs which are provided by SMT respectively ASP solvers. These do not permit for human intervention in the proof process, which would not be adequate for the user group we target. Symboleo
\cite{sarifi_parvizimosaed_amyot_logrippo_mylopoulos_Symboleo_spec_legal_contracts_2020}
and the NAI Suite
\cite{libal_steen_nai_suite_draft_reason_legal_texts_jurix_2019} emphasise
deontic logic rather than defeasible reasoning (the former is so far not
considered in our L4).

As a result of a long series of logics, see for example
\cite{governatori_carnead_defeas_logic_icail_2011,governatori21:_unrav_legal_refer_defeas_deont_logic},
Governatori and colleagues have developed the Turnip
system\footnote{\url{https://turnipbox.netlify.com/}} that is based on a
combination of defeasible and deontic logic. The system is applied, among
others, to modelling traffic rules
\cite{governatori_Traffic_Rules_Encoding_using_Defeasible_jurix_2020}.

It seems vain to attempt an exhaustive
review of defeasible reasoning. Before the backdrop of foundational law
theory \cite{hart_concept_of_law_1997}, there are sometimes diverging
proposals for integrating defeasibility, sometimes opting for non-monotonic
logics \cite{hage_law_and_defeasibility_2003}, sometimes taking a more
classical stance \cite{alchourron_makinson_hierarchies_of_regulations_1981}. 
Defeasible rule-based reasoning in the context of argumentation theory is
discussed in \cite{dung_argumentation_theory_1995,amgoud_besnard_rule_based_argumentation_systems_2019}.

On a more practical side, Answer Set Programming (ASP) \cite{asp_background}\remms{find
  some more representative refs, also in high-level journals}
goes beyond logic programming and increasingly integrates techniques from
constraint solving, such as in the sCASP\remms{introduce macro?} system
\cite{arias_phd_2019}. In spite of a convergence of SMT and CASP technologies,
there are few attempts to use SMT for ASP, see
\cite{shen_lierler_smt_answer_set_kr_2018}. For the technologies used in our
own implementation, please see \secref{sec:conclusions}. 




%%% Local Variables:
%%% mode: latex
%%% TeX-master: "main"
%%% End:
