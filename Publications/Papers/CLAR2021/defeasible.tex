\section{Defeasible Reasoning}\label{sec:defeasible}

% ----------------------------------------------------------------------
\subsection{Facets of Defeasible Reasoning}\label{sec:facets}

\begin{tcolorbox}[title=To be done]
Discussion on existing notions of defeasibility in particular in a legal context
\begin{itemize}
\item \cite{hart_concept_of_law_1997}
\item \cite{alchourron_makinson_hierarchies_of_regulations_1981}
\item \cite{hage_law_and_defeasibility_2003}
\item More recent: \cite{governatori21:_unrav_legal_refer_defeas_deont_logic}
\end{itemize}
\end{tcolorbox}

Given a plethora of different notions of ``defeasibility'', we had to make a
choice as to which notions to support, and which semantics to give to them. We
will here concentrate on two concepts, which we call \emph{rule modifiers},
that limit the applicability of rules and make them ``defeasible''. They will
we be presented in \secref{sec:rule_modifiers}, and we will see how to give
them a semantics in a classical first-order logic, but also in a non-monotonic
logic in a system based on Answer Set Programming
(\secref{sec:answer_set_programming}). An orthogonal question is that of
arriving at conclusion in absence of complete information, which, via the
mechanism of negation as \remms{as/by??} failure, often prones the use of
non-monotonic logics. In \secref{sec:rule_inversion}, we will see how similar
effects can be achieved by means of \emph{rule inversion}. An Answer Set
Programming approach to defeasible reasoning, with an emphasis on rule
modifiers, will be presented in \secref{sec:answer_set_programming}. We finish
with a comparison of the two strands of reasoning in \secref{sec:comparison}.


% ----------------------------------------------------------------------
\subsection{Introducing Rule Modifiers}\label{sec:intro_rule_modifiers}

We will concentrate on two rule modifiers that restrict the applicability of
rules and that frequently occur in law texts: \emph{subject to} and
\emph{despite}. To illustrate their use, we consider an excerpt of Singapore's
Professional Conduct Rules \S~34 \cite{professional_conduct_rules} (also see
\cite{morris21:_const_answer_set_progr_tool} for a more detailed treatment of
this case study):

\begin{description}
\item[(1)] A legal practitioner must not accept any executive appointment associated with any of the following businesses:
  \begin{description}
  \item[(a)] any business which detracts from, is incompatible with, or
    derogates from the dignity of, the legal profession;
  \item[(b)] any business which materially interferes with the legal
    practitioner’s primary occupation of practising as a lawyer; (\dots)
  \end{description}
\item[(5)] Despite paragraph (1)(b), but subject to paragraph (1)(a) and (c)
  to (f), a locum solicitor may accept an executive appointment in a business
  entity which does not provide any legal services or law-related services, if
  all of the conditions set out in the Second Schedule are satisfied.
\end{description}

The two main notions developed in the Conduct Rules are which executive appointments a legal
practictioner \emph{may} or \emph{must not} accept under which
circumstances. As there is currently no direct support for deontic logics in
L4, these notions are defined as two predicates \texttt{MayAccept} and
\texttt{MustNotAccept}, with the intended meaning that these two notions are
contradictory, and this is indeed what will be provable after a complete
formalization.

Let us here concentrate on the modifiers \emph{despite} and \emph{subject
  to}. A synonym of ``despite'' that is often used in legal texts is
``notwithstanding'',  and a synonym of
``subject to'' is ``except as provided in'', see \cite{adams_contract_drafting_2004}.

The reading of rule (5) is the following:
\begin{itemize}
\item ``subject to paragraph (1)(a) and (c) to (f)'' means: rule (5) applies
  as far as (1)(a) and (c) to (f) is not established. Differently said, rules
  (1)(a) and (c) to (f) undercut or defeat rule (5).

  One way of explicitating the ``subject to'' clause would be to rewrite (5)
  to: ``Despite paragraph (1)(b), provided the business does not detract from,
  is incompatible with, or derogate from the dignity of, the legal profession;
  and provided that not [clauses (1)(c) to (f)]; then a locum
  sollicitor\footnote{in our class-based terminology, a subclass of legal
    practitioner} may accept an executive appointment.''

\item ``despite paragraph (1)(b)'' expresses that rule (5) overrides rule
  (1)(b). In a similar spirit as the ``subject to'' clause, this can be made
  explicit by introducing a proviso, however not locally in  rule (5), but
  remotely in rule (1)(b).

  One way of explicitating the ``despite'' clause of rule (5) would be to
  rewrite (1)(b) to: ``Provided (5) is not applicable, a legal practitioner
  must not accept any executive appointment associated with any business which
  materially interferes with the legal practitioner’s primary occupation of
  practising as a lawyer.''
\end{itemize}

The astute reader will have remarked that the treatment in both cases is
slightly different, and this is not related to the particular semantics of
\emph{subject to} and \emph{despite}: we can state defeasibility
\begin{itemize}
\item either in the form of (negated) preconditions of rules: ``rule $r_1$ is
  applicable if the preconditions of $r_2$ do not hold'';
\item or in the form of (negated) derivability of the postcondition of rules: ``rule $r_1$ is
  applicable if the postcondition of $r_2$ does not hold''.
\end{itemize}
We will subsequently come back to this difference\remms{where?}.

Before looking at a formalization, let us summarize this informal exposition
of defeasibility rule modifiers as follows:
\remms{Make the writing of the modifiers more homogenous: in italics or in quotes}
\begin{itemize}
\item ``$r_1$ subject to $r_2$'' and ``$r_1$ despite $r_2$'' are complementary
  ways of expressing that one rule may override the other rule. They have in
  common that $r_1$ and $r_2$ have contradicting conclusions. The conjunction
  of the conclusions can either be directly unsatisfiable (may accept vs.{}
  must not accept) or unsatisfiable \wrt{} an intended background theory
  (obtaining different maximal speeds is inconsistent when expecting
  \texttt{maxSp} to be functional in its fourth argument).
\item Both modifiers differ in that ``subject to'' modifies the rule to which
  it is attached, whereas ``despite'' has a remote effect.
\item They permit to structure a legal text, favoring conciseness and
  modularity: In the case of \emph{despite}, the overridden, typically more
  general rule need not be aware of the overriding, typically subordinate rules.
\item Even though these modifiers appear to be mechanisms on the meta-level in
  that they reasoning about rules, they can directly be reflected on the
  object-level.
\end{itemize}
Making the meaning of the modifiers explicit can therefore be understood as a
\emph{compilation} problem, which will be described in the following.

In L4, rule modifiers are introduced with the aid of \emph{rule annotations}, with a
list of rule names following the keywords \texttt{subjectTo} and
\texttt{despite}. We return to our running example and modify rule
\texttt{maxSpCarHighway} of \figref{fig:rules} with

\begin{lstlisting}
rule <maxSpCarHighway>
  {restrict: {despite: maxSpCarWorkday}}
# rest of rule unchanged
\end{lstlisting}

For the delight of the public of the country with the highest density of
sports cars, we also introduce a new rule:\remms{Problem with spaces in lstlisting}

\begin{lstlisting}
rule <maxSpSportsCar>
  {restrict: {subjectTo: maxSpCarWorkday, 
              despite: maxSpCarHighway}}
   for v: Vehicle, d: Day, r: Road
   if isSportsCar v && isHighway r
   then maxSp v d r 320
 \end{lstlisting}

In the following, we will examine the interaction of these rules.

% ----------------------------------------------------------------------
\subsection{Rule Modifiers in Classical Logic}\label{sec:rule_modifiers_in_classical_logic}

In this section, we will describe how to rewrite rules, progressively
eliminating the instructions appearing in the rule annotations so that in the
end, only purely logical rules remain. Whereas the first preprocessing steps
(\secref{sec:preprocessing}) are generic, we will discuss two variants of
conversion into logical format (\secrefs{sec:restr_precond} and
\ref{sec:restr_deriv}).



% ......................................................................
\subsubsection{Preprocessing}\label{sec:preprocessing}

Preprocessing consists of several elimination steps that are carried out in a
fixed order.

\paragraph{\textbf{``Despite''  elimination}}

As can be concluded from the discussion in \secref{sec:facets}, a
$\mathtt{despite}\; r_2$ clause appearing in rule $r_1$ is equivalent to a
$\mathtt{subjectTo}\; r_1$ clause in rule $r_2$. The first rule transformation
consists in applying exhaustively the following \emph{despite elimination}
rule transformer:
\remms[inline]{In the whole discussion (and the implementation), make a
  clearer distinction between rule set transformer \texttt{restrict} and rule
  generator / transformer \texttt{derived}.}

\noindent
\emph{despiteElim:}\\
$
\{r_1 \{\mathtt{restrict}: \{\mathtt{despite}\; r_2\} \uplus a_1\},\;\;
r_2\{\mathtt{restrict}: a_2\}, \dots\} \longrightarrow$\\
$\{r_1 \{\mathtt{restrict}: a_1\},\;\;
r_2\{\mathtt{restrict}:  \{\mathtt{subjectTo}\; r_1\} \uplus a_2\}, \dots\}
$

\begin{example}\label{ex:rewrite_despite}\mbox{}\\
Application of this rewrite rule to the three example rules \texttt{maxSpCarWorkday},
\texttt{maxSpCarHighway} and  \texttt{maxSpSportsCar} changes them to:

\begin{lstlisting}
rule <maxSpCarWorkday>
  {restrict: {subjectTo: maxSpCarHighway}}
rule <maxSpCarHighway>
  {restrict: {subjectTo: maxSpSportsCar}}
rule <maxSpSportsCar>
  {restrict: {subjectTo: maxSpCarWorkday}}
\end{lstlisting}
Here, only the headings are shown, the bodies of the rules are
unchanged. 
\end{example}

One defect of the rule set already becomes apparent to the human reader at
this point: the circular dependency of the rules. We will however continue
with our algorithm, applying the next step which will be to rewrite the
\texttt{\{restrict: \{subjectTo: \dots\}\}} clauses.  Please note that each
rule can be \texttt{subjectTo} several other rules, each of which may have a
complex structure as a result of transformations that are applied to it.


\paragraph{\textbf{``Subject'' To elimination}}

The rule transformer \emph{subjectToElim} does the following: it splits up the
rule into two rules: its source (the rule body as originally given), and its
definition as the result of applying a rule transformation function to several rules.

\begin{example}\label{ex:rewrite_subject_to}\
Before stating the rule transformer, we show its effect on rule
\texttt{maxSpCarWorkday} of \exampleref{ex:rewrite_despite}. On rewriting
with \emph{subjectToElim}, the rule is transformed into two rules:

\begin{lstlisting}
# new rule name, body of rule unchanged
rule <maxSpCarWorkday'Orig>
   {source}
   for v: Vehicle, d: Day, r: Road
   if isCar v && isWorkday d
   then maxSp v d r 90

# rule with header and without body
rule <maxSpCarWorkday>
 {derived: {apply: 
 {restrictSubjectTo maxSpCarWorkday'Orig  maxSpSportsCar}}}
\end{lstlisting}
\remms[inline]{check syntax of apply}
\end{example}

We can now state the transformation (after grouping the
\texttt{subjectTo $r_2$}, \dots, \texttt{subjectTo $r_n$} into
\texttt{subjectTo $[r_2 \dots r_n]$}):


\noindent
\emph{subjectToElim:}\\
$
\{r_1 \{\mathtt{restrict}: \{\mathtt{subjectTo}\; [r_2, \dots, r_n]\}\}, \dots \} \longrightarrow$\\
$\{r_1^o \{\mathtt{source}\}, r_1 \{\mathtt{derived:}\; \{\mathtt{apply:}\; \{
\mathtt{restrictSubjectTo}\;\; r_1^o\; [r_2 \dots r_n] \}\}\}, \dots \}
$



\paragraph{\textbf{Computation of derived rule}}
The last step consists in generating the derived rules, by evaluating the
value of the rule transformer expression marked by \texttt{apply}. The rules
appearing in these expressions may themselves be defined by complex
expressions. However, direct or indirect recursion is not allowed. For
simplifying the expressions in a rule set, we compute a rule dependency order
$\preceq_R$ defined by: $r \preceq_R r'$ iff $r$ appears in the defining
expression of $r'$. If $\preceq_R$ is not a partial order (in particular, if
it is not cycle-free), then evaluation fails. Otherwise, we order the rules
topologically by $\preceq_R$ and evaluate the expressions starting from the
minimal elements.

\begin{example}
It is at this point that the cyclic dependence already remarked after
\exampleref{ex:rewrite_despite} will be discovered. We have:

\noindent
\texttt{maxSpCarWorkday'Orig}, \texttt{maxSpCarHighway} $\preceq_R$ \texttt{maxSpCarWorkday}\\
\texttt{maxSpCarHighway'Orig}, \texttt{maxSpSportsCar} $\preceq_R$ \texttt{maxSpCarHighway}\\
\texttt{maxSpSportsCar'Orig}, \texttt{maxSpCarWorkday} $\preceq_R$  \texttt{maxSpSportsCar}\\
\noindent
which cannot be totally ordered.

Let us fix the problem by changing the heading of rule
\texttt{maxSpCarHighway} from \texttt{despite} to \texttt{subjectTo}:
\begin{lstlisting}
rule <maxSpCarHighway>
  {restrict: {subjectTo: maxSpCarWorkday}}
\end{lstlisting}

After rerunning \emph{despiteElim} and \emph{subjectToElim}, we can now order
the rules:

% \begin{lstlisting}
% rule <maxSpCarWorkday>
% rule <maxSpCarHighway>
%   {restrict: {subjectTo: maxSpCarWorkday, maxSpSportsCar}}
% rule <maxSpSportsCar>
%   {restrict: {subjectTo: maxSpCarWorkday}}
% \end{lstlisting}

\noindent
\{ \texttt{maxSpSportsCar'Orig}
\texttt{maxSpCarHighway'Orig},
\texttt{maxSpCarWorkday} \} $\preceq_R$
\texttt{maxSpSportsCar} $\preceq_R$
\texttt{maxSpCarHighway}\\
and will use this order for rule elaboration.
\end{example}


% ......................................................................
\subsubsection{Restriction via Preconditions}\label{sec:restr_precond}

Here, we propose one possible implementation of the rule transformer
\texttt{restrictSubjectTo} introduced in \secref{sec:preprocessing} that takes
a rule $r_1$ and a list of rules $[r_2 \dots r_n]$ and produces a new rule, by
adding the negation of the preconditions of $[r_2 \dots r_n]$ to $r_1$. More
formally:

\begin{itemize}
\item $\mathtt{restrictSubjectTo}\; r_1\; [] = r_1$
\item $\mathtt{restrictSubjectTo}\; r_1\; (r' \uplus rs) =$\\
  $\mathtt{restrictSubjectTo}\; (r_1(precond := precond(r_1) \AND \NOT precond(r')))\; rs$
\end{itemize}
where $precond(r)$ selects the precondition of rule $r$ and $r(precond:=p)$
updates the precondition of rule $r$ with $p$.

There is one proviso to the application of \texttt{restrictSubjectTo}: the
rules have to have the same \emph{parameter interface}: the number and types
of the parameters in the rules' \texttt{for} clause have to be the same.
Rules with different parameter interfaces can be adapted via the
\texttt{remap} rule transformer.


%========================================

% R1: 
% for x1:Tx1, x2: Tx2. body1(x1, x2)

% R2:
% for y1:Ty1, y2: Ty2, y3: Ty3. body2(y1, y2, y3)


% remap R2 [x1:Tx1, x2: Tx2] [y1 := x1, y2 := x2, y3 := e]
% yields:
% for x1:Tx1, x2: Tx2.
% body2(x1, x2, e)

% remap R1 [y1:Ty1, y2: Ty2, y3] [x1 := y1, x2:= y2]
% yields:
% for y1:Ty1, y2: Ty2, y3: Ty3.
% body1(y1, y2)


%========================================



% ......................................................................
\subsubsection{Restriction via Derivability}\label{sec:restr_deriv}



% ----------------------------------------------------------------------
\subsection{Rule Inversion}\label{sec:rule_inversion}


% ----------------------------------------------------------------------
\subsection{Answer Set Programming}\label{sec:answer_set_programming}

\remms[inline]{section probably does not have an adequate name}

% ----------------------------------------------------------------------
\subsection{Comparison}\label{sec:comparison}





%%% Local Variables:
%%% mode: latex
%%% TeX-master: "main"
%%% End:
