\section{Conclusions}\label{sec:conclusions}

This paper has discussed different approaches for representing defeasibility
as used in law texts, by annotating rules with modifiers that explicate their
relation to other rules. We have notably presented two encodings in classical
logic (\secrefs{sec:restr_precond} and \ref{sec:restr_deriv}) and explained
how they are related (\secref{sec:comparison}). Quite a different approach,
based on Answer Set Programming, is presented in
\secref{sec:defeasible_asp}.

An experimental implementation of the L4 ecosystem is under
way\footnote{\url{https://github.com/smucclaw/baby-l4}}, but it has not yet
reached a stable, user-friendly status. It features an IDE based on VS Code
and natural language processing via Grammatical
Framework\cite{ranta_grammatical_2004}. Currently, only the coding of
\secrefs{sec:restr_precond} has been implemented -- the coding of
\secref{sec:restr_deriv} would require a much more profound transformation of
rules and assertions. Likewise, the ASP coding of \secref{sec:defeasible_asp}
has only been done manually. The interaction with SMT solvers is done through
an SMT-LIB \cite{BarFT_SMTLIB} interface, thus opening the possibility to
interact with a wide range of solvers. As our rules typically contain
quantification, reasoning with quantifiers is crucial, and the
best support currently seems to be provided by Z3 \cite{demoura_bjorner_z3_2008}.

We are still at the beginning of the journey. A theoretical comparison of the
classical and ASP approaches presented here still has to be carried out, and
it has to be propped up by an empirical evaluation. For this purpose, we are
currently in the process of coding some real-life law texts in L4. We are
fully aware of shortcomings of the current L4, which will strongly evolve in
the next months, to include reasoning about deontics and about temporal
relations. Integrating these aspects will not be easy, and this is one reason
for not committing prematurely on one particular logical framework.


%%% Local Variables:
%%% mode: latex
%%% TeX-master: "main"
%%% End:
