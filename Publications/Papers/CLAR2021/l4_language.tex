\section{The L4 Language}\label{sec:l4_language}

This section gives an account of the L4 language as it is currently defined --
as an experimental language, L4 will evolve over the next months. In our
discussion, we will ignore some features such as a natural language interface
\cite{} that are not relevant for the topic of this paper.

As a language intended for representing legal texts and reasoning about them,
an L4 module is essentially composed of four sections:
\begin{itemize}
\item a terminology in the form of \emph{class definitions};
\item \emph{declarations} of functions and predicates;
\item \emph{rules} representing the core of a law text, specifying what is
  considered as legal behaviour;
\item \emph{assertions} for stating and proving properties about the rules.
\end{itemize}

In the following, these elements will be presented in more detail and
illustrated with a running example, a (fictitious) reglementation of speed
limits for different types of vehicles.


% ----------------------------------------------------------------------
\subsection{Class Definitions}\label{sec:classdefs}

The definition in \figref{fig:classdefs} introduces classes for vehicles, days
and roads. 

\begin{figure}[h]
  \begin{framed}
\begin{alltt}
class Vehicle \{
   weight: Integer
\}
class Car extends Vehicle \{
   doors: Integer
\}
class Truck extends Vehicle

class Day
class Workday extends Day
class Holiday extends Day

class Road
class Highway extends Road
\end{alltt}
\end{framed}
  \caption{Class definitions of speedlimit example}\label{fig:classdefs}
\end{figure}

Classes are arranged in a tree-shaped hierarchy, having a class named
\texttt{Class} as its top element. Classes that are not explicitly derived
from another class via \texttt{extends} are implicitly derived from
\texttt{Class}. A class $S$ derived from a class $C$ by \texttt{extends} will
be called a subclass of $C$, and the immediate subclasses of \texttt{Class}
will be called \emph{sorts} in the following. Intuitively, classes are meant
to be sets of entitities, with subclasses being interpreted as
subsets. Different subclasses of a class are not meant to be disjoint.

Class definitions can come with attributes, in braces. These attributes can be
of simple type, as in the given example, or of higher type (the notion of type
will be explained in \secref {sec:fundecls}). In a declarative reading,
attributes can be understood as a shorthand for function declarations that
have the class they are defined in as additional codomain. Thus, the attribute
\texttt{weight} corresponds to a top-level declaration \texttt{weight: Vehicle
  -> Integer}. In a more operational reading, L4 classes can be understood as
prototypes of classes in object-oriented programming languages, and an
alternative field selection syntax can be used: For \texttt{v: Vehicle}, the
expression \texttt{v.weight} is equivalent to \texttt{weight(v)}, at least
logically, even though its operational interpretation may differ.


% ----------------------------------------------------------------------
\subsection{Function Declarations}\label{sec:fundecls}

L4 is an \emph{explicitely} and \emph{strongly typed} language: all entities
such as functions, predicates and variables have to be declared before being
used. One purpose of this measure is to ensure that the executable sublanguage
of L4, based on the simply-typed lambda calculus with subtyping, enjoys a type
soundness property: evaluation of a function cannot produce a dynamic type
error.



\begin{figure}[h]
\begin{framed}

\begin{alltt}
\end{alltt}

\end{framed}
  \caption{Declarations of speedlimit example}\label{fig:fundecls}
\end{figure}




% ----------------------------------------------------------------------
\subsection{Rules}\label{sec:rules}

\begin{figure}[h]
\begin{framed}

\begin{alltt}
\end{alltt}

\end{framed}
  \caption{Rules of speedlimit example}\label{fig:rules}
\end{figure}



% ----------------------------------------------------------------------
\subsection{Assertions}\label{sec:assertions}

\begin{figure}[h]
\begin{framed}

\begin{alltt}
\end{alltt}

\end{framed}
  \caption{Assertions of speedlimit example}\label{fig:assertions}
\end{figure}




%%% Local Variables:
%%% mode: latex
%%% TeX-master: "main"
%%% End:
