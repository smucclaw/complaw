\section{An Overview of the L4 Language: Details}\label{sec:l4_language_app}

Let us give some more details about the L4 language: its class and type
definition mechanism, and the way it handles proof obligations.

% ----------------------------------------------------------------------
\subsection{Terminology and Class Definitions}\label{sec:classdefs}

The definition in \figref{fig:classdefs} introduces classes for vehicles, days
and roads. 

\begin{figure}[h!]
%\begin{mdframed}
\begin{lstlisting}
class Vehicle {
   weight: Integer
}
class Car extends Vehicle {
   doors: Integer
}
class Truck extends Vehicle
class SportsCar extends Car

class Day
class Workday extends Day
class Holiday extends Day

class Road
class Highway extends Road
\end{lstlisting}
%\end{mdframed}
  \caption{Class definitions of speedlimit example}\label{fig:classdefs}
\end{figure}

Classes are arranged in a tree-shaped hierarchy, having a class named
\texttt{Class} as its top element. Classes that are not explicitly derived
from another class via \texttt{extends} are implicitly derived from
\texttt{Class}. A class $S$ derived from a class $C$ by \texttt{extends} will
be called a subclass of $C$, and the immediate subclasses of \texttt{Class}
will be called \emph{sorts} in the following. Intuitively, classes are meant
to be sets of entitities, with subclasses being interpreted as
subsets. Different subclasses of a class are not meant to be disjoint.

Class definitions can come with attributes, in braces. These attributes can be
of simple type, as in the given example, or of higher type (the notion of type
will be explained in \secref {sec:fundecls}). In a declarative reading,
attributes can be understood as a shorthand for function declarations that
have the class they are defined in as additional domain. Thus, the attribute
\texttt{weight} corresponds to a top-level declaration \texttt{weight: Vehicle
  -> Integer}. In a more operational reading, L4 classes can be understood as
prototypes of classes in object-oriented programming languages, and an
alternative field selection syntax can be used: For \texttt{v: Vehicle}, the
expression \texttt{v.weight} is equivalent to \texttt{weight(v)}, at least
logically, even though the operational interpretations may differ.


% ----------------------------------------------------------------------
\subsection{Types and Function Declarations}\label{sec:fundecls}

L4 is an \emph{explicitly} and \emph{strongly typed} language: all entities
such as functions, predicates and variables have to be declared before being
used. One purpose of this measure is to ensure that the executable sublanguage
of L4, based on the simply-typed lambda calculus with subtyping, enjoys a type
soundness property: evaluation of a function cannot produce a dynamic type
error.

\figref{fig:fundecls} shows two function declarations. Functions with
\texttt{Boolean} result type will sometimes be called \emph{predicates} in the
following, even though there is no syntactic difference. All the declared
classes are considered as elementary types, as well as \texttt{Integer},
\texttt{Float}, \texttt{String} and \texttt{Boolean} (which are internally also treated as
classes). If $T_1, T_2, \dots T_n$ are types, then so are function types
\texttt{$T_1$ -> $T_2$} and tuple types \texttt{($T_1$, $\dots$ ,$T_n$)}. The
type system and the expression language, to be presented later, are
higher-order, but extraction to some solvers (\secref{}) will be limited to
(restricted) first-order theories.

\begin{figure}[h]
%\begin{mdframed}
\begin{lstlisting}
decl isCar : Vehicle -> Boolean
decl maxSp : Vehicle -> Day -> Road -> Integer -> Boolean
\end{lstlisting}
%\end{mdframed}
  \caption{Declarations of speedlimit example}\label{fig:fundecls}
\end{figure}

The nexus between the terminological and the logical level is established with
the aid of \emph{characteristic predicates}. Each class $C$ which is a
subclass of sort $S$ gives rise to a declaration \texttt{is$C$: $S$ ->
  Boolean}. An example is the declaration of \texttt{isCar} in
\figref{fig:fundecls}. In the L4 system, this declaration, as well as the
corresponding class inclusion axiom (\secref{sec:rules}), are generated
automatically.\remms{Promise}

Two classes derived from the same base class (thus: \texttt{$C_1$ extends $B$}
and \texttt{$C_2$ extends $B$}) are not necessarily disjoint. 

From the subclass relation, a \emph{subtype} relation $\preceq$ can be defined
inductively as follows: if \texttt{$C$ extends $B$}, then $C \preceq B$, and
for types $T_1, \dots, T_n, T_1', \dots, T_n'$,
if $T_1 \preceq T_1', \dots, T_n \preceq T_n'$, 
then \texttt{$T_1'$ -> $T_2 \; \preceq \; T_1$ -> $T_2'$} 
and \texttt{($T_1$, $\dots$, $T_n$) $\preceq$ ($T_1'$, $\dots$, $T_n'$)}.

Without going into details of the type system, let us remark that it has been
designed to be compatible with subtyping: if an element of a type is
acceptable in a given context, then so is an element of a subtype. In
particular,
\begin{itemize}
\item for field selection, if $C'$ is a class having field $f$ of type $T$,
  and $C \preceq C'$, and $c : C$, then field selection is well-typed with $c.f : T$.
\item for function application, if $f: A' \mbox{\texttt{->}} B$ and $a:A$ and
  $A \preceq A'$, then function application is well-typed with $f\; a : B$.
\end{itemize}



% ----------------------------------------------------------------------
\subsection{Assertions}\label{sec:assertions}


Assertions are statements that the L4 system is meant to verify or to
reject -- differently said, they are proof obligations. These assertions are verified
relative to a rule set comprising some or all of the rules and facts stated
before. 


The active rule set used for verification can be configured, by adding rules
to or deleting rules from a default set. Assume the active rule set consists
of $n$ rules whose logical representation is $R_1 \dots R_n$, and assume the
formula of the assertion is $A$. The proof obligation can then be checked for
\begin{itemize}
\item  \emph{satisfiability}: in this case, $R_1 \AND \dots \AND R_n \AND A$
  is checked for satisfiability.
\item \emph{validity}: in this case, $R_1 \AND \dots \AND R_n \IMPL A$ is
  checked for validity.
\end{itemize}
In either case, if the proof fails, a model resp.{} countermodel is produced.
In the given example, the SMT solver checks the validity of the formula and
indeed returns a countermodel that leads to contradictory prescriptions of the
maximal speed: if the vehicle is a car, the day a workday and the road a
highway, the maximal speed can be 90 or 130, depending on the rule applied.
\remms{Put output of model checker here?}

The assertion \texttt{maxSpFunctional} can be considered an essential
consistency requirement and a rule system violating it is inconsistent \wrt{} the intended semantics of \texttt{maxSp}. One
remedial action is to declare one of the rules as default and the other rule
as overrriding it. In \secref{sec:defeasible}, we will discuss different solutions
implemented in L4.

After this repair action, \texttt{maxSpFunctional} will be provable (under
additional natural conditions described in \secref{sec:rule_inversion}). We can now
continue to probe other consistency requirements, such as exhaustiveness
stating that a maximal speed is defined for every combination of vehicle:

\begin{lstlisting}
assert <maxSpExhaustive>
   exists sp: Integer. maxSp instVeh instDay instRoad sp
\end{lstlisting}

The intended usage scenario of L4 is that by an interplay of proving
assertions and repairing broken rules, one arrives at a rule set satisfying
general principles of coherence, completeness and other, more elusive
properties such as fairness (at most temporary exclusion from essential
resources or rights).\remms{maybe put this into the intro}


%%% Local Variables:
%%% mode: latex
%%% TeX-master: "main"
%%% End:
