\documentclass{beamer}


% Packages from LIPICS

\usepackage{fancyhdr}

\usepackage{amsmath}
\usepackage{amssymb}
\usepackage[T1]{fontenc}
\usepackage[utf8]{inputenc}

\usepackage{alltt}
\usepackage{makeidx}  % allows for indexgeneration
\usepackage{listings}
\usepackage{caption}
%\usepackage{subcaption}  % does not seem to work with llncs
\usepackage{subfig}  % does not seem to work with llncs
%\usepackage[pdftex]{graphicx}
%\usepackage{enumerate}
\usepackage{mathpartir}
%\usepackage{isabelle,isabellesym}
%\isabellestyleit
\usepackage{todonotes}
\usepackage{wrapfig}
\usepackage{framed}
\usepackage{mathtools}
\usepackage{stmaryrd}
\usepackage{soul}

\usepackage{rail}    % For railroad syntax diagrams


\setcounter{secnumdepth}{3}
    
\usepackage{tikz}
\usetikzlibrary{arrows.meta}
\usetikzlibrary{trees}
\usetikzlibrary{graphs}
\usetikzlibrary{arrows}
\usetikzlibrary{decorations.pathmorphing}
\usetikzlibrary{shapes.multipart}
\usetikzlibrary{shapes.geometric}
\usetikzlibrary{shapes.misc}
\usetikzlibrary{calc}
\usetikzlibrary{positioning} 
\usetikzlibrary{fit}
\usetikzlibrary{backgrounds}


\usepackage[colorlinks,hyperindex,bookmarks,linkcolor=blue,citecolor=blue,urlcolor=blue]{hyperref}


%4 mars 2016 setting for Java code template
\renewcommand{\lstlistingname}{Code}


\definecolor{dkgreen}{rgb}{0,0.6,0}
\definecolor{gray}{rgb}{0.5,0.5,0.5}
\definecolor{mauve}{rgb}{0.58,0,0.82}

% \lstset{frame=tb,
%   language=Java,
%   aboveskip=3mm,
%   belowskip=3mm,
%   showstringspaces=false,
%   columns=flexible,
%   basicstyle={\footnotesize\ttfamily},
%   numbers=none,
%   numberstyle=\tiny\color{gray},
%   keywordstyle=\color{blue},
%   commentstyle=\color{dkgreen},
%   stringstyle=\color{mauve},
%   breaklines=true,
%   breakatwhitespace=true,
%   tabsize=2
% }



% Configuration of lstlisting for L4
\lstdefinelanguage{L4}
{morekeywords={
    assert
    , assign
    , chan
    , class
    , clock
    , decl   
    , defn   
    , derivable
    , else 
    , exists
    , extends
    , fact   
    , false
    , for  
    , forall
    , guard
    , if   
    , in
    , init
    , let   
    , lexicon
    , not   
    , process
    , rule
    , state
    , system
    , then
    , trans
    , true 
},    
sensitive=false,
morecomment=[l]{\#},
morestring=[b]",
}


\lstset{frame=tb,
  language=L4,
  aboveskip=3mm,
  belowskip=3mm,
  showstringspaces=false,
  columns=flexible,
  basicstyle={\footnotesize\ttfamily},
  numbers=none,
  numberstyle=\tiny\color{gray},
  keywordstyle=\color{blue},
  commentstyle=\color{dkgreen},
  stringstyle=\color{mauve},
  breaklines=true,
  breakatwhitespace=true,
  tabsize=2,
  keepspaces=true
}

\lstdefinelanguage{L4Sugary}
{  morekeywords={EVERY,UNLESS,UPON,WITHIN,MUST,LEST,MAY,BEFORE,PARTY,HENCE}
,  keepspaces=true
}


\setlength{\intextsep}{5pt}
\setlength{\textfloatsep}{5pt}

%-----------------------------
%%% Local Variables: 
%%% mode: latex
%%% TeX-master: "main"
%%% End: 


% Definition of colors
\definecolor{mauve}{rgb}{0.88, 0.69, 1.0}
\newcommand{\blue}[1]{{\color{blue}#1}}
\newcommand{\green}[1]{{\color{green}#1}}
\newcommand{\red}[1]{{\color{red}#1}}
\newcommand{\mauve}[1]{{\color{mauve}#1}}

% Remark macros for the authors
\newcommand{\remam}[2][]{\todo[color=blue!40,#1]{AM: #2}}
\newcommand{\remms}[2][]{\todo[color=green!40,#1]{MS: #2}}
\newcommand{\remmww}[2][]{\todo[color=blue!20,#1]{MWW: #2}}

% Common abbreviations
\newcommand{\eg}{\textit{e.g.\ }}
\newcommand{\etal}{\textit{et al.\ }}
\newcommand{\etc}{\textit{etc}}
\newcommand{\ie}{\textit{i.e.\ }}
\newcommand{\viz}{\textit{viz.\ }}
\newcommand{\wrt}{\textit{w.r.t.\ }}


%%% Local Variables: 
%%% mode: latex
%%% TeX-master: "main"
%%% End: 


\title{Overview of the CCLAW L4 project}

\author{Avishkar Mahajan, Martin Strecker, Meng Weng Wong}
\date{2022-01-16}


%======================================================================

\begin{document}


%======================================================================

\begin{frame}
  \titlepage
\end{frame}



%======================================================================
\section{Overview}


%-------------------------------------------------------------
\begin{frame}[fragile]\frametitle{Internal slide}

  \red{For internal discussion}
  
  \blue{Duration:}
  \begin{itemize}
  \item Time slot: 20 min, 
  \item which means: max. 16 min talk, 
  \item thus maybe 8-10 slides
  \end{itemize}

  \blue{Time and Date:}
  \begin{itemize}
  \item Sun 16 Jan 5pm local time (US Eastern time)
  \item which is: Sun 16 Jan 11pm in Europe
  \item which is: Mon 17 Jan 6am in Singapore
  \end{itemize}

\end{frame}


%-------------------------------------------------------------
\begin{frame}[fragile]\frametitle{Internal slide}

  \red{For internal discussion}


  \blue{Global positioning / central message}

  \begin{itemize}
  \item \red{We cannot be exhaustive and have to focus on a few central aspects}
  \item Since ProLaLa is mostly programming-language centric:
    \begin{itemize}
    \item L4 as a DSL
    \item High-level (Natural L4) and low-level (Core L4)
    \item Transpilation
    \item Core L4 amenable to ``direct'' definition of semantics (by translation to logic)
    \item Semantics of Natural L4 defined by translation to Core L4
    \end{itemize}
  \end{itemize}

\end{frame}


%-------------------------------------------------------------
\begin{frame}[fragile]\frametitle{Administrative overview of the project}

  \blue{Administrative overview:} \red{Do we want that?}
  \begin{itemize}
  \item Goals and structure of CCLAW
  \item Pragmatic orientation
  \end{itemize}

\end{frame}


%-------------------------------------------------------------
\begin{frame}[fragile]\frametitle{Essential tenets}

  \blue{Fundamental notions:}
  \begin{itemize}
  \item The law / a contract defines
    \begin{itemize}
    \item a \emph{set} of admissible situations (static)
    \item a \emph{set} of admissible behaviours (dynamic)
    \end{itemize}
  \item L4 is therefore primarily a \emph{declarative} and not an
    \emph{executable} language
  \end{itemize}
  
  \blue{Fundamental questions} we try to answer:
  \begin{itemize}
  \item is a law free of contradictions? (behaviour admissible \emph{and} inadmissible))
  \item is a law complete? (some behaviours not qualified as (in)admissible)
  \item is an individual behaviour admissible wrt. a law?
  \item how can laws / contracts be composed?
  \end{itemize}


\end{frame}


%-------------------------------------------------------------
\begin{frame}[fragile]\frametitle{Overall approach}

  \blue{Hierarchy of languages}, here:
  \begin{itemize}
  \item user-friendly \emph{Natural L4}, formally reducible to Core L4
  \item compact \emph{Core L4}, amenable to precise model-theoretic semantics
  \end{itemize}

  \blue{Principle:} Minimize complexity of the core language.
  \begin{itemize}
  \item Avoid debatable logics\\
    \emph{e.g.:} the ``right'' default logic\\
    \emph{instead:} compile away defeasibility, statically
  \item Avoid debatable combinations of logics\\
    \emph{e.g.:} combination of temporal and deontic logics\\
    \emph{instead:} combined treatment as attainable states in a Kripke structure
  \end{itemize}
  \red{Illustrated in the following}
  
\end{frame}


%-------------------------------------------------------------
\begin{frame}[fragile]\frametitle{Technical overview of the project}


  \blue{High-level technical overview} (see GoogleDoc diagram p. 2)

\end{frame}



%======================================================================
\section{Natural L4}


%-------------------------------------------------------------
\begin{frame}[fragile]\frametitle{PDPA Example Rule}



\end{frame}


%======================================================================
\section{Core L4 - Static}


%-------------------------------------------------------------
\begin{frame}[fragile]\frametitle{Defeasibility}

  (Martin's part of CLAR paper)

\end{frame}

%-------------------------------------------------------------
\begin{frame}[fragile]\frametitle{Defeasible Reasoning via Answer Set Programming} 

 - Various defeasible logics have been proposed to automate legal reasoning, particularly in argumentation theory\\
 
 - Our aim is to be able to simulate various non-monotonic semantics in ASP, which uses just \textit{negation-as-failure} interpreted in terms of the stable model semantics as its single core non-monotonic operation\\
 
 
 - We achieve this by separating the raw-rules from a set of defeasibility 'meta-rules', all written in ASP.\\
 
 - Changing the defeasibility meta-rules changes the semantics given to the rules

\end{frame}

%-------------------------------------------------------------
\begin{frame}[fragile]\frametitle{Answer Set Programming based expert system}

- Overarching aim : Combine forward chaining and backward chaining to allow user to interact closely with the rule set

For example:
-Infer facts to support given conclusions (abductive reasoning), use this as a form of question generation

-Explore multiple derivations/proofs of conclusions based on. which predicates are subject to CWA and which are not

-Justify conclusions derived by forward reasoning mechanism


\end{frame}




%======================================================================
\section{Core L4 - Dynamic}


%-------------------------------------------------------------
\begin{frame}[fragile]\frametitle{Translation of dynamic rules to Timed Automata}



\end{frame}

%-------------------------------------------------------------
\begin{frame}[fragile]\frametitle{Conclusions and outlook}

  \blue{What we have achieved:}

  \blue{Open challenges:}


\end{frame}



%-------------------------------------------------------------


\end{document}


%%% Local Variables: 
%%% mode: latex
%%% TeX-master: t
%%% coding: utf-8
%%% End: 
