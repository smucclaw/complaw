\documentclass[runningheads]{llncs}

% Packages from LIPICS

\usepackage{fancyhdr}

\usepackage{amsmath}
\usepackage{amssymb}
\usepackage[T1]{fontenc}
\usepackage[utf8]{inputenc}

\usepackage{alltt}
\usepackage{makeidx}  % allows for indexgeneration
\usepackage{listings}
\usepackage{caption}
%\usepackage{subcaption}  % does not seem to work with llncs
\usepackage{subfig}  % does not seem to work with llncs
%\usepackage[pdftex]{graphicx}
%\usepackage{enumerate}
\usepackage{mathpartir}
%\usepackage{isabelle,isabellesym}
%\isabellestyleit
\usepackage{todonotes}
\usepackage{wrapfig}
\usepackage{framed}
\usepackage{mathtools}
\usepackage{stmaryrd}
\usepackage{soul}

\usepackage{rail}    % For railroad syntax diagrams


\setcounter{secnumdepth}{3}
    
\usepackage{tikz}
\usetikzlibrary{arrows.meta}
\usetikzlibrary{trees}
\usetikzlibrary{graphs}
\usetikzlibrary{arrows}
\usetikzlibrary{decorations.pathmorphing}
\usetikzlibrary{shapes.multipart}
\usetikzlibrary{shapes.geometric}
\usetikzlibrary{shapes.misc}
\usetikzlibrary{calc}
\usetikzlibrary{positioning} 
\usetikzlibrary{fit}
\usetikzlibrary{backgrounds}


\usepackage[colorlinks,hyperindex,bookmarks,linkcolor=blue,citecolor=blue,urlcolor=blue]{hyperref}


%4 mars 2016 setting for Java code template
\renewcommand{\lstlistingname}{Code}


\definecolor{dkgreen}{rgb}{0,0.6,0}
\definecolor{gray}{rgb}{0.5,0.5,0.5}
\definecolor{mauve}{rgb}{0.58,0,0.82}

% \lstset{frame=tb,
%   language=Java,
%   aboveskip=3mm,
%   belowskip=3mm,
%   showstringspaces=false,
%   columns=flexible,
%   basicstyle={\footnotesize\ttfamily},
%   numbers=none,
%   numberstyle=\tiny\color{gray},
%   keywordstyle=\color{blue},
%   commentstyle=\color{dkgreen},
%   stringstyle=\color{mauve},
%   breaklines=true,
%   breakatwhitespace=true,
%   tabsize=2
% }



% Configuration of lstlisting for L4
\lstdefinelanguage{L4}
{morekeywords={
    assert
    , assign
    , chan
    , class
    , clock
    , decl   
    , defn   
    , derivable
    , else 
    , exists
    , extends
    , fact   
    , false
    , for  
    , forall
    , guard
    , if   
    , in
    , init
    , let   
    , lexicon
    , not   
    , process
    , rule
    , state
    , system
    , then
    , trans
    , true 
},    
sensitive=false,
morecomment=[l]{\#},
morestring=[b]",
}


\lstset{frame=tb,
  language=L4,
  aboveskip=3mm,
  belowskip=3mm,
  showstringspaces=false,
  columns=flexible,
  basicstyle={\footnotesize\ttfamily},
  numbers=none,
  numberstyle=\tiny\color{gray},
  keywordstyle=\color{blue},
  commentstyle=\color{dkgreen},
  stringstyle=\color{mauve},
  breaklines=true,
  breakatwhitespace=true,
  tabsize=2,
  keepspaces=true
}

\lstdefinelanguage{L4Sugary}
{  morekeywords={EVERY,UNLESS,UPON,WITHIN,MUST,LEST,MAY,BEFORE,PARTY,HENCE}
,  keepspaces=true
}


\setlength{\intextsep}{5pt}
\setlength{\textfloatsep}{5pt}

%-----------------------------
%%% Local Variables: 
%%% mode: latex
%%% TeX-master: "main"
%%% End: 


% Definition of colors
\definecolor{mauve}{rgb}{0.88, 0.69, 1.0}
\newcommand{\blue}[1]{{\color{blue}#1}}
\newcommand{\green}[1]{{\color{green}#1}}
\newcommand{\red}[1]{{\color{red}#1}}
\newcommand{\mauve}[1]{{\color{mauve}#1}}

% Remark macros for the authors
\newcommand{\remam}[2][]{\todo[color=blue!40,#1]{AM: #2}}
\newcommand{\remms}[2][]{\todo[color=green!40,#1]{MS: #2}}
\newcommand{\remmww}[2][]{\todo[color=blue!20,#1]{MWW: #2}}

% Common abbreviations
\newcommand{\eg}{\textit{e.g.\ }}
\newcommand{\etal}{\textit{et al.\ }}
\newcommand{\etc}{\textit{etc}}
\newcommand{\ie}{\textit{i.e.\ }}
\newcommand{\viz}{\textit{viz.\ }}
\newcommand{\wrt}{\textit{w.r.t.\ }}


%%% Local Variables: 
%%% mode: latex
%%% TeX-master: "main"
%%% End: 


% Possilby remove for final version
%\pagestyle{plain}

\begin{document}
%\title{Computer-Supported Defeasible Reasoning in Law}
%\title{Computer-Supported Defeasible Reasoning in Legal Texts}
\title{Overview of the CCLAW L4 project}

\author{Martin Strecker\orcidID{0000-0001-9953-9871}}
\institute{Singapore Management University}
\maketitle

\begin{abstract}
L4 is a domain-specific specification language that facilitates semantically rigorous formalization of legal expressions found in legislation and contracts. In our talk we will demo the working pieces (as of late 2021 / early 2022): a low-level core and a high-level language, a transpiler to reasoning back-ends including static analysis / formal verification and an expert system web app, and a real-world case study.
\end{abstract}

\keywords{
  Knowledge representation and reasoning,
  Argumentation and law,
  Computational Law,
  Defeasible reasoning,
  Programming language theory,
  Specification languages
}

%----------------------------------------------------------------------

\section{Context}\label{sec:context}

Computerized support for legal reasoning has a long history going back to the
1970s; in the decades since, a wide variety of approaches have appeared at venues such as ICAIL and JURIX.
This paper shares the main tenets of our approach at the Centre
for Computational Law\footnote{\url{https://cclaw.smu.edu.sg/}} at Singapore
Management University. We describe some principles and requirements in
this section, and then give an overview of current work
(\secref{sec:current_work}).

The first principle focuses our work on rule-based reasoning in law (\emph{vs} case-based reasoning). Our approach lies in the formalist tradition, closer to pure civil law than pure common law. We explore the deductive and abductive modes of reasoning, leaving inductive reasoning (supported by similarity metrics) to other work in the very active domain of machine learning.
The second principle is that law is meant to
describe a set of admissible behaviours, and not a single behaviour, so our
framework is declarative, in contrast to
\emph{smart contract} languages which are in fact programming
languages. We seek to combine formal rigour with
user-friendliness and accessibility to lay legal workers. This has led us to conceive our own domain-specific language
(DSL) called L4, instead of adopting an existing general-purpose
language.
Finally, the language aims at supporting a variety of tasks relevant
in a law context, ranging from legal drafting and reasoning \emph{about} rules
(such as consistency checking of a rule set) through compliance checking of
business processes to reasoning \emph{with} rules in order to derive
consequences of an individual usage scenario.

L4 is inspired by prior research at Gothenburg, at Copenhagen, at Data61, at INRIA, by the
LegalRuleML working group, and by many others.

\section{Current Work}\label{sec:current_work}

\subsection{Language}\label{sec:language}

We will now highlight some aspects and usage patterns of the L4
platform. We borrow from RuleML the notions of \emph{constitutive} rules which provide ontology, and \emph{prescriptive} rules which regulate behaviour by actors. We first describe static prescriptive rules and their possible uses, then dynamic prescriptive rules for time-dependent, multi-agent processes.

In their simplest form, in the core language, \textbf{rules} are \emph{if-then} statements, where both the
antecedent and consequent are expressions. We encode an
excerpt of the Professional Conduct Rules from Singapore's Legal Profession
Act:\footnote{\url{https://sso.agc.gov.sg/SL/LPA1966-S706-2015}}

\begin{lstlisting}[language=L4]
rule <r1a>
for lpr: LegalPractitioner, app: Appointment
if (exists bsn : Business. AssociatedWithAppB app bsn && IncompatibleDignity bsn)
then MustNotAcceptApp lpr app
\end{lstlisting}

\noindent The core \textbf{expression} language here is a simply-typed lambda calculus. As seen in
the example, there is no \emph{a priori} restriction to using complex
(quantified, higher-order, \dots) expressions in rules, as long as they are
well typed, but further processing steps may be restricted to fragments of the
rule language or require previous (automated) transformations. With this, the
above style of rule can be used without further ado in a Prolog-like fashion, to represent constitutive rules and simple prescriptive rules based on them.

In legal formalization, questions of deontics and of defeasible reasoning are
inevitable. L4 supports neither of them in the core language, \ie{} in the logic, but
provides mechanisms for expressing them by other means. In dynamic, state-based system
specifications, individual states can be flagged as breach states,
allowing for a fine-grained reasoning about traces that avoid, reach or repair
these breaches.

Neither is there native support for \emph{default reasoning}. As an extra-logical feature, a rule can be marked with
modifiers like \emph{subject to} or \emph{despite}, which prioritizes rules and
provides a \emph{defeasible} reasoning pattern that, in absence
of knowledge to the contrary, a given rule is applicable. In a similar spirit,
negation by failure can be simulated by rule inversion. It is worth noting
that these mechanisms do not rely on an extension of the logic, but on
compilation of rules.

The core language is augmented by a higher-level language, whose syntax is inspired by SQL, which transpiles to the core and to other target languages. The following fragment, based on Singapore's Personal Data Protection Act\footnote{\url{https://sso.agc.gov.sg/Act/PDPA2012?ProvIds=P1VIA-##pr26C-}} and related Regulations\footnote{\url{https://sso.agc.gov.sg/SL/PDPA2012-S64-2021?DocDate=20210930&WholeDoc=1}} expresses a prescriptive rule -- actually three sub-rules, connected by the \texttt{HENCE} and \texttt{LEST} keywords:
\begin{lstlisting}[language=L4Sugary]
 EVERY  Organisation  ("You")
UNLESS  the Organisation is a Public Agency
  UPON  becoming aware a data breach may have occurred
WITHIN  30 days
  MUST  assess
        if   it is a Notifiable Data Breach
  LEST      PARTY  Personal Data Protection Commission  ("PDPC")
           WITHIN  30 days
              MAY  demand
                   an explanation   for your inaction
            HENCE  You MUST  respond
                             to   PDPC
                     BEFORE  28 days
\end{lstlisting}

\subsection{Back-End Services}\label{sec:services}

The L4 transpiler integrates the developer-facing front-end with a variety of back-ends, including:
\begin{itemize}
\item SAT/SMT solvers and related tools such as Uppaal\footnote{\url{https://uppaal.org/}}, for model checking, static analysis, symbolic execution, and formal verification;
\item Answer Set Programming and (Constraint) Logic Programming, to help perform abductive reasoning to solve planning problems arising in legal contexts;
\item Expert systems, to assist with rule optimization and operationalization in an end-user UI (typically a web app or chatbot providing decision support);
\item Natural language generation, for validation round-tripping from natural language text to natural language text, and to improve the UI generally; and
\item Visualization, to aid comprehension of rule sets at a variety of levels of detail, benefitting both end-users and drafters.
\end{itemize}

We provide \textbf{reasoning services} by translating rules to other languages or
solvers. According to the principle of usability for the target audience, we
aim at a push-button technology, which precludes the use of (say) interactive
proof assistants. We provide the following, mostly complementary, services:

\begin{itemize}
\item \textbf{SMT solvers:} Their purpose is to permit reasoning about a rule
  set and thus to explore conditions of consistency, completeness or other
  user-defined criteria. In the above example of professional conduct rules,
  one might be interested in knowing whether
  \texttt{MustNotAcceptApp} and \texttt{MayAcceptApp} are complete and
  mutually exclusive. It has to be noted that the law text explicitly defines
  both predicates, so one cannot simply be assumed to be the negation of the
  other. For verification, users can write down a proof obligation (a formula). The L4
  tool set translates this formula and the rules (eventually transformed, as
  mentioned above) to expressions in the the SMTLIB format, sends them to a
  SMT solver and retrieves the result.  Other consistency checks (such as a search for
  overlapping rules) could be done automatically, but has not yet been
  implemented. SMT solvers are in principle also capable of reasoning about
  specific scenarios, but the following methods are more appropriate.
\item \textbf{Answer Set Programming:}
\item \textbf{``Expert system'':} By this, we understand a rule based
  production system where rules are applied to data stored in a fact base to
  derive new data, and this until a fixed point is reached. We currently
  experiment with the systems Drools\footnote{\url{https://www.drools.org}},
  Clara\footnote{\url{http://www.clara-rules.org/}} and
  O'Doyle\footnote{\url{https://github.com/oakes/odoyle-rules}}. As compared
  to primarily logical emthods, the strength
  of these engines lies in their capability to process larger quantities of
  data.
\end{itemize}
Our current work concentrates on providing readable justifications for traces
of rule applications or models produced by solvers.


\begin{itemize}
\item Natural language processing
\item Visualization
\end{itemize}

\subsection{Tool support}\label{sec:tool_support}

The reader should be aware that most of what has been described is still work
in progress. Work on the above is ongoing, with an open-source development in
Haskell available on CCLAW's Github\footnote{\url{https://github.com/smucclaw}}, with an integration into
Visual Studio Code that features syntax highlighting, on-the fly type checking
and code completion. An alternative IDE is being prototyped as a convention for
drafting rules inside a spreadsheet environment such as Excel or Google Sheets.

\paragraph{Acknowledgements.}
This research is supported by the National Research Foundation (NRF),
Singapore, under its Industry Alignment Fund –- Pre-Positioning Programme, as
the Research Programme in Computational Law. Any opinions, findings and
conclusions or recommendations expressed in this material are those of the
authors and do not reflect the views of National Research Foundation,
Singapore.

%----------------------------------------------------------------------
% \bibliographystyle{splncs04}
% \bibliography{main}
%% \bibliographystyle{abbrv}


%----------------------------------------------------------------------

\end{document}

%%% Local Variables: 
%%% mode: latex
%%% TeX-master: t
%%% coding: utf-8
%%% End: 