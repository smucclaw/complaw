\section{Completeness for restricted class of problems}\label{sec:completeness}
We will now prove the following theorem:
\begin{theorem}[Completeness]\label{thm:completeness}
  If there exists a general solution $S$ to $\langle R,q,U,\emptyset,N\rangle$ then there
  exists a ASP solution $S_{ASP}$ to $\langle R,q,U,\emptyset,N\rangle$.
\end{theorem}
Here by a general solution $S$, we mean any set of facts $S$ satisfying the following conditions that we also mentioned in Section \ref{formalsetup}. Namely we have that: \begin{enumerate}
    \item $q$ is in some answer set $\mathcal{A}$ of $F\cup R$.
    \item $F$ contains all the positive facts from $U$ and $F$ does not violate any of the integrity constraints in $U$ Ie. $F$ does not contain abducibles that are specifically disallowed by the constraints on abducibles in $U$.
    \item $F$ does not have any atoms whose depth level is greater than $N$.
\end{enumerate}

For the completeness proof we make the following assumptions leaving an
investigation of the general case for future work. Given $\langle
R,q,U,C,N\rangle$ we make the following assumptions. 

\begin{enumerate}
\item $R$ contains no negation as failure.
\item $C$ is empty.
\item The set of predicates which occur in user supplied facts in $U$ is
  disjoint from the set of predicates which are prevented from being used as
  abducibles
\item No post condition of any rule in $R$ contains repeated variables. For example the rule:\\
$p(X,X):-r(X,X,Y)$ is not allowed but the rule: $p(X,Y):-r(X,X,Y)$ is allowed.\remms{Linearity. The same holds for patterns of functional programs}
\end{enumerate}

Note that that these conditions mean that without loss of generality, for the sake of proving completeness, we may assume that the set of user supplied facts is empty. We will first need a preliminary lemma.

\begin{lemma}
Assume $\langle R,q,U,\emptyset,N\rangle$, $\langle R,q',U,\emptyset,N\rangle$ are two
problems satisfying the conditions above but $q'$ is obtained from $q$ by
possibly changing some or all of the arguments of $q$. Then $\langle
R,q,U,\emptyset,N\rangle$ has a solution $F$ derived from the ASP program $P_{\langle R,q,U,\emptyset,N\rangle}^{res}$ if and only if $\langle R,q',U,\emptyset,N\rangle$ has a solution $F'$ derived from the ASP program $P_{\langle R,q',U,\emptyset,N\rangle}^{res}$
does.
\end{lemma}

$Proof$ We prove the lemma by induction on $N$.
The case $N = 0$ is trivial. If there exists a solution for $\langle
R,q,U,\emptyset,0\rangle$ then any instance of the predicate in $q$ can be
abduced therefore $\langle R,q',U,\emptyset,0\rangle$ has a solution. Assume
the result for some $N = k$ such that $k>0$. A non trivial solution to
$\langle R,q,U,\emptyset,k+1\rangle$ implies the existence of a solution to
the following set of problems $\langle R,q_{pre_{i}},U,\emptyset,k\rangle$
where the set of $q_{pre_{i}}$ where the $q_{pre_{i}}$ are the set of
pre-conditions of a rule $r$ in $R$, under some substitution $\theta$ where
$\theta$ applied to the post-condition of $r$ gives $q$. But then since $r$
contains no repeated variables, if $q$ unifies with the post condition of $r$
under some substituion $\theta$, then there exists some other substitution
$\theta'$ for the variables in $r$ such that $\theta'$ applied to the
postcondition of $r$ gives $q'$. Therefore a solution to the set of problems
$\langle R,q'_{pre_{i}},U,\emptyset,k\rangle$ is also a solution to $\langle
R,q',U,\emptyset,k+1\rangle$ where each $q'_{pre_{i}}$ is obtained from
$q'_{pre_{i}}$ by possibly changing the arguments in the predicate. By the
induction hypothesis such a solution to the set of problems $\langle
R,q'_{pre_{i}},U,\emptyset,k\rangle$. Hence a solution to  $\langle
R,q,U,\emptyset,k+1\rangle$ exists. This proves the lemma.

\remms[inline]{Intuitively, the linearity requirement is not clear to me. Do you have a counterexample for the non-linear case?}
We now prove the main theorem by induction on $N$. 
\remms[inline]{It is not clear what a ``general'' and an ``ASP'' solution is. The structures $\langle R,q,U,\emptyset,N\rangle$ have always been defined to be ASP problems, see section 2.1}

$Proof$ The case $N=0$ is trivial. Assume
the result for some $N=k>0$. The existence of a non-trivial general solution
to $\langle R,q,U,\emptyset,k+1\rangle$ implies the existence of a solution to
the following set of problems, $\langle R,q_{pre_{i}},U,\emptyset,k\rangle$
where the set of $q_{pre_{i}}$ where the $q_{pre_{i}}$ are the set of
pre-conditions of a rule $r$ in $R$, under some substitution $\theta$ where
$\theta$ applied to the post-condition of $r$ gives $q$. But then by the
inductive hypothesis there is an ASP solution to the same set of problems. By
the lemma proved above there exists an ASP solution to following set of
problems $\langle R,q'_{pre_{i}},U,\emptyset,k\rangle$, where each
$q'_{pre_{i}}$ is obtained from the corresponding $q_{pre_{i}}$, by possibly
replacing some predicate arguments with the appropriate skolem terms for
predicate arguments that correspond to existential variables in the
preconditions of $r$. Hence there exists an ASP solution to
$\langle R,q,U,\emptyset,N\rangle$.

%%% Local Variables:
%%% mode: latex
%%% TeX-master: "main"
%%% End:
