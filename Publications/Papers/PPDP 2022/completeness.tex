\section{Completeness for simple  problems}\label{sec:completeness}
We will now prove the following theorem:
\begin{theorem}[Completeness]\label{thm:completeness}
  Given a simple a simple abductive proof generation task,if there exists a general solution $S$ to that task then there
  exists a ASP solution $S_{ASP}$ to that task corresponding to an answer set of the ASP program $P_{\langle R,q,U,\emptyset,N\rangle}^{res}$.
\end{theorem}
Note that without loss of generality, for the sake of proving completeness, we may assume that the set of user supplied facts is empty. We will first need a preliminary lemma.

\begin{lemma}
Assume $\langle R,q,U,\emptyset,N\rangle$, $\langle R,q',U,\emptyset,N\rangle$ are two
problems satisfying the conditions above but $q'$ is obtained from $q$ by
possibly changing some or all of the arguments of $q$. Then $\langle
R,q,U,\emptyset,N\rangle$ has a solution $S_{ASP}$ derived from the ASP program $P_{\langle R,q,U,\emptyset,N\rangle}^{res}$ if and only if $\langle R,q',U,\emptyset,N\rangle$ has a solution $S'_{ASP}$ derived from the ASP program $P_{\langle R,q',U,\emptyset,N\rangle}^{res}$ We call such solutions ASP solutions for short.
\end{lemma}

\begin{proof}
We prove the lemma by induction on $N$.
The case $N = 0$ is trivial. If there exists a ASP solution for $\langle
R,q,U,\emptyset,0\rangle$ then any instance of the predicate in $q$ can be
abduced therefore $\langle R,q',U,\emptyset,0\rangle$ has a ASP solution. Assume
the result for all $N<k$ such that $k>0$. A non trivial ASP solution to
$\langle R,q,U,\emptyset,k+1\rangle$ (ie a solution in which no instance of the predicate in $q$ can be abduced) implies the existence of some ASP solution to
each of the following set of problems $\langle R,q_{pre_{i}},U,\emptyset,k\rangle$
where the $q_{pre_{i}}$ are a set of
pre-conditions of a rule $r$ in $R$, under some substitution $\theta$ where
$\theta$ applied to the post-condition of $r$ gives $q$. But then since the post- condition of $r$
contains no repeated variables, if $q$ unifies with the post condition of $r$
under some substituion $\theta$, then there exists some other substitution
$\theta'$ for the variables in $r$ such that $\theta'$ applied to the
postcondition of $r$ gives $q'$. Therefore the union of ASP solutions to the set of problems
$\langle R,q'_{pre_{i}},U,\emptyset,k\rangle$ is also a ASP solution to $\langle
R,q',U,\emptyset,k+1\rangle$ where each $q'_{pre_{i}}$ is obtained from
$q_{pre_{i}}$ by possibly changing the arguments in the predicate. By the
induction hypothesis an ASP solution to each of the problems $\langle
R,q'_{pre_{i}},U,\emptyset,k\rangle$ exists. Hence by taking the union of the solutions we get an ASP solution to  $\langle
R,q',U,\emptyset,k+1\rangle$. This proves the lemma.
\end{proof}

We now prove the main theorem by induction on $N$. 

\begin{proof}
The case $N=0$ is trivial. Assume
the result for all $N<k$, where $k>0$. The existence of a non-trivial general solution
to $\langle R,q,U,\emptyset,k+1\rangle$ implies the existence of a solution to each of the following set of problems, $\langle R,q_{pre_{i}},U,\emptyset,k\rangle$ where the $q_{pre_{i}}$ are the set of
pre-conditions of a rule $r$ in $R$, under some substitution $\theta$ where
$\theta$ applied to the post-condition of $r$ gives $q$. But then by the
inductive hypothesis there is an ASP each of the same set of problems. By
the lemma proved above there exists an ASP solution to each of the following set of
problems $\langle R,q'_{pre_{i}},U,\emptyset,k\rangle$, where each
$q'_{pre_{i}}$ is obtained from the corresponding $q_{pre_{i}}$, by possibly
replacing some predicate arguments with the appropriate skolem terms for
predicate arguments that correspond to existential variables in the
preconditions of $r$. Hence taking a union of these ASP solutions we get an ASP solution to
$\langle R,q,U,\emptyset,N\rangle$.
\end{proof}


\section{Summary of results - I}
Let us just comment on the results above in a slightly broader context . Firstly it is not difficult to see that the above results for finiteness and completeness hold for a slightly larger class of abductive proof generation tasks than the class of simple tasks. Namely we can in fact relax condition codition 7 in the definition of simple tasks, to allow $q$ to be un-ground or only partially ground. Call this class of abduction tasks $semi$-$simple$.  Also, the completeness result in fact holds if $P_{R,q,U,C,N}^{res}$ is replaced with $P_{R,q,U,C,N}^{semi-res}$. In summary what we have then is the following\\
Given a $\textit{semi-simple}$ task $\langle R,q,U,\emptyset,N\rangle$. $P_{R,q,U,C,N}^{res}$, $P_{R,q,U,C,N}^{semi-res}$, both enjoy the completeness and finiteness properties. However only $P_{R,q,U,C,N}^{semi-res}$ supports full implicit term substitution whereas $P_{R,q,U,C,N}^{res}$ only supports partial term substitution. We shall formulate and prove a formal result regarding term substitution for $P_{R,q,U,C,N}^{res}$ next.    

%%% Local Variables:
%%% mode: latex
%%% TeX-master: "main"
%%% End:
