\section{Extra Material}

\subsection{Proof of Finiteness}\label{sec:proof_finiteness}

We will prove the finiteness result of Theorem~\ref{thm:finiteness} via a series of lemmas. Also we will refer to specific lines of the encoding given below but the arguments in the proof are fully general and can be easily extended to the encoding of any abductive proof generation problem

\begin{lstlisting}[numbers=left]
max_ab_lvl(5).
query(relA(bob),0).
goal:-holds(relA(bob)).
:-not goal.


holds(relA(P)) :- holds(relB(P, R)),holds(relD(R)).
holds(relB(P, R)) :- holds(relA(R)), holds(relC(P)).

explains(relB(P, R), relA(P) ,N) :- createSub(subInst_r1(P,R),N).
explains(relD(R), relA(P) ,N) :- createSub(subInst_r1(P,R),N).


createSub(subInst_r1(P,skolemFn_r1_R(P)),N+1) :- query(relA(P) ,N),max_ab_lvl(M),N<M-1.
createSub(subInst_r2(P,Q),N+1) :- query(relB(P, Q) ,N),max_ab_lvl(M),N<M-1.


explains(relA(R), relB(P,R) ,N) :- createSub(subInst_r2(P,R),N).
explains(relC(P), relB(P,R) ,N) :- createSub(subInst_r2(P,R),N).



createSub(subInst_r1(P,R),M-1) :- createSub(subInst_r1(V_P,V_R),N), holds(relB(P, R)),max_ab_lvl(M).
createSub(subInst_r1(V_P,R),M-1) :- createSub(subInst_r1(V_P,V_R),N), holds(relD(R)),max_ab_lvl(M).

createSub(subInst_r2(V_P,R),M-1) :- createSub(subInst_r2(V_P,V_R),N), holds(relA(R)),max_ab_lvl(M).

createSub(subInst_r2(P,V_R),M-1) :- createSub(subInst_r2(V_P,V_R),N), holds(relC(P)),max_ab_lvl(M).


query(X,N):-explains(X,Y,N),max_ab_lvl(M),N<M.
{abducedFact(X)}:-query(X,N).
holds(X):-abducedFact(X).
holds(X):-user_input(pos,X).
\end{lstlisting}


\begin{lemma}
Let $S$ be the set of uninterpreted skolem functions appearing in the abducible
generation encoding. Let $C$ be the set of constants occurring in either a user
inputed fact in $U$ or the query $q$. Let $T_{k}$ denote the set of unique
terms that can be constructed from $S$ and $C$ with skolem depth at most
$k$. Let $T$  be the possibly infinite set consisting of terms of unbounded
depth. Then for any positive integer $k$, $T_{k}$ is finite.
\end{lemma}
% \remms{Better write: $T$ is the union of the $T_k$. However, the lemma does not say anything about $T$} 

\begin{proof}
This is a standard result which follows easily by induction. Clearly $|T_{0}|$ = $|C|$, which is finite. Let $|T_{k}|$ = $B$ be finite. Let $S$ be a set of $l$ functions having arity at most $d$. Then $|T_{k+1}|\leq B + lB^{d}$. Hence $T_{k+1}$ is finite.  
\end{proof}

\begin{lemma}
 For any predicate $p$ inside an $abducedFact$ atom the arguments of $p$ are elements of the set $T_{N}$. Similarly for any predicate $p$ inside an $holds$ atom the arguments of $p$ are elements of the set $T_{N}$. Similarly for the other atoms in any answer set of $P_{\langle R,q,U,C,N \rangle}^{res}$. Terms that correspond to arguments of predicates occuring in the input rules are always elements of $T_{N}$.
\end{lemma}

\begin{proof}\textit{(sketch)}. Firstly note that for any predicate $p$ inside a
$abducedFact$ or $holds$ atom, where $p$ comes from the input rule set, the
arguments of $p$ belong to the (possibly infinite) set $T$. We can see this by observing that
since no rule in $R$ contains a function symbol, the encoding of the rules
themselves such as in line 7,8 of the ASP program above, cannot introduce new
terms in the arguments of predicates inside $holds$ atoms. Similarly lines
like 23, 24 etc. also cannot introduce new terms inside $holds$, $create\_sub$
atoms in the final answer set, which means also that no new terms appear as
predicate arguments inside $abducedFact$ atoms.

Only lines like 14,15 are able to create fresh terms that become arguments of
various atoms in the answer set. But then it follows that any $holds$ atom,
$create\_sub$, $abducedFact$, $query$, $explains$ atoms can only have input
rule predicate arguments from the set $T$.

Next we show that the skolem depth of these terms is at most $N$. Any fresh
term $t$ from $T$ appearing in an atom in an answer set of the ASP program,
such that $t$ does not appear in any user provided fact or in the original
query $q$, must have been constructed from rules like in lines 14,15. But
these encode a 'static' space of abducibles, which unlike lines 23, 24 is
independent of the forward reasoning component or any user provided additional
facts. This is because any $query$ atom that gives an instantiation of the
right hand side of rules like the one in line 14, must have integer argument
less than $M-1=N$, (recall that we have $M=N+1$), whereas rules like the one
in line 24, gives us $query$ atoms with integer argument $M-1$. It is not hard
to see that because of this fresh terms created via ASP rules like the one on
line 14 must have skolem depth at most $N$. Therefore all terms inside atoms
of the answer-set that correspond to input rule predicate arguments must
belong to the set $T_{N}$.
\end{proof}

\begin{lemma}
Since the set $T_{N}$ is finite it follows that any answer set of $P_{\langle
  R,q,U,C,N \rangle}^{res}$ is finite.
\end{lemma}

\begin{proof}
Since $T_{N}$ is finite and the integer argument of any $create\_sub$, $explains$, $query$ 
atom is bounded by $N$, any atom in the final answer set has only finitely
many instantiations. This proves finiteness of the resulting answer set.
\end{proof}

\subsection{Proof of Completeness}\label{sec:proof_completenes}


Note that without loss of generality, for the sake of proving completeness, we may assume that the set of user supplied facts is empty. We will first need a preliminary lemma.

\begin{lemma}
Assume $\langle R,q,U,\emptyset,N\rangle$, $\langle R,q',U,\emptyset,N\rangle$ are two
problems satisfying the conditions above but $q'$ is obtained from $q$ by
possibly changing some or all of the arguments of $q$. Then $\langle
R,q,U,\emptyset,N\rangle$ has a solution $S_{ASP}$ derived from the ASP program $P_{\langle R,q,U,\emptyset,N\rangle}^{res}$ if and only if $\langle R,q',U,\emptyset,N\rangle$ has a solution $S'_{ASP}$ derived from the ASP program $P_{\langle R,q',U,\emptyset,N\rangle}^{res}$ We call such solutions ASP solutions for short.
\end{lemma}

\begin{proof}
We prove the lemma by induction on $N$.
The case $N = 0$ is trivial. If there exists a ASP solution for $\langle
R,q,U,\emptyset,0\rangle$ then any instance of the predicate in $q$ can be
abduced therefore $\langle R,q',U,\emptyset,0\rangle$ has a ASP solution. Assume
the result for all $N<k$ such that $k>0$. A non trivial ASP solution to
$\langle R,q,U,\emptyset,k+1\rangle$ (ie a solution in which no instance of the predicate in $q$ can be abduced) implies the existence of some ASP solution to
each of the following set of problems $\langle R,q_{pre_{i}},U,\emptyset,k\rangle$
where the $q_{pre_{i}}$ are a set of
pre-conditions of a rule $r$ in $R$, under some substitution $\theta$ where
$\theta$ applied to the post-condition of $r$ gives $q$. But then since the post- condition of $r$
contains no repeated variables, if $q$ unifies with the post condition of $r$
under some substituion $\theta$, then there exists some other substitution
$\theta'$ for the variables in $r$ such that $\theta'$ applied to the
postcondition of $r$ gives $q'$. Therefore the union of ASP solutions to the set of problems
$\langle R,q'_{pre_{i}},U,\emptyset,k\rangle$ is also a ASP solution to $\langle
R,q',U,\emptyset,k+1\rangle$ where each $q'_{pre_{i}}$ is obtained from
$q_{pre_{i}}$ by possibly changing the arguments in the predicate. By the
induction hypothesis an ASP solution to each of the problems $\langle
R,q'_{pre_{i}},U,\emptyset,k\rangle$ exists. Hence by taking the union of the solutions we get an ASP solution to  $\langle
R,q',U,\emptyset,k+1\rangle$. This proves the lemma.
\end{proof}

We now prove the main theorem by induction on $N$. 

\begin{proof}
The case $N=0$ is trivial. Assume
the result for all $N<k$, where $k>0$. The existence of a non-trivial general solution
to $\langle R,q,U,\emptyset,k+1\rangle$ implies the existence of a solution to each of the following set of problems, $\langle R,q_{pre_{i}},U,\emptyset,k\rangle$ where the $q_{pre_{i}}$ are the set of
pre-conditions of a rule $r$ in $R$, under some substitution $\theta$ where
$\theta$ applied to the post-condition of $r$ gives $q$. But then by the
inductive hypothesis there is an ASP each of the same set of problems. By
the lemma proved above there exists an ASP solution to each of the following set of
problems $\langle R,q'_{pre_{i}},U,\emptyset,k\rangle$, where each
$q'_{pre_{i}}$ is obtained from the corresponding $q_{pre_{i}}$, by possibly
replacing some predicate arguments with the appropriate skolem terms for
predicate arguments that correspond to existential variables in the
preconditions of $r$. Hence taking a union of these ASP solutions we get an ASP solution to
$\langle R,q,U,\emptyset,N\rangle$.
\end{proof}




\section{Completeness for restricted class of problems}
For the completeness proof we make the following assumptions leaving an investigation of the general case for future work. Given $\langle R,q,U,C,N\rangle$ we make the following assumptions.

\begin{enumerate}
\item $R$ contains no negation as failure.
\item $C$ is empty.
\item The set of predicates which occur in user supplied facts in $U$ is
  disjoint from the set of predicates which are prevented from being used as
  abducibles
\item No post condition of any rule in $R$ contains repeated variables. For example the rule:
$p(X,X):-r(X,X,Y)$ is not allowed but the rule: $p(X,Y):-r(X,X,Y)$ is allowed.
\end{enumerate}

Note that that these conditions mean that without loss of generality, for the
sake of proving completeness, we may assume that the set of user supplied
facts is empty.

\subsection{Lemma}
Assume $\langle R,q,U,C,N\rangle$, $\langle R,q',U,C,N\rangle$ are two problems satisfying the conditions above but $q'$ is obtained from $q$ by possibly changing some or all of the arguments of $q$. Then $\langle R,q,U,C,N\rangle$ has a solution if and only if $\langle R,q',U,C,N\rangle$ does.

First we prove the lemma by induction on $N$.
The case $N = 0$ is trivial. If there exists a solution for $\langle R,q,U,\emptyset,0\rangle$ then any instance of the predicate in $q$ can be abduced therefore $\langle R,q',U,\emptyset,0\rangle$ has a solution. Assume the result for some $N = k$ such that $k>0$. A non trivial solution to $\langle R,q,U,\emptyset,k+1\rangle$ implies the existence of a solution to the following set of problems $\langle R,q_{pre_{i}},U,\emptyset,k\rangle$ where the set of $q_{pre_{i}}$ where the $q_{pre_{i}}$ are the set of pre-conditions of a rule $r$ in $R$, under some substitution $\theta$ where $\theta$ applied to the post-condition of $r$ gives $q$. But then since $r$ contains no repeated variables, if $q$ unifies with the post condition of $r$ under some substituion $\theta$, then there exists some other substitution $\theta'$ for the variables in $r$ such that $\theta'$ applied to the postcondition of $r$ gives $q'$. Therefore a solution to the set of problems $\langle R,q'_{pre_{i}},U,\emptyset,k\rangle$ is also a solution to $\langle R,q',U,\emptyset,k+1\rangle$ where each $q'_{pre_{i}}$ is obtained from $q'_{pre_{i}}$ by possibly changing the arguments in the predicate. By the induction hypothesis such a solution to the set of problems $\langle R,q'_{pre_{i}},U,\emptyset,k\rangle$. Hence a solution to  $\langle R,q,U,\emptyset,k+1\rangle$ exists. This proves the lemma.

\subsection{Proof of main theorem}
We now prove the main theorem by induction on $N$. That is we prove that if
there exists a general solution $S$ to $\langle R,q,U,\emptyset,N\rangle$ then there exists
a ASP solution $S_{ASP}$ to $\langle R,q,U,\emptyset,N\rangle$. We prove this by induction
on $N$. The case $N=0$ is trivial. Assume the result for some $N=k>0$. The
existence of a non-trivial general solution to $\langle R,q,U,\emptyset,k+1\rangle$ implies
the existence of a solution to the following set of problems,
$\langle R,q_{pre_{i}},U,\emptyset,k\rangle$ where the set of $q_{pre_{i}}$ where the
$q_{pre_{i}}$ are the set of pre-conditions of a rule $r$ in $R$, under some
substitution $\theta$ where $\theta$ applied to the post-condition of $r$
gives $q$. But then by the inductive hypothesis there is an ASP solution to
the same set of problems. By the lemma proved above there exists an ASP
solution to following set of problems $\langle R,q'_{pre_{i}},U,\emptyset,k\rangle$, where
each $q'_{pre_{i}}$ is obtained from the corresponding $q_{pre_{i}}$, by
possibly replacing some predicate arguments with the appropriate skolem terms
for predicate arguments that correspond to existential variables in the
preconditions of $r$. Hence there exists an ASP solution to
$\langle R,q,U,\emptyset,N\rangle$.

\section{Pf Idea}

$Proof:$ By induction on $N$. The case $N=0$ is trivial. The inductive case crucially uses property (2). $q'$ unifies with the post condition of a rule if and only if $q$ does.
The main theorem is now again proved using induction on $N$. The case $N=0$ is trivial and the inductive case uses the lemma above.

$Pf$ $idea$: General non-trivial solution at depth $N$ impliies general solution to conjunction of pre-conditions at depth $N-1$ with some substitution of existential variables.

By the inductive hypothesis, this implies ASP solution to conjunction of pre-conditions with that substitution at depth $N-1$.

By the lemma this implies ASP solution to conjunction of pre-conditiions where existential variables are replaced with skolem terms. 

This implies ASP solution to the original problem at depth $N$.


\section{Proof sketch of Completeness}

Assume $\langle R,q,U,C,N\rangle$ is as above but that $C$ is empty and the predicates ocurring in positive ground atoms in $U$ and integrity constraints in $U$ form disjoint sets. Furthermore for the completeness proof we make two further assumptions on $R$.

\begin{enumerate}
\item For any finite set of facts $F$, the set $(F,R)$ is never unsatisfiable.

\item For any rule $r$ in $R$, the postcondition of $r$ contains no repeated
  variables. For example the rule:
  
$p(X,X):-r(X,X,Y)$ is not allowed but the rule: $p(X,Y):-r(X,X,Y)$ is allowed.

\item None of the positive atoms user supplied facts in $U$ contain skolem terms

Then the proof search procedure is complete and furthermore we have proof simplification if the proof graph is filled from bottom up. Later we suggest a modification to the proof search encoding that allows 2) to be relaxed.
\end{enumerate}

Given a arbitrary solution $S$, to $\langle R,q,U,\emptyset,N\rangle$, let
$S$, $U_{f}$ be such that $S\cup U_{f}$ is subset minimal where $U_{f}$ is the
set of user supplied facts in $U$ we can obtain a solution $S'$ by constructed
from $S\cup U_{f}$, replacing all existential variables in $S$, $U_{f}$, by
the appropriate skolem terms. (This crucially makes use of property 2 above)
Then this solution $S'$ is a (possibly sub-optimal) solution to $\langle
R,q,U,\emptyset,N\rangle$ which lies in the solution space generated by the
ASP program.  (Note : if pf can't be made rigorous just restrict to rules
without NAF, which can be proved rigorously).

\subsection{Proof simplification}
Proof graph is simplified if filled from bottom up.

\section{Further novel functionalities}
Allows one to be more 'fine-grained' about what things are abducible and what are not eg:
\begin{verbatim}
:-abducedFact(p(john)).    
\end{verbatim}
Means that $p(john)$ is not abducible but any other instance of the predicate $p$ is.\\
\newline
The constraint:
\begin{verbatim}
:-abducedFact(p(X)),abducedFact(q(X,Y)).
\end{verbatim}
Means that both together cannot be abduced but either one on it's own may be. Can extend this sort of 'conditional abduction' further eg.
\begin{verbatim}
:- abducedFact(p(X)),holds(h(X,Y)).  
\end{verbatim}


let $abducedFact(f)$ be an element of some optimal ASP solution $S$ corresponding to an optimal answer set of the program $P_{<R,q,U,\emptyset,N>}^{Comp}$ such that $f$ is of the form $p_{i}(v_{1},v_{2},..v_{n})$. Suppose the atom $p_{i}(t_{1},t_{2},..t_{n})$ gets added to $U$, where each $t_{j} = v_{j}$ if $v_{j}$ was a non skolem term. Call this new set of user-inputs $U'$. Then the program $P_{<R,q,U',\emptyset,N>}^{Comp}$ has an answer set where the corresponding solution $S'$ is obtained from $S$ in the following way. Replace every occurrence of the term $v_{k}$ in the arguments of an abduced atom in $S$ with the term $t_{k}$, where $v_{k}$ is a skolem term. Then remove $p_{i}(t_{1},t_{2},..t_{n})$ itself from the resulting set.

To prove this we first need the notion of a $minimal$ $proof$ $graph$ corresponding to an optimal ASP solution $S$ of a given problem $\langle R,q,U,\emptyset,N\rangle$.

$\textit{Pf Idea}$ : Work with min proof graphs corresponding to min solns. Show by induction that filling any atom in the min pf graph at depth $k$ automatically causes any remaining abducibles to get filled. Start by proving $k=1$ case.

\section{Second pf attempt}
Suppose there exists an answer set $A$ of $P_{<R,q,U,\emptyset,N>}^{Comp}$ such that $query(p_{i}(v_{1},v_{2},..v_{n}),k)$, $query(p_{j}(w_{1},w_{2},..w_{m}),l)$ are in $A$ for some $0\leq k, l \leq N$. Then replacing the place-holder arguments in one induces an identical substitution of the same place-holder arguments in the other. Example:

If an answer set of the original problem has:
$query(p_{i}(a,var_{1},var_{2},c,var_{3}),k)$ and $query(p_{j}(h,var_{3},var_{1},o),l)$ then adding the fact $holds(p_{i}(a,e,y,c,k))$ means that there exists an answer set of the new problem such that $query(p_{j}(h,c,e,o),l)$ is in it. 

To prove this we will first need the following lemma.

\subsubsection{Lemma 1}
Given a fixed rule set $R$, integer $N$ and predicate $p_{i}$, consider $\langle R,q,U,\emptyset,N\rangle$, where $U$ is arbitrary and $q$ is some arbitrary instantiation of the predicate $p_{i}$. Then given any rule $r_{k}$ in $R$ and any answer-set $A$ of $P_{<R,q,U,\emptyset,N>}^{Comp}$, the set of integers $i$ such that $createSub(subInst\_r_{k}(...),i)$ appears in $A$ is exactly some fixed subset of the set $\{1,...,N\}$. Given an $r_{k}$ call this set the $\textit{occurence set}$ of $r_{k}$. Given a rule $r_{k}$ it's occurence set $O_{k}$ can be constructed inductively as follows:\\
$1 \in O_{k}$ if and only if the rule $r_{k}$ has the predicate $q$ as it's
post-condition. For any $i>1$ and any rule $r_{l}$ having post condition
predicate $p$, $i\in O_{l}$ if and only if there exists a rule $r_{f}$ such
that $r_{f}$ contains $i-1$ in it's occurence set and the predicate $p$ occurs
in the pre-conditions of $r_{f}$. Firstly note that condition (4) from the
previous section ensurest that any $query(p(...),g)$ gives rises to
$createSub(subInst\_r_{h}(...),g+1)$ where $g$ is less than $N-1$ and $p$ is
the predicate that is the post-condition of rule $r_{h}$. In particular it
does not matter what the arguments of $p$ in $query(p(...),g)$ actually
are. So this establishes that the occurence set of a rule $r_{f}$ does
actually contain all the integers that occur as a result of the inductive
construction. Next we explain why the occurence set cannot contain integers
other than those captured by the inductive definition above. Clearly proof
expansion rules of type E1, E2 cannot introduce new occurence levels. A-priori
the only concern maybe the rule E3,
\begin{verbatim}
query(Y,N-1):-explains(X,Y,N),0<N, N<M, max_ab_lvl(M)
    
\end{verbatim} 
which may introduce fresh $query$ atoms and thus add to the occurence levels of a rule. However any $explains$ atom comes from a rule of the form 
\begin{verbatim}
explains(relB(P, R), relA(P) ,_N) :- createSub(subInst_r1(P,R),_N).
\end{verbatim} 
so any for any fresh $query(Y,N-1)$ created due to E3, there was already another atom of the form $query(Y',N-1)$ present in the answer set where $Y$, $Y'$ are different instantiations of the same predicate. Therefore the rule E3 does not add to the occurence set of any rule in the original rule-set $R$. Therefore the occurence set is independent of $U$ and the arguments of $q$ in the original problem $\langle R,q,U,\emptyset,N\rangle$.

\subsubsection{Lemma 2}
Fix some term $T$ and consider ${<R,q,U,\emptyset,N>}$ where some predicate argument of $q$ is the term $T$. Then given some rule $r_{k}$ in $R$ if $j$ is in the occurence set of $r_{k}$ then in any answer set $A$
of $P_{<R,q,U,\emptyset,N>}^{Comp}$ there exists the atom $createSub(subInst\_r_{k}(T,T,...,T),j)$. 
\subsubsection{Proof of main theorem}
Given $query(p_{i}(a,var_{1},var_{2},c,var_{3}),k)$ and $query(p_{j}(h,var_{3},var_{1},o),l)$. The main theorem follows by induction on $k$, via the rule 
\begin{verbatim}
query(Y,N-1):-explains(X,Y,N),0<N, N<M, max_ab_lvl(M).   
\end{verbatim}
The case $k=0$ follows from Lemma 2 and re-write rules of the form 
\begin{verbatim}
createSub(subInst_r1(X1,Y2,Y3),N):-createSub(subInst_r1(Y1,Y2,Y3),N),
createSub(subInst_r1(X1,X2,X3),N).
createSub(subInst_r1(Y1,X2,Y3),N):-createSub(subInst_r1(Y1,Y2,Y3),N),
createSub(subInst_r1(X1,X2,X3),N).
createSub(subInst_r1(Y1,Y2,X3),N):-createSub(subInst_r1(Y1,Y2,Y3),N),
createSub(subInst_r1(X1,X2,X3),N).
\end{verbatim}

\section{Discusion of skolem term and place-holder term elimination}
An interesting avenue for further work is to identify a minimal set of proof-expansion rules necessary to achive a reasonable notion of skolem term/place-holder term elimination in minimal abductive solutions. For example what kind of term elimination properties are achieved when only proof expansion rules of the following kind are included:
\begin{verbatim}
createSub(subInst_r1(P,R),_N) :- createSub(subInst_r1(V_P,V_R),_N), holds(relB(P, R)).
createSub(subInst_r1(V_P,R),_N) :- createSub(subInst_r1(V_P,V_R),_N), holds(relD(R)).

createSub(subInst_r2(V_P,R),_N) :- createSub(subInst_r2(V_P,V_R),_N), holds(relA(R)).
createSub(subInst_r2(P,V_R),_N) :- createSub(subInst_r2(V_P,V_R),_N), holds(relC(P)).    
\end{verbatim}



We present a method for calculating possible proofs of a query with respect to a given Answer Set Programming (ASP) rule set using an abductive process where the space of abducibles is automatically constructed just from the input rules alone. Given a (possibly empty) set of user provided facts, our method is able to infer any additional facts that may be needed for the entailment of a query and then output the set of abducibles, all without the user needing to explicitly specify the space of all abducibles. We also present a method to generate a set of directed edges corresponding to the the justification graph for the query. Furthermore, through different forms of term substitution, our method is able to take user provided facts into account and adapt the proofs accordingly. Past work in implementing abductive reasoning systems has been primarily based on top-down execution methods with either Prolog or more recently, goal-directed ASP as the underlying engine. However goal directed methods may result in solvers that are not truly declarative. Much less work has been done on realizing abduction in a bottom up solver like Clingo. The work in this paper describes novel ASP programs which can be run directly in Clingo to yield the abductive solutions and directed edge sets without needing to modify the underlying solving engine. 

We present a method for calculating possible proofs of a query with respect to
a given Answer Set Programming (ASP) rule set using an abductive process where
the space of abducibles is automatically constructed just from the input rules
alone. Given a (possibly empty) set of user provided facts, our method is able
to infer any additional facts that may be needed for the entailment of a query
and then output the set of abducibles, without the user needing to explicitly
specify the space of all abducibles. We also present a method to generate a
set of directed edges corresponding to the the justification graph for the
query. Furthermore, through different forms of term substitution, our method
is able to take user provided facts into account and suitably modify the
abductive solutions. Past work on abduction has been primarily based on goal
directed methods. However these methods can result in solvers that are not
truly declarative. Much less work has been done on realizing abduction in a
bottom up solver like the Clingo ASP solver. We describe novel ASP programs
which can be run directly in Clingo to yield the abductive solutions and
directed edge sets without needing to modify the underlying solving engine.

Given a n -ary predicate $p_{i}$ consider the complete level $N$ proof graph of $p_{i}(v_{1},v_{2},...,v_{n})$. Now suppose that given some $\langle R,q,U,\emptyset,N\rangle$, where $q$ is some instantiation of the predicate $p_{i}$, there exists a an answer set $A$ of $P_{<R,q,U,\emptyset,N>}^{Comp}$ such that $A$ contains the complete proof level $N$ graph of $p_{i}(t_{1},t_{2},...,t_{n})$ for some set of terms  $T = t_{1},..,t_{n}$. (Here the $t_{j}s$ need not be distinct)

Define the map $\phi$ by $\phi(v_{k})= t_{k}$. It follows that the complete level $N$ proof graph  of $p_{i}(t_{1},t_{2},...,t_{n})$ can be obtained from the complete level $N$ proof graph of $p_{i}(v_{1},v_{2},...,v_{n})$ by applying the map $\phi$ to the corresponding atoms of the proof graph. (This is ensured by condition 4 above which ensures that unification with the post-condition of a rule can always happen) Now say $query(p_{j}(t_{d_{1}},...,t_{d_{f}}),k)$ occurs in the complete level $N$ proof graph of $p_{i}(t_{1},t_{2},...,t_{n})$ where clearly $k\leq N$.

Suppose now we add the atom $query(p_{j}(a_{1},...,a{f}),k)$ to the ASP program $P_{<R,q,U,\emptyset,N>}^{Comp}$. Recall that $v_{d_{s}}$ is the pre-image of $t_{d_{s}}$ under $\phi$. Now say there exists a map $\psi$ mapping each $v_{d_{c}}$ to the corresponding $a_{c}$. Such a map may not always exist for example if it forces the map to be one - to - many)

Then there exists an answer set $A'$ of $P_{<R,q,U,\emptyset,N>}^{Comp}$ such that $A'$ contains the complete level $N$ proof graph for $p_{i}(b_{1},b_{2},...,b_{n})$, where each $b_{i}$ is the image of $v_{i}$ under the map $\phi'$ where $\phi'$ is defined as follows: for some $v_{g}$ if it is the pre-image of some $t_{d_{l}}$ then $\phi'$ of $v_{g}$ is $\psi(v_{g})$, otherwise $\phi'(v_{g}) = \phi(v_{g})= t_{g}$.

$\textit{Proof}$ First note that an answer set contains the complete level $N$ proof graph for $p_{i}(b_{1},b_{2},...,b_{n})$ if and only if it contains the atom $query(p_{i}(b_{1},b_{2},...,b_{n}),0)$. We will now prove this result by induction on $k$, the second argument of the atom $query(p_{j}(a_{1},...,a{f}),k)$. Firstly note that if $k = 0$ then $p_{j} = p_{i}$ and the map $\phi'$ is given by $\phi'(v_{i}) = a_{i}$. This establishes the result for $k=0$.  Say the result is true for all $k\leq o$, where $o<N$. Now say the complete level $N$ proof graph of $p_{i}(t_{1},t_{2},...,t_{n})$ contains the atom $query(p_{z}(t_{w_{1}},t_{w_{2}},..,t_{w_{r}}),o+1)$, and we add the atom $query(p_{z}(h_{1},h_{2},..,h_{r}),o+1)$ to the ASP program. Also suppose there exists the appropriate map $\psi$ from the pre-images of the $t_{w_{i}}s$ under $\phi$ to the corresponding $h_{i}s$. Then there exists a rule $r_{c}$ such that $p_{z}(t_{w_{1}},t_{w_{2}},..,t_{w_{r}})$ is a pre-condition of some appropriate instantiation of $r_{c}$ in the proof graph. Therefore there exists an atom of the form $create\_sub(subInst\_r_{c}(W),o+1)$ where $W$ is a tuple made up of variables in the instantiation of $r_{c}$ and $W$ is obtained by applying the map $\phi$ to the corresponding $create\_sub$ atom in the proof graph of $p_{i}(v_{1},v_{2},...,v_{n})$. Now a proof search expansion rule of the form E7 is instantiated. Specifically we get the following instantiation 
\begin{verbatim}
createSub(subInst_r_c(W'),o+1) :- createSub(subInst_r_c(W),o+1), query(p_z(h1,h2,..,hr),o+1),
max_ab_lvl(M).
\end{verbatim}

Here $W$ is as above and $W'$ is now given by the image of the tuple $I$ under
the appropriate map $\phi'$ where $I$ is the pre-image of $W$ under $\phi$ and
$\phi'$ is the map constructed from the maps $\phi$ and $\psi$ as described
before. Now say in the original proof graph for
$p_{i}(t_{1},t_{2},...,t_{n})$, the atom
$query(p_{u}(t_{h_{1}},t_{h_{2}},...t_{h_{q}}),o)$ corresponds to the post
condition of the particular instantiation of $r_{c}$ due to which
$query(p_{z}(t_{w_{1}},t_{w_{2}},..,t_{w_{r}}),o+1)$ occurs in the proof
graph. Note that because no rule in $R$ has existential variables in
pre-conditions, it must be the case that
$\{t_{w_{1}},...,t_{w_{r}}\}\subseteq\{t_{h_{1}},...,t_{h_{q}}\}$. Now because
of ASP rules of the form $query(...):-createSub(...)$ and the rule
\begin{verbatim}
query(Y,N-1):-explains(X,Y,N),0<N, N<M, max_ab_lvl(M).    
\end{verbatim}

There exists an atom $query(p_{u}(x_{1},x_{2},..,x_{q}),o)$ in the ASP program
where each $x_{i}$ is given by applying the map $\phi'$ from above on the
pre-image of the corresponding $t_{h_{i}}$ under $\phi$. So by the inductive
hypothesis we are done.


Proof strategy: Induction on $k$
\begin{enumerate}
    \item Prove result for $k =0 $
    \item Let $[t_{1},t_{2},..,t_{j}]$,$[a_{1},a_{2},...,a_{j}]$ be such that if $t_{x}\neq a_{x}$, then $e_{x}$ first appears in $G_{R,p,n}$ inside a $query$ atom whose integer argument is $k$.
    \item Prove general result by showing that $query(p_{i}(a_{1},a_{2},..,a_{j}),k)$ generates a $query$ atom with integer $k-1$, which also induces the same substitution except possibly for those $a_{m}s$ which refer to existential vars. So this allows us to use (2), and inductive hyp. for $k-1$ $query$ atom.
\end{enumerate}
\section{Novelty Factor}
Integration of abductive reasoning and 'why not' explanations. I don't think why-not explanations have been implemented interms of abductive reasoning in the literature. Rather the approach is to compute De Morgan duals of rules. Check Schulz's survey paper and some other more recent work like xclingo. 
\begin{verbatim}
https://hal.archives-ouvertes.fr/hal-03032897/document

https://arxiv.org/pdf/2009.10242v1.pdf
\end{verbatim}
Advantage of this explanation method is that we get a lot more control over depth of explanation, we get a more 'complete' why not explanation, it cn take user-inputs into account. Get some form of minimality in explanations. All this seems to neccesitate a very generalized abductive framework like the one presented.

\section{Papers to read}
Survey of Explanation approaches - Schulz etc.

See section 3.6.4 eg: "For example, justification approaches where all derivation steps are included in the
justification, that is all approaches other than LABAS justifications, may struggle
with the succinctness when explaining a large logic program, as explanations grow
with longer derivations. In contrast, LABAS justifications are independent of the
derivation length. However, a large logic program may also comprise more dependencies on negative literals, thus increasing the size of LABAS justifications. More
generally, it is an open problem how to effectively deal with the growing size (as
well as the previously mentioned exponential number) of justifications" The paper also mentions dealing with variables as only a partially dealt with challenge, and also mentions delaling with disjunction in rule heads as a potential challenge. However our system can infact be easily adapted to deal with disjunctions. 

See LABAS website
\begin{verbatim}
http://labas-justification.herokuapp.com/Tutorial.html    
\end{verbatim}
"The only approach able to handle non-normal
logic programs, i.e. logic programs with disjunctive heads, is the causal justification
approach, which can also deal with nested expressions in the body."

Open ASP, Abduction - PA Bonnati\\
Abduction with side effects - LM Pereira etc.\\
Abduction over unbounded domains via ASP - PA Bonnati\\ - Involves a complex procedure to reduce an input program to a 'relevant' sub-program. Only works on so called 'finitary programs'. Doesn't seem to be any support for explanation optimization. Requires one to define abducible skolem symbols from before hand.

CIFF proof procedure - F Toni etc.\\
Combining open world Description Logic reasonong with closed world ASP - Ian Horrocks etc.\\
sCASP papers\\
Focussing Proofs\\
Prolog based prover ie. LeanTAP\\
PTTP - Prolog Technology Theorem Prover\\
Open answer set Programming (OASP)\\
Implementing
Probabilistic Abductive Logic Programming
with Constraint Handling Rules - Henning Christiansen\\
Modelling variations of First Order Horn Abduction with Answer Set Programming- P.Schuller - Seems very similar to what I'm doing but there is no notion of integer 'levels' in the abductive proof thus easy to get into infinite backward search. Also, it is not integrated with an explanation mechanism as it is here. Lastly substitution of values for uninterpreted skolem functions is not really done automatically but rather handled through explicit equality. However overall this paper comes very close to the work here and we would need to study that paper carefully. (See FWD-A encoding) (Done, see file with tests on the complete encoding)

\subsection{Mapping distinct variables to the same skolem terms}
These extra proof search rules are needed to expand the proof search space when the rule set $R$ does not satify the unique arguments property. Say rule $r1$ involves 3 variables $X, Y, Z$. Then we add the following to the proof-search encoding:
\begin{verbatim}
createSub(subInst_r1(X,X,Z),M-1)    :-createSub(subInst_r1(X,Y,Z),N),max_ab_lvl(M).

createSub(subInst_r1(X,Y,X),M-1)    :-createSub(subInst_r1(X,Y,Z),N),max_ab_lvl(M).

createSub(subInst_r1(Y,Y,Z),M-1)    :-createSub(subInst_r1(X,Y,Z),N),max_ab_lvl(M).

createSub(subInst_r1(X,Y,Y),M-1)    :-createSub(subInst_r1(X,Y,Z),N),max_ab_lvl(M).

createSub(subInst_r1(Z,Y,Z),M-1)    :-createSub(subInst_r1(X,Y,Z),N),max_ab_lvl(M).

createSub(subInst_r1(X,Z,Z),M-1)    :-createSub(subInst_r1(X,Y,Z),N),max_ab_lvl(M).
\end{verbatim}


%%% Local Variables:
%%% mode: latex
%%% TeX-master: "main"
%%% End:
