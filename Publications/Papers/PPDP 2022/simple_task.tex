\section{Simple Abductive Proof Generation Task}

We shall define here the notion of a
\textit{Simple Abductive Proof Generation Task}, as all of our formal results
will apply to this restricted class of abductive proof generation tasks.

\begin{definition}[Simple Abductive Proof Generation Task]\label{simpletask}
Given an abductive proof generation task $\langle R,q,U,C,N \rangle$, we say that this task is a $\textit{simple abductive proof generation task}$ if the following hold:
\begin{enumerate}
\item $R$ contains no negation as failure.
\item $R$ contains no function symbols, arithmetic operators.
\item No post condition of any rule in $R$ contains repeated variables. For example the rule:\\
$p(X,X):-r(X,X,Y)$ is not allowed but the rule: $p(X,Y):-r(X,X,Y)$ is allowed.
\item $C$ is empty.
\item Any constraint on abducibles in $U_{a}$ must consist of only a single
  positive fully un-ground atom with no repeated variables amongst its
  arguments. For example if $p$ is a binary predicate, then the constraint
  $:-abducedFact(p(X,Y)).$ is allowed but the constraint
  $:-abducedFact(p(X,X)).$ Constraints where more than one atom appers are
  also not allowed. For instance the following would be disallowed:
  $:-abducedFact(p(X,Y)),abducedFact(r(X)).$ Finally, constraints containing
  partially or fully ground atoms are disallowed. For example the following
  would be disallowed\\ $:-abducedFact(p(james,Y)).$
\item If $U_{a}$ is such that no instance of some predicate $p$ can be
  abduced, then $U_{f}$ must not contain any instantiation of $p$.
\item $q$ must be positive and fully ground.
\end{enumerate}
\end{definition}


%%% Local Variables:
%%% mode: latex
%%% TeX-master: "main"
%%% End:

