\section{Proof sketch of Finiteness}\label{sec:proof_finiteness}

\begin{theorem}[Finiteness]\label{thm:finiteness}
Assume that $\langle R,q,U,C,N \rangle$ is such that $R$ is function free and
also contains no arithmetic operations. $N$ is finite. $q$ is a ground
positive atom containing no function symbols. $U$ contains some integrity
constraints and ground positive atoms containing no function symbols. Then
$P_{\langle R,q,U,C,N \rangle}^{res}$ cannot have infinite answer sets. 
\end{theorem}

\begin{lemma}
Let $S$ be the set of uninterpreted skolem functions appearing in the proof
search encoding. Let $C$ be the set of constants occurring in either a user
inputed fact in $U$ or the query $q$. Let $T_{k}$ denote the set of unique
terms that can be constructed from $S$ and $C$ with skolem depth atmost
$k$. Let $T$  be the possibly infinite set consisting of terms of unbounded
depth.\remms{Better write: $T$ is the union of the $T_k$. However, the lemma does not say anything about $T$} Then for any positive integer $k$, $T_{k}$ is finite.
\end{lemma}

$Proof:$ This is a standard result which follows easily by induction. Clearly $|T_{0}|$ = $|C|$, which is finite. Let $|T_{k}|$ = $B$ be finite. Let $S$ be a set of $l$ functions having arity at most $d$. Then $|T_{k+1}|\leq B + lB^{d}$. Hence $T_{k+1}$ is finite.  

\begin{lemma}
 For any predicate $p$ inside an $abducedFact$ atom the arguments of $p$ are elements of the set $T_{N}$. Similarly for any predicate $p$ inside an $holds$ atom the arguments of $p$ are elements of the set $T_{N}$. Similarly for the other atoms in any answer set of $P_{\langle R,q,U,C,N \rangle}^{res}$. Terms that correspond to arguments of rule predicates are always elements of $T_{N}$.
\end{lemma}

$Proof$ $Sketch$: Firstly note that for any predicate $p$ inside a
$abducedFact$ or $holds$ atom, where $p$ comes from the original rule set, the
arguments of $p$ belong to the set $T$. We can see this by observing that
since no rule in $R$ contains a function symbol, the encoding of the rules
themselves such as in line 2 of the ASP program above, cannot introduce new
terms in the arguments of predicates inside $holds$ atoms. Similarly lines
like 24, 25 also cannot introduce new terms inside $holds$, $create\_sub$
atoms in the final answer set, which means also that no new terms appear as
predicate arguments inside $abducedFact$ atoms. 

Only lines like 18,19 are able to create fresh terms that become arguments of
various atoms in the answer set. But then it follows that any $holds$ atom,
$create\_sub$, $abducedFact$ can only have predicate arguments from the set $T$.

Next we show that the skolem depth of these terms is at most $N$. Any fresh
term $t$ from $T$ appearing in the final answer set must have been constructed
from rules like in lines 18,19. But these encode a 'static' space of
abducibles, which unlike lines 24,25 is independent of the forward reasoning
component or any user provided additional facts. It is not hard to see that
terms created via these ASP rules must have skolem depth at most $N$. 
\remms[inline]{For formal reasons, I would not refer to an example in a proof, but rather make the argument general (``in phase AG1 of the translation''). For illustration, you can then come back to the example and say: this general behavior can also be seen to hold in the proof.}

\begin{lemma}
Since the set $T_{N}$ is finite it follows that any answer set of $P_{\langle
  R,q,U,C,N \rangle}^{res}$ is finite.
\end{lemma}

$Proof:$ Since $T_{N}$ is finite and the integer argument of any $create\_sub$
atom is bounded by $N$, any atom in the final answer set has only finitely
many instantiations. This proves finiteness of the resulting answer set.

%%% Local Variables:
%%% mode: latex
%%% TeX-master: "main"
%%% End:
