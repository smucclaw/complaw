\section{Abductive Proof Generation Task}\label{sec:abductive_proof}

\subsection{Formal Setup}\label{formalsetup}


\begin{definition}[Abductive Proof Generation Task]\label{def:abductive_proof_generation_task}

Given a source ASP rule set $R$, consider the tuple $\langle R,q,U,C,N \rangle$, which we will refer to as the $\textit{Abductive Proof Generation Task}$. In this tuple,  $R$ denotes a set of input ASP rules, which we shall also refer to as the input rules or the source rules throughout the rest of this paper. A possibly non-ground or partially ground positive atom or, a ground negation-as-failure atom $q$. $q$ intuitively represents the goal of our abductive reasoning process. In the context of an abductive proof generation task we may also sometimes refer to $q$ as the $query$. 
%\remms{say what intuitively $q$, $U$ etc. mean} 
The set $U$ consists of 2 subsets, $U = U_{f} \cup U_{a}$. Here $U_{f}$ is a set of user provided facts. $U_{a}$ is a set of integrity constraints that prevents certain atoms from being abduced. Throughout the rest of this paper we may sometimes just refer to the set $U$ as a whole making it clear what is contained in the subsets. $C$ denotes a set of ASP constraints which constrain which atoms may or may not appear in the complete model that results from the rules, user provided facts and abducibles. Finally we have the non-negative integer $N$. This acts as the depth control parameter for abductive proof generation. 
 \end{definition}
Given an abductive proof generation task $\langle R,q,U,C,N \rangle$,let us define what we mean by a $\textit{General Solution}$ to the task

\begin{definition}[General Solution]\label{def:abductive_proof_generation_solution} \
Given an abductive proof generation task $\langle R,q,U,C,N \rangle$, we say that $F$ is a general solution to this task if:
\begin{enumerate}
    \item If $q$ is a positive atom, it is in some answer set $\mathcal{A}$ of $F\cup U_{f}\cup R\cup C$ where the un-ground variables in $q$ have been replaced with any set of ground terms. If $q$ is a ground NAF atom, then there exists an answer set $\mathcal{A}$ of $F\cup U_{f}\cup R\cup C$ such that $q$ is $not$ in $\mathcal{A}$.
    \item $F$ does not violate any of the integrity constraints in $U_{a}$ Ie. $F$ does not contain abducibles that are specifically disallowed by the constraints in $U_{a}$.
    \item $F$ does not have any atoms whose depth level is greater than $N$.
\end{enumerate}
\end{definition} We will next state some assumptions we make on the set of input ASP rules $R$ and then also define what we mean by the depth level of an atom. After this we shall exemplify all these definitions with an example. 
\subsection{Input ASP program}
When considering any abductive proof generation task we make the following assumptions on the input ASP rule set $R$. We assume that each source ASP rule has exactly the following form:
\begin{lstlisting}[frame=none]
pre_con_1(V1),pre_con_2(V2),...,pre_con(Vk),
    not pre_con(Vk+1),...,not pre_con_n(Vn) -> post_con(V).
\end{lstlisting}
We further make the folllowing assumptions:

\begin{enumerate}
    \item Each pre-condition $pre\_con_{i}(V_{i})$ is atomic and so is the post-condition $post\_con(V)$.
    \item The $not$ in front of the pre-conditions denotes \textit{negation as failure} interpreted under the \textit{stable model semantics}
    \item $V_{i}$ is the set of variables occuring in the $i^{th}$ pre-condition which is either $pre\_con_{i}(V_{i})$ or $not$ $pre\_con_{i}(V_{i})$ and $V$ is the set of variables occuring in the post condition $post\_con(V)$. We assume that $V\subseteq V_{1}\cup V_{2}\cup ... \cup V_{n}$.
    \item Each variable occurring in the post condition is universally quantified over, and each variable that occurs in some pre-condition but not the post condition is existentially quantified. In particular together with (3), this means that there are no existentially quantified variables in rule post-conditions.
    \item Each variable that occurs in a negation-as-failure pre-condition also occurs in some positive pre-condition.
    \item Each input rule of the form above is assigned some unique rule id.
\end{enumerate}

We further assume that given any integrity constraint $c$ in $C$, every variable in a negation-as-failure atom in $c$ also occurs in some positive atom in $c$. We will now define the depth level of an atom with respect to a given input source ASP rule set $R$ and query $q$.

\begin{definition}[Depth level of an atom]\label{depthlvl}
Given a ASP rule set $R$ and some ground or partially ground positive atom $q$ which we shall call the query, we define a map $\phi_{R,q}$ that maps an arbitrary positive atom to a set of non-negative integers. We
will describe this map rather informally. For any atom  $q'$ such that $q'$ is obtained from $q$ by replacing the variables in $q$, with some ground terms we have, $0\in
\phi_{R,q}(q')$. Now the rest of the definition is recursive. Given an atom $a$, the
non-negative integer $n$, $n\in \phi_{R,q}(a)$ if and only if there exists a rule
$r$ in $R$ and some substitution $\theta$ of the variables in $r$
such that there exists some precondition (NAF or positive) $p$ of $r$ such
that that $\theta$ applied to $p$ gives $a$ and $\theta$ applied to the post
condition of $r$ gives some atom $a'$ where $n-1$ $\in$ $\phi_{R,q}(a')$.

Given an atom $a$ let $\phi^{min}_{R,q}(a)$ be $-1$ if the set $\phi_{R,q}(a)$ is empty
and let $\phi^{min}_{R,q}(a)$ be the minimum member of the set $\phi_{R,q}(a)$
otherwise. Then $\phi^{min}_{R,q}(a)$ is defined to be the $depth$ $level$ of $a$
with respect to the rule set $R$ and query $q$. If $q$ is a ground $NAF$ atom then given some some positive atom $a$ $\phi^{min}_{R,q}(a)$ is simply given by $\phi^{min}_{R,\tilde{q}}(a)$, where $\tilde{q}$ is obtained from $q$ by removing the $not$ operator from in front of $q$.  
\end{definition}
For example if $R$ consisted of the rules
\begin{lstlisting}[frame=none]
a(X):-b(X).
b(X):-a(X).
\end{lstlisting} and $q$ was $a(X)$, meaning $q$ is un-ground then we would have that given any term $t$, $\phi_{R,q}^{min}(a(t)) = 0$, $\phi_{R,q}^{min}(b(t)) = 1$, and for any other atom $p$, $\phi_{R,q}^{min}(p) = -1$.  


%%% Local Variables:
%%% mode: latex
%%% TeX-master: "main"
%%% End:
