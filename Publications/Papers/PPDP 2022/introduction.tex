\section{Introduction}\label{sec:introduction}

The goal of this paper is to describe a method to automatically genenrate a
space of abducibles, that will allow the creation of a fact set that results
in the entailment of a given query with respect to a given ASP rule set.

A novel feature of the method is that depth control for proof search is done
in a purely declarative way as part of the proof search encoding
itself. Furthermore, adding facts to the program automatically triggers a kind
of controlled variable substitution where skolem\remms{Capitalize?} terms or
other 'place-holder' terms occurring in abducibles are replaced away so that
the resulting proof is simplified, without any need for an explicit
representation of equality between terms. We present two main sets of
proof-search encodings. One is guaranteed to only produce finite answer sets,
but which does not support complete variable substitution in abducibles. The
other encoding may produce infinite answer sets in the presence of skolem
functions but supports complete variable substitution and still only produces
finite answer sets if there are no skolem terms involved. Next we present an
encoding which generates a set of directed edges representing a justification
graph of any desired depth. \remms{Set off the related work section from
  general intro}The novel encodings for variable substitution and depth
control set this work apart from other attempts like [Schuller ref, Petr
Homola ref] to realize abduction in a Clingo-like bottom-up reasoner. Unlike
sCASP our method keeps the justification production and proof search separate
from each other which allows the user to have greater control on the
justification produced without affecting the proof search
procedure. Furthermore, in general, abduction problems where the input rules
involve function symbols often either lead to infinite grounding or infinite
backward proof search and problems that are intractable with a top-down
reasoning mechanism often yield to a bottom-up reasoner and vice
versa.\remms{Sounds as if reasoning style does not matter} Lastly, as
mentioned in the abstract, there may be use cases where one wants to know all
the resulting consequences of an abductive solution to a query with respect to
a rule-set. Informally one can relate this to the problem of
\emph{bi-abduction} which has been explored in separation logic [Bi-ab
ref]. Hence our work can be regarded as addressing the problem of
\emph{bi-abduction} but in an ASP setting. (However we shall not dwell on this
connection any further, instead focus only on the abductive reasoning
process). Therefore an abductive reasoner based on a solver like Clingo can
complement the abilities of goal directed reasoners like sCASP, CIFF etc.

The rest of the paper is organised as
follows. Section~\ref{sec:abductive_proof} provides some further background
and motivation for the subject of the paper and also describes the problem
being tackled more formally. Section~\ref{sec:derived_asp} presents 3 core
sets of meta-rules that facilitate the abductive proof generation
process. Section~\ref{sec:sample_problem} describes some formal results about
the generated proof trees. Section~\ref{sec:proof_finiteness} discusses
applications and comparisons to other work and Section~\ref{sec:conclusion}
concludes.

Test citations:\remms{Remove} \cite{lim22:_autom_defeas_reason_law,mahajan22:_overv_cclaw_l4}

%%% Local Variables:
%%% mode: latex
%%% TeX-master: "main"
%%% End:
