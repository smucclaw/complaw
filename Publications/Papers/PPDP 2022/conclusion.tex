\section{ Conclusions}\label{sec:conclusion}
We have presented several encodings for abductive proof generation in ASP,
incorporating notions of depth control and  novel implementations of term
substitution. We have also given an encoding that allows one to generate a set
of directed edges representing a justification graph.

It seems to us that some of the ideas involved in the term substituion
mechanism are similar to the ideas involved when one uses $\textit{Sideways
  Information Passing Strategies}$ \cite{beeri91} to re-write datalog rules
for more efficient evaluation of queries by incorporating elements of top-down
reasoning. However we have not explored this connection in detail. Those techniques 
typically involve a complete re-write of the input
rules according some chosen fixed sideways information passing strategy, which
makes that whole approach quite different to ours. \cite{DBLP:journals/jar/Stickel94} describes an approach 
to doing abductive reasoning in a bottom up manner. He uses 'continuation predicates' to pass substitutions from previously evaluated rule pre-conditions to rule pre-conditions yet to be evaluated, given the rule post-condition as the 'goal'. This somewhat resembles our use of 'createSub' predicates. However it seems to us that
that approach 
imposes a strict order on the evaluation of preconditions of a rule, which makes 
that method much less general than ours. 

There are several possible directions for future work. One possible line of theoretical investigation 
could be to study how the abductive solutions calculated by our methods (and any resulting extra
consequences of the abductive solution and input rules) could be generalised to 
sentences in first order logic. Roughly speaking, given an abductive solution involving
instances of 'extVar' the aim would be to map these solutions to solutions where instances
of 'extVar' are replaced by universally quantified variables (where perhaps such a variable may
not take values from some finite set).  Distinguishing between instances of 'extVar' that should 
get mapped to distinct universally quantified variables can be done by adding certain facts 
that would result in the generated abductive solution being modified so that the 'matching' 
occurrences of 'extVar' get replaced by some other fresh constant. Intuitively, it seems to 
us that our method of calculating and simplifying abductive solutions without grounding over
the entire domain of constants gives an appropriate 
setting to explore some of these ideas. Of course the correctness/applicability of such techniques 
would have to be investigated in a rigorous and formal manner.


Another possible future line of work may include extending the formal results
presented here to a larger class of abductive proof generation problems. 
It also seems to us that the main technique used to generate the directed edge
set representing the justification graph could be adapted for use in SAT/SMT
solvers to get justifications out of them. 

We also have yet to study the complexity problems
associated with the methods presented in this paper. \cite{DBLP:journals/tcs/EiterGL97} provides a thorough study 
of the complexity of abductive reasoning. It remains to be seen how those results, 
many of which deal with propositional logic, could be carried over to our setting, where
we aim to compute abductive solutions without complete grounding of rules. %\remms[inline]{Also mention Haskell implemenation}




%%% Local Variables:
%%% mode: latex
%%% TeX-master: "main"
%%% End:
