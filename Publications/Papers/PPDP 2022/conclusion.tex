\section{Future and Related work, Conclusion}\label{sec:conclusion}

\section{Novelty Factor}
Integration of abductive reasoning and 'why not' explanations. I don't think why-not explanations have been implemented interms of abductive reasoning in the literature. Rather the approach is to compute De Morgan duals of rules. Check Schulz's survey paper and some other more recent work like xclingo. 
\begin{verbatim}
https://hal.archives-ouvertes.fr/hal-03032897/document

https://arxiv.org/pdf/2009.10242v1.pdf
\end{verbatim}
Advantage of this explanation method is that we get a lot more control over depth of explanation, we get a more 'complete' why not explanation, it cn take user-inputs into account. Get some form of minimality in explanations. All this seems to neccesitate a very generalized abductive framework like the one presented.

\section{Papers to read}
Survey of Explanation approaches - Schulz etc.

See section 3.6.4 eg: "For example, justification approaches where all derivation steps are included in the
justification, that is all approaches other than LABAS justifications, may struggle
with the succinctness when explaining a large logic program, as explanations grow
with longer derivations. In contrast, LABAS justifications are independent of the
derivation length. However, a large logic program may also comprise more dependencies on negative literals, thus increasing the size of LABAS justifications. More
generally, it is an open problem how to effectively deal with the growing size (as
well as the previously mentioned exponential number) of justifications" The paper also mentions dealing with variables as only a partially dealt with challenge, and also mentions delaling with disjunction in rule heads as a potential challenge. However our system can infact be easily adapted to deal with disjunctions. 

See LABAS website
\begin{verbatim}
http://labas-justification.herokuapp.com/Tutorial.html    
\end{verbatim}
"The only approach able to handle non-normal
logic programs, i.e. logic programs with disjunctive heads, is the causal justification
approach, which can also deal with nested expressions in the body."

Open ASP, Abduction - PA Bonnati\\
Abduction with side effects - LM Pereira etc.\\
Abduction over unbounded domains via ASP - PA Bonnati\\ - Involves a complex procedure to reduce an input program to a 'relevant' sub-program. Only works on so called 'finitary programs'. Doesn't seem to be any support for explanation optimization. Requires one to define abducible skolem symbols from before hand.

CIFF proof procedure - F Toni etc.\\
Combining open world Description Logic reasonong with closed world ASP - Ian Horrocks etc.\\
sCASP papers\\
Focussing Proofs\\
Prolog based prover ie. LeanTAP\\
PTTP - Prolog Technology Theorem Prover\\
Open answer set Programming (OASP)\\
Implementing
Probabilistic Abductive Logic Programming
with Constraint Handling Rules - Henning Christiansen\\
Modelling variations of First Order Horn Abduction with Answer Set Programming- P.Schuller - Seems very similar to what I'm doing but there is no notion of integer 'levels' in the abductive proof thus easy to get into infinite backward search. Also, it is not integrated with an explanation mechanism as it is here. Lastly substitution of values for uninterpreted skolem functions is not really done automatically but rather handled through explicit equality. However overall this paper comes very close to the work here and we would need to study that paper carefully. (See FWD-A encoding) (Done, see file with tests on the complete encoding)



%%% Local Variables:
%%% mode: latex
%%% TeX-master: "main"
%%% End:
