\section{ Conclusions}\label{sec:conclusion}
We have presented several encodings for abductive proof generation in ASP,
incorporating notions of depth control and  novel implementations of term
substitution. We have also given an encoding that allows one to generate a set
of directed edges representing a justification graph.

It seems to us that some of the ideas involved in the term substituion
mechanism are similar to the ideas involved when one uses $\textit{Sideways
  Information Passing Strategies}$ \cite{beeri91} to re-write datalog rules
for more efficient evaluation of queries by incorporating elements of top-down
reasoning. However we have not explored this connection in detail. It seems to
us that those techniques typically involve a complete re-write of the input
rules according some chosen fixed sideways information passing strategy, which
makes that whole approach quite different to ours.

Possible future directions of work may include extending the formal results
presented here to a larger class of abductive proof generation problems. It
also seems to us that the main technique used to generate the directed edge
set representing the justification graph could be adapted for use in SAT/SMT
solvers to get justifications out of them. This too could be a potentially
interesting avenue for future work.
%\remms[inline]{Also mention Haskell implemenation}




%%% Local Variables:
%%% mode: latex
%%% TeX-master: "main"
%%% End:
