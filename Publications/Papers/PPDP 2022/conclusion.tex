\section{ Conclusions}\label{sec:conclusion}
We have presented several encodings for abductive proof generation in ASP,
incorporating notions of depth control and  novel implementations of term
substitution. We have also given an encoding that allows one to generate a set
of directed edges representing a justification graph.

It seems to us that some of the ideas involved in the term substituion
mechanism are similar to the ideas involved when one uses $\textit{Sideways
  Information Passing Strategies}$ \cite{beeri91} to re-write datalog rules
for more efficient evaluation of queries by incorporating elements of top-down
reasoning. However we have not explored this connection in detail. Those techniques 
typically involve a complete re-write of the input
rules according some chosen fixed sideways information passing strategy, which
makes that whole approach quite different to ours. \cite{DBLP:journals/jar/Stickel94} describes an approach 
to doing abductive reasoning in a bottom up manner. However it seems to us that
that approach 
imposes a strict order on the evaluation of preconditions of a rule, which makes 
that method much less general than ours. 
Possible future directions of work may include extending the formal results
presented here to a larger class of abductive proof generation problems. It
also seems to us that the main technique used to generate the directed edge
set representing the justification graph could be adapted for use in SAT/SMT
solvers to get justifications out of them. We also have yet to study the complexity problems
associated with the methods presented in this paper. \cite{DBLP:journals/tcs/EiterGL97} provides a thorough study 
of the complexity of abductive reasoning. It remains to be seen how those results 
which mostly deal with propositional logic could be carried over to our setting, where
we aim to compute abductive solutions without grounding rules over the complete 
domain of constants. %\remms[inline]{Also mention Haskell implemenation}




%%% Local Variables:
%%% mode: latex
%%% TeX-master: "main"
%%% End:
