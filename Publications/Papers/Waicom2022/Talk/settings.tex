\mode<presentation>
{
  \usetheme{Darmstadt}
  \useoutertheme{infolines}
}

\usepackage{alltt}

\usepackage{etex}         % to avoid compilation erros of xypic
\usepackage[curve]{xypic}
%\xyoption{pdf}
%\usepackage[pdf]{xy}


\usepackage[utf8]{inputenc}
\usepackage{proof}
\inferLineSkip=4pt  % increase spacing between lines; default is 2pt

\usepackage{amsmath}
\usepackage{amssymb}
\usepackage{fontawesome}
\usepackage{times}
\usepackage[T1]{fontenc}
\usepackage{listings}
%\lstset{language=haskell,basicstyle=\ttfamily}

\usepackage{multicol}
\usepackage{mathpartir}
\usepackage{mathtools}
\usepackage{stmaryrd}
\usepackage{soul}
\usepackage{tikzsymbols}

% Delete this, if you do not want the table of contents to pop up at
% the beginning of each subsection:
\AtBeginSection[]
{
  \begin{frame}<beamer>
    \frametitle{Plan}
    \tableofcontents[sectionstyle=show/shaded,subsectionstyle=hide]
  \end{frame}
}

\AtBeginSubsection[]
{
  \begin{frame}<beamer>
    \frametitle{Plan}
    \tableofcontents[sectionstyle=show/hide,subsectionstyle=show/shaded/hide]
%    \tableofcontents[subsectionstyle=show/shaded/hide]
  \end{frame}
}

\setbeamertemplate{footline}
{%
%\begin{beamercolorbox}{section in head/foot}
%\vskip2pt\insertnavigation{\paperwidth}\vskip2pt
%\end{beamercolorbox}%
\insertpagenumber
\insertshorttitle[width={5cm},center]
\insertshortinstitute[width={3cm},center]
\insertshortdate[width={3cm},center]
}


% If you wish to uncover everything in a step-wise fashion, uncomment
% the following command: 

%\beamerdefaultoverlayspecification{<+->}

\usepackage{tikz}
\usetikzlibrary{trees}
\usetikzlibrary{arrows}
\usetikzlibrary{decorations.pathmorphing}
\usetikzlibrary{shapes.multipart}
\usetikzlibrary{shapes.geometric}
\usetikzlibrary{calc}
\usetikzlibrary{positioning} 
\usetikzlibrary{fit}
\usetikzlibrary{backgrounds}
\usetikzlibrary{automata}


% ----------------------------------------------------------------------
% lstlisting setup for L4

\renewcommand{\lstlistingname}{Code}


\definecolor{dkgreen}{rgb}{0,0.6,0}
\definecolor{gray}{rgb}{0.5,0.5,0.5}
\definecolor{mauve}{rgb}{0.58,0,0.82}

% \lstset{frame=tb,
%   language=Java,
%   aboveskip=3mm,
%   belowskip=3mm,
%   showstringspaces=false,
%   columns=flexible,
%   basicstyle={\footnotesize\ttfamily},
%   numbers=none,
%   numberstyle=\tiny\color{gray},
%   keywordstyle=\color{blue},
%   commentstyle=\color{dkgreen},
%   stringstyle=\color{mauve},
%   breaklines=true,
%   breakatwhitespace=true,
%   tabsize=2
% }



% Configuration of lstlisting for L4
\lstdefinelanguage{L4}
{morekeywords={
    assert
    , assign
    , bool
    , chan
    , class
    , clock
    , decl
    , defn
    , derivable
    , else
    , exists
    , extends
    , fact
    , false
    , for
    , forall
    , guard
    , if
    , in
    , init
    , let
    , lexicon
    , not
    , process
    , rule
    , state
    , system
    , then
    , trans
    , true
},
sensitive=false,
morecomment=[l]{\#},
morestring=[b]",
}


\lstset{%frame=tb,
  language=L4,
  aboveskip=3mm,
  belowskip=3mm,
  showstringspaces=false,
  columns=flexible,
  basicstyle={\footnotesize\ttfamily},
  numbers=none,
  numberstyle=\tiny\color{gray},
  keywordstyle=\color{blue},
  commentstyle=\color{dkgreen},
  stringstyle=\color{mauve},
  breaklines=true,
  breakatwhitespace=true,
  tabsize=2,
  keepspaces=true
}

%%% Local Variables: 
%%% mode: latex
%%% TeX-master: "main"
%%% End: 
