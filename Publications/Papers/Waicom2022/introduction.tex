%----------------------------------------------------------------------
\section{Introduction}

With the present paper, we intend to provide solutions to the problem of
specifying and enforcing compliance in particular in state-based,
time-dependent systems. Our solutions are far from complete or conclusive --
our short note is meant to present a snapshot of our current work and to
contribute to a discussion about relevant issues: when are legal requirements
coherent (so that they can be implemented)? How can compliance be technically
enforced? How can violations be detected and error scenarios be communicated?

Our contribution first presents a case study that we have been working on at
Singapore's Centre for Computational Law,
CCLAW\footnote{\url{https://cclaw.smu.edu.sg/}}, dealing with a particularly
relevant fragment of publicly available legislation and regulation:
Singapore's Personal Data Protection Act. It will be described informally  in
\secref{sec:case_study_pdpa}. We will then proceed to a formal analysis in
\secref{sec:formal_analysis}, highlighting some problems of the current
legislation that could be termed an internal inconsistency and that has in
part been revealed by the formal analysis. We then conclude with a description
of desiderata on a language and verification framework for reasoning about
compliance, in \secref{sec:discussion}. 

Our approach continues work on ``contracts as automata''
\cite{flood_goodenough_contract_as_automaton_2022} and has similarities with
other automata-based approaches for reasoning about contracts, in particular
\cite{azzopardi_pace_schapachnik_schneider_contract_automata_2016,parvizimosaed_roveri_rasti_amyot_logrippo_mylopoulos_model_checking_symboleo_2022}. 


% \cite{governatori_shalt_is_not_will_2015}



%%% Local Variables:
%%% mode: latex
%%% TeX-master: "main"
%%% End:
