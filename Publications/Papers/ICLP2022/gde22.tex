\documentclass[]{ceurart}
%%
%% One can fix some overfulls
\sloppy


% Packages from LIPICS

\usepackage{fancyhdr}

\usepackage{amsmath}
\usepackage{amssymb}
\usepackage[T1]{fontenc}
\usepackage[utf8]{inputenc}

\usepackage{alltt}
\usepackage{makeidx}  % allows for indexgeneration
\usepackage{listings}
\usepackage{caption}
%\usepackage{subcaption}  % does not seem to work with llncs
\usepackage{subfig}  % does not seem to work with llncs
%\usepackage[pdftex]{graphicx}
%\usepackage{enumerate}
\usepackage{mathpartir}
%\usepackage{isabelle,isabellesym}
%\isabellestyleit
\usepackage{todonotes}
\usepackage{wrapfig}
\usepackage{framed}
\usepackage{mathtools}
\usepackage{stmaryrd}
\usepackage{soul}

\usepackage{rail}    % For railroad syntax diagrams


\setcounter{secnumdepth}{3}
    
\usepackage{tikz}
\usetikzlibrary{arrows.meta}
\usetikzlibrary{trees}
\usetikzlibrary{graphs}
\usetikzlibrary{arrows}
\usetikzlibrary{decorations.pathmorphing}
\usetikzlibrary{shapes.multipart}
\usetikzlibrary{shapes.geometric}
\usetikzlibrary{shapes.misc}
\usetikzlibrary{calc}
\usetikzlibrary{positioning} 
\usetikzlibrary{fit}
\usetikzlibrary{backgrounds}


\usepackage[colorlinks,hyperindex,bookmarks,linkcolor=blue,citecolor=blue,urlcolor=blue]{hyperref}


%4 mars 2016 setting for Java code template
\renewcommand{\lstlistingname}{Code}


\definecolor{dkgreen}{rgb}{0,0.6,0}
\definecolor{gray}{rgb}{0.5,0.5,0.5}
\definecolor{mauve}{rgb}{0.58,0,0.82}

% \lstset{frame=tb,
%   language=Java,
%   aboveskip=3mm,
%   belowskip=3mm,
%   showstringspaces=false,
%   columns=flexible,
%   basicstyle={\footnotesize\ttfamily},
%   numbers=none,
%   numberstyle=\tiny\color{gray},
%   keywordstyle=\color{blue},
%   commentstyle=\color{dkgreen},
%   stringstyle=\color{mauve},
%   breaklines=true,
%   breakatwhitespace=true,
%   tabsize=2
% }



% Configuration of lstlisting for L4
\lstdefinelanguage{L4}
{morekeywords={
    assert
    , assign
    , chan
    , class
    , clock
    , decl   
    , defn   
    , derivable
    , else 
    , exists
    , extends
    , fact   
    , false
    , for  
    , forall
    , guard
    , if   
    , in
    , init
    , let   
    , lexicon
    , not   
    , process
    , rule
    , state
    , system
    , then
    , trans
    , true 
},    
sensitive=false,
morecomment=[l]{\#},
morestring=[b]",
}


\lstset{frame=tb,
  language=L4,
  aboveskip=3mm,
  belowskip=3mm,
  showstringspaces=false,
  columns=flexible,
  basicstyle={\footnotesize\ttfamily},
  numbers=none,
  numberstyle=\tiny\color{gray},
  keywordstyle=\color{blue},
  commentstyle=\color{dkgreen},
  stringstyle=\color{mauve},
  breaklines=true,
  breakatwhitespace=true,
  tabsize=2,
  keepspaces=true
}

\lstdefinelanguage{L4Sugary}
{  morekeywords={EVERY,UNLESS,UPON,WITHIN,MUST,LEST,MAY,BEFORE,PARTY,HENCE}
,  keepspaces=true
}


\setlength{\intextsep}{5pt}
\setlength{\textfloatsep}{5pt}

%-----------------------------
%%% Local Variables: 
%%% mode: latex
%%% TeX-master: "main"
%%% End: 


% Definition of colors
\definecolor{mauve}{rgb}{0.88, 0.69, 1.0}
\newcommand{\blue}[1]{{\color{blue}#1}}
\newcommand{\green}[1]{{\color{green}#1}}
\newcommand{\red}[1]{{\color{red}#1}}
\newcommand{\mauve}[1]{{\color{mauve}#1}}

% Remark macros for the authors
\newcommand{\remam}[2][]{\todo[color=blue!40,#1]{AM: #2}}
\newcommand{\remms}[2][]{\todo[color=green!40,#1]{MS: #2}}
\newcommand{\remmww}[2][]{\todo[color=blue!20,#1]{MWW: #2}}

% Common abbreviations
\newcommand{\eg}{\textit{e.g.\ }}
\newcommand{\etal}{\textit{et al.\ }}
\newcommand{\etc}{\textit{etc}}
\newcommand{\ie}{\textit{i.e.\ }}
\newcommand{\viz}{\textit{viz.\ }}
\newcommand{\wrt}{\textit{w.r.t.\ }}


%%% Local Variables: 
%%% mode: latex
%%% TeX-master: "main"
%%% End: 

\newtheorem{example}{Example}
\newtheorem{proposition}{Proposition}

\usepackage{amsmath}
\usepackage{graphicx}
\usepackage{multirow}


%%
%% Minted listings support 
%% Need pygment <http://pygments.org/> <http://pypi.python.org/pypi/Pygments>
\usepackage{listings}
%% auto break lines
\lstset{breaklines=true}

%%
%% end of the preamble, start of the body of the document source.

\begin{document}


%% Rights management information.
%% CC-BY is default license.
\copyrightyear{2022}
\copyrightclause{Copyright for this paper by its authors.
  Use permitted under Creative Commons License Attribution 4.0
  International (CC BY 4.0).}

%% This command is for the conference information
\conference{2nd Workshop on Goal-directed Execution of Answer Set
  Programs (GDE'22), August 1, 2022}

\title{Automating Defeasible Reasoning in Law with Answer Set Programming}

\author{Lim How Khang}[%
style=chinese,
orcid=0000-0002-9333-1364
%email=name@org.com,
%url=http://www. ,
]
%
\author{Avishkar Mahajan}[%
orcid=0000-0002-9925-1533
%email=name@org.com,
]
%
\author{Martin Strecker}[%
orcid=0000-0001-9953-9871
%email=name@org.com,
]
%
\author{Meng Weng Wong}[%
orcid=0000-0003-0419-9443
%email=name@org.com,
]
\address{Singapore Management University}

\begin{abstract}
The paper studies defeasible reasoning in rule-based systems, in particular
about legal norms and contracts. We identify rule modifiers that specify how
rules interact and how they can be overridden. We then define rule
transformations that eliminate these modifiers, leading in the end to a
translation of rules to formulas. For reasoning with and about rules, we
contrast two approaches, one in a classical logic with SMT solvers, which is
only briefly sketched, and one using non-monotonic logic with Answer Set
Programming solvers, described in more detail.
\end{abstract}

\begin{keywords}
  Knowledge representation and reasoning
  \sep
  Argumentation and law
  \sep
  Computational Law
  \sep
  Defeasible reasoning
\end{keywords}

\maketitle




%----------------------------------------------------------------------

%----------------------------------------------------------------------
\section{Introduction}

With the present paper, we intend to provide solutions to the problem of
specifying and enforcing compliance in particular in state-based,
time-dependent systems. Our solutions are far from complete or conclusive --
our short note is meant to present a snapshot of our current work and to
contribute to a discussion about relevant issues: when are legal requirements
coherent (so that they can be implemented)? How can compliance be technically
enforced? How can violations be detected and error scenarios be communicated?

Our contribution first presents a case study that we have been working on at
Singapore's Centre for Computational Law,
CCLAW\footnote{\url{https://cclaw.smu.edu.sg/}}, dealing with a particularly
relevant fragment of publicly available legislation and regulation:
Singapore's Personal Data Protection Act. It will be described informally  in
\secref{sec:case_study_pdpa}. We will then proceed to a formal analysis in
\secref{sec:formal_analysis}, highlighting some problems of the current
legislation that could be termed an internal inconsistency and that has in
part been revealed by the formal analysis. We then conclude with a description
of desiderata on a language and verification framework for reasoning about
compliance, in \secref{sec:discussion}. 

Our approach continues work on ``contracts as automata''
\cite{flood_goodenough_contract_as_automaton_2022} and has similarities with
other automata-based approaches for reasoning about contracts, in particular
\cite{azzopardi_pace_schapachnik_schneider_contract_automata_2016,parvizimosaed_roveri_rasti_amyot_logrippo_mylopoulos_model_checking_symboleo_2022}. 


% \cite{governatori_shalt_is_not_will_2015}



%%% Local Variables:
%%% mode: latex
%%% TeX-master: "main"
%%% End:


% file intentionally commented out, text now integrated into l4_language_app.tex
% \section{The L4 Language}\label{sec:l4_language}

This section gives an account of the L4 language as it is currently defined --
as an experimental language, L4 will evolve over the next months. In our
discussion, we will ignore some features such as a natural language interface
\cite{} that are not relevant for the topic of this paper.

As a language intended for representing legal texts and reasoning about them,
an L4 module is essentially composed of four sections:
\begin{itemize}
\item a terminology in the form of \emph{class definitions};
\item \emph{declarations} of functions and predicates;
\item \emph{rules} representing the core of a law text, specifying what is
  considered as legal behaviour;
\item \emph{assertions} for stating and proving properties about the rules.
\end{itemize}

In the following, these elements will be presented in more detail and
illustrated with a running example, a (fictitious) reglementation of speed
limits for different types of vehicles.


% ----------------------------------------------------------------------
\subsection{Class Definitions}\label{sec:classdefs}

The definition in \figref{fig:classdefs} introduces classes for vehicles, days
and roads. 

\begin{figure}[h]
  \begin{framed}
\begin{alltt}
class Vehicle \{
   weight: Integer
\}
class Car extends Vehicle \{
   doors: Integer
\}
class Truck extends Vehicle

class Day
class Workday extends Day
class Holiday extends Day

class Road
class Highway extends Road
\end{alltt}
\end{framed}
  \caption{Class definitions of speedlimit example}\label{fig:classdefs}
\end{figure}

Classes are arranged in a tree-shaped hierarchy, having a class named
\texttt{Class} as its top element. Classes that are not explicitly derived
from another class via \texttt{extends} are implicitly derived from
\texttt{Class}. A class $S$ derived from a class $C$ by \texttt{extends} will
be called a subclass of $C$, and the immediate subclasses of \texttt{Class}
will be called \emph{sorts} in the following. Intuitively, classes are meant
to be sets of entitities, with subclasses being interpreted as
subsets. Different subclasses of a class are not meant to be disjoint.

Class definitions can come with attributes, in braces. These attributes can be
of simple type, as in the given example, or of higher type (the notion of type
will be explained in \secref {sec:fundecls}). In a declarative reading,
attributes can be understood as a shorthand for function declarations that
have the class they are defined in as additional codomain. Thus, the attribute
\texttt{weight} corresponds to a top-level declaration \texttt{weight: Vehicle
  -> Integer}. In a more operational reading, L4 classes can be understood as
prototypes of classes in object-oriented programming languages, and an
alternative field selection syntax can be used: For \texttt{v: Vehicle}, the
expression \texttt{v.weight} is equivalent to \texttt{weight(v)}, at least
logically, even though its operational interpretation may differ.


% ----------------------------------------------------------------------
\subsection{Function Declarations}\label{sec:fundecls}

L4 is an \emph{explicitely} and \emph{strongly typed} language: all entities
such as functions, predicates and variables have to be declared before being
used. One purpose of this measure is to ensure that the executable sublanguage
of L4, based on the simply-typed lambda calculus with subtyping, enjoys a type
soundness property: evaluation of a function cannot produce a dynamic type
error.



\begin{figure}[h]
\begin{framed}

\begin{alltt}
\end{alltt}

\end{framed}
  \caption{Declarations of speedlimit example}\label{fig:fundecls}
\end{figure}




% ----------------------------------------------------------------------
\subsection{Rules}\label{sec:rules}

\begin{figure}[h]
\begin{framed}

\begin{alltt}
\end{alltt}

\end{framed}
  \caption{Rules of speedlimit example}\label{fig:rules}
\end{figure}



% ----------------------------------------------------------------------
\subsection{Assertions}\label{sec:assertions}

\begin{figure}[h]
\begin{framed}

\begin{alltt}
\end{alltt}

\end{framed}
  \caption{Assertions of speedlimit example}\label{fig:assertions}
\end{figure}




%%% Local Variables:
%%% mode: latex
%%% TeX-master: "main"
%%% End:


% ----------------------------------------------------------------------
\section{Reasoning with and about Rules}\label{sec:resasoning_with_rules}
% ----------------------------------------------------------------------

We can only give a brief outline of the L4 rule format here and defer a more
thorough discussion of the L4 language to \ref{sec:l4_language_app}.
We will illustrate the main concepts with an example, a (fictitious) regulation of speed
limits for different types of vehicles, subdivided into class \texttt{Car}
and its subclass \texttt{SportsCar}, furthermore classes \texttt{Day} and \texttt{Road}. We will
in particular be interested in specifying the maximal speed \texttt{maxSp} of
a vehicle on a particular day and type of road, and this will be the purpose
of the rules.

In its most complete form, a \emph{rule} is composed of a list of variable
declarations introduced by the keyword \texttt{for}, a precondition introduced
by \texttt{if} and a post-condition introduced by
\texttt{then}. \figref{fig:rules} gives an example of rules of our speed limit
scenario, stating, respectively, that the maximal speed of cars is 90 km/h on a
workday,
and that they may drive at 130 km/h if the road is a highway.  Note that in
general, both pre- and post-conditions are Boolean formulas that can be
arbitrarily complex, thus are not limited to conjunctions of literals in the
preconditions or atomic formulas in the post-conditions.
Rules whose precondition is \texttt{true} can be written as \texttt{fact}.

\begin{figure}[h!]
  \begin{lstlisting}
rule <maxSpCarWorkday> 
   for v: Vehicle, d: Day, r: Road
   if isCar v && isWorkday d
   then maxSp v d r 90
rule <maxSpCarHighway>
   for v: Vehicle, d: Day, r: Road
   if isCar v && isHighway r
   then maxSp v d r 130
\end{lstlisting}
  \caption{Rules of speed limit example}\label{fig:rules}
\end{figure}

\begin{figure}[h]
\begin{lstlisting}
assert <maxSpFunctional> {SMT: {valid}}
   maxSp instCar instDay instRoad instSpeed1 &&
   maxSp instCar instDay instRoad instSpeed2
   --> instSpeed1 == instSpeed2
\end{lstlisting}
  \caption{Assertions of speedlimit example}\label{fig:assertions}
\end{figure}


The purpose of our formalization efforts is to be able to make assertions
and prove them, such as the statement in \figref{fig:assertions} which claims
that the predicate 
\texttt{maxSp} behaves like a function, \ie{} given the same car, day and
road, the speed will be the same. Instead of a universal quantification, we
here use variables \texttt{inst...} that have been declared globally, because they
produce more readable (counter-)models. 

Given a plethora of different notions of \emph{defeasibility}, we had to make a
choice as to which notions to support, and which semantics to give to them. We
will here concentrate on two concepts, which we call \emph{rule modifiers},
that limit the applicability of rules and make them ``defeasible''. They will
be presented informally in the following. Giving them a precise semantics in
classical, monotonic logic is the topic of \secref{sec:defeasible_classical};
a semantics based on Answer Set Programming will be provided in
\secref{sec:defeasible_asp}.

We will concentrate on two rule modifiers that restrict the applicability of
rules and that frequently occur in law texts: \emph{subject to} and
\emph{despite}. A motivating discussion justifying their informal semantics is
given in \appref{sec:resasoning_with_rules_app}, drawn from a detailed
analysis of Singapore's Professional Conduct Rules. In the following, however,
we return to our running example.

\begin{example}
  The rules \texttt{maxSpCarHighway} and \texttt{maxSpCarWorkday} are not
  mutually exclusive and contradict another because they postulate different
  maximal speeds. For disambiguation, we would like to say:
  \texttt{maxSpCarHighway} holds \emph{despite} rule
  \texttt{maxSpCarWorkday}. In L4, rule modifiers are introduced with the aid
  of \emph{rule annotations}, with a list of rule names following the keywords
  \texttt{subjectTo} and \texttt{despite}. Thus, we modify rule
  \texttt{maxSpCarHighway} of \figref{fig:rules} with
\begin{lstlisting}
rule <maxSpCarHighway>
  {restrict: {despite: maxSpCarWorkday}}
# rest of rule unchanged
\end{lstlisting}
Furthermore, to the delight of the public of the country with the highest
density of sports cars, we also introduce a new rule \texttt{maxSpSportsCar}
that holds \emph{subject to} \texttt{maxSpCarWorkday} and \emph{despite}
\texttt{maxSpCarHighway}:
\begin{lstlisting}
rule <maxSpSportsCar>
  {restrict: {subjectTo: maxSpCarWorkday, 
              despite: maxSpCarHighway}}
   for v: Vehicle, d: Day, r: Road
   if isSportsCar v && isHighway r
   then maxSp v d r 320
 \end{lstlisting}
\end{example}

We will now give an informal characterization of these modifiers:
\begin{itemize}
\item $r_1$ \emph{subject to} $r_2$ and $r_1$ \emph{despite} $r_2$ are complementary
  ways of expressing that one rule may override the other rule. They have in
  common that $r_1$ and $r_2$ have contradicting conclusions. The conjunction
  of the conclusions can either be directly unsatisfiable (such as: ``may hold'' vs.{}
  ``must not hold'') or unsatisfiable \wrt{} an intended background theory
  (obtaining different maximal speeds is inconsistent when expecting
  \texttt{maxSp} to be functional in its fourth argument).
\item Both modifiers differ in that \emph{subject to} modifies the rule to which
  it is attached, whereas \emph{despite} has a remote effect on the rule given
  as argument.
\item They permit to structure a legal text, favouring conciseness and
  modularity: In the case of \emph{despite}, the overridden, typically more
  general rule need not be aware of the overriding, typically subordinate rules.
\item Even though these modifiers appear to be mechanisms on the meta-level in
  that they reasoning about rules, they can directly be reflected on the
  object-level.
\end{itemize}

%%% Local Variables:
%%% mode: latex
%%% TeX-master: "main"
%%% End:


%% ----------------------------------------------------------------------
\section{Defeasible Reasoning in a Classical Logic}\label{sec:defeasible_classical}
% ----------------------------------------------------------------------

In this section, we will describe how to give a precise semantics to the rule
modifiers, by rewriting rules, progressively eliminating the instructions
appearing in the rule annotations so that in the end, only purely logical
rules remain. Making the meaning of the modifiers explicit can therefore be
understood as a \emph{compilation} problem.  Whereas the first preprocessing
steps (\secref{sec:preprocessing}) are generic, we will discuss two variants
of conversion into logical format (\secrefs{sec:restr_precond} and
\ref{sec:restr_deriv}). We will then discuss rule inversion
(\secref{sec:rule_inversion}) which gives our approach a non-monotonic flavour
while remaining entirely in a classical setting. Rule inversion will also be
instrumental for comparing the conversion variants in \secref{sec:comparison}.


% ----------------------------------------------------------------------
\subsection{Rule Modifiers in Classical Logic}\label{sec:rule_modifiers_in_classical_logic}

% ......................................................................
\subsubsection{Preprocessing}\label{sec:preprocessing}

Preprocessing consists of several elimination steps that are carried out in a
fixed order.

\paragraph{\textbf{``Despite''  elimination}}

As can be concluded from the previous discussion, a
$\mathtt{despite}\; r_2$ clause appearing in rule $r_1$ is equivalent to a
$\mathtt{subjectTo}\; r_1$ clause in rule $r_2$. The first rule transformation
consists in applying exhaustively the following \emph{despite elimination}
rule transformer:
% \remms[inline]{In the whole discussion (and the implementation), make a
%   clearer distinction between rule set transformer \texttt{restrict} and rule
%   generator / transformer \texttt{derived}.}

\noindent
\emph{despiteElim:}\\
$
\{r_1 \{\mathtt{restrict}: \{\mathtt{despite}\; r_2\} \uplus a_1\},\;\;
r_2\{\mathtt{restrict}: a_2\}, \dots\} \longrightarrow$\\
$\{r_1 \{\mathtt{restrict}: a_1\},\;\;
r_2\{\mathtt{restrict}:  \{\mathtt{subjectTo}\; r_1\} \uplus a_2\}, \dots\}
$

\begin{example}\label{ex:rewrite_despite}
Application of this rewrite rule to the three example rules \texttt{maxSpCarWorkday},
\texttt{maxSpCarHighway} and  \texttt{maxSpSportsCar} changes them to:

\begin{lstlisting}
rule <maxSpCarWorkday>
  {restrict: {subjectTo: maxSpCarHighway}}
rule <maxSpCarHighway>
  {restrict: {subjectTo: maxSpSportsCar}}
rule <maxSpSportsCar>
  {restrict: {subjectTo: maxSpCarWorkday}}
\end{lstlisting}
Here, only the headings are shown, the bodies of the rules are
unchanged. 
\end{example}

One defect of the rule set already becomes apparent to the human reader at
this point: the circular dependency of the rules. We will however continue
with our algorithm, applying the next step which will be to rewrite the
\texttt{\{restrict: \{subjectTo: \dots\}\}} clauses.  Please note that each
rule can be \texttt{subjectTo} several other rules, each of which may have a
complex structure as a result of transformations that are applied to it.


\paragraph{\textbf{``Subject'' To elimination}}

The rule transformer \emph{subjectToElim} does the following: it splits up the
rule into two rules, (1) its source (the rule body as originally given), and
(2) its definition as the result of applying a rule transformation function to
several rules.

\begin{example}\label{ex:rewrite_subject_to}
Before stating the rule transformer, we show its effect on rule
\texttt{maxSpCarWorkday} of \exampleref{ex:rewrite_despite}. On rewriting
with \emph{subjectToElim}, the rule is transformed into two rules:

\begin{lstlisting}
# new rule name, body of rule unchanged
rule <maxSpCarWorkday'Orig>
   {source}
   for v: Vehicle, d: Day, r: Road
   if isCar v && isWorkday d
   then maxSp v d r 90

# rule with header and without body
rule <maxSpCarWorkday>
 {derived: {apply: 
 {restrictSubjectTo maxSpCarWorkday'Orig  maxSpSportsCar}}}
\end{lstlisting}
\end{example}

We can now state the transformation (after grouping the
\texttt{subjectTo $r_2$}, \dots, \texttt{subjectTo $r_n$} into
\texttt{subjectTo $[r_2 \dots r_n]$}):


\noindent
\emph{subjectToElim:}\\
$
\{r_1 \{\mathtt{restrict}: \{\mathtt{subjectTo}\; [r_2, \dots, r_n]\}\}, \dots \} \longrightarrow$\\
$\{r_1^o \{\mathtt{source}\}, r_1 \{\mathtt{derived:}\; \{\mathtt{apply:}\; \{
\mathtt{restrictSubjectTo}\;\; r_1^o\; [r_2 \dots r_n] \}\}\}, \dots \}
$



\paragraph{\textbf{Computation of derived rule}}
The last step consists in generating the derived rules, by evaluating the
value of the rule transformer expression marked by \texttt{apply}. The rules
appearing in these expressions may themselves be defined by complex
expressions. However, direct or indirect recursion is not allowed. For
simplifying the expressions in a rule set, we compute a rule dependency order
$\prec_R$ defined by: $r \prec_R r'$ iff $r$ appears in the defining
expression of $r'$. If $\prec_R$ is not a strict partial order (in particular, if
it is not cycle-free), then evaluation fails. Otherwise, we order the rules
topologically by $\prec_R$ and evaluate the expressions starting from the
minimal elements. Obviously, the order $\prec_R$ does not prevent rules from being recursive.

\begin{example}
It is at this point that the cyclic dependence already remarked after
\exampleref{ex:rewrite_despite} will be discovered. We have:

\noindent
\texttt{maxSpCarWorkday'Orig}, \texttt{maxSpCarHighway} $\prec_R$ \texttt{maxSpCarWorkday}\\
\texttt{maxSpCarHighway'Orig}, \texttt{maxSpSportsCar} $\prec_R$ \texttt{maxSpCarHighway}\\
\texttt{maxSpSportsCar'Orig}, \texttt{maxSpCarWorkday} $\prec_R$  \texttt{maxSpSportsCar}\\
\noindent
which cannot be totally ordered.

Let us fix the problem by changing the heading of rule
\texttt{maxSpCarHighway} from \texttt{despite} to \texttt{subjectTo}:
\begin{lstlisting}
rule <maxSpCarHighway>
  {restrict: {subjectTo: maxSpCarWorkday}}
\end{lstlisting}

After rerunning \emph{despiteElim} and \emph{subjectToElim}, we can now order
the rules:

% \begin{lstlisting}
% rule <maxSpCarWorkday>
% rule <maxSpCarHighway>
%   {restrict: {subjectTo: maxSpCarWorkday, maxSpSportsCar}}
% rule <maxSpSportsCar>
%   {restrict: {subjectTo: maxSpCarWorkday}}
% \end{lstlisting}

\noindent
\{ \texttt{maxSpSportsCar'Orig}
\texttt{maxSpCarHighway'Orig},
\texttt{maxSpCarWorkday} \} $\prec_R$
\texttt{maxSpSportsCar} $\prec_R$
\texttt{maxSpCarHighway}\\
and will use this order for rule elaboration.
\end{example}


% ......................................................................
\subsubsection{Restriction via Preconditions}\label{sec:restr_precond}

Here, we propose one possible implementation of the rule transformer
\texttt{restrictSubjectTo} introduced in \secref{sec:preprocessing} that takes
a rule $r_1$ and a list of rules $[r_2 \dots r_n]$ and produces a new rule, by
adding the negation of the preconditions of $[r_2 \dots r_n]$ to $r_1$. More
formally:

\begin{itemize}
\item $\mathtt{restrictSubjectTo}\; r_1\; [] = r_1$
\item $\mathtt{restrictSubjectTo}\; r_1\; (r' \uplus rs) =$\\
  $\mathtt{restrictSubjectTo}\; (r_1(precond := precond(r_1) \AND \NOT precond(r')))\; rs$
\end{itemize}
where $precond(r)$ selects the precondition of rule $r$ and $r(precond:=p)$
updates the precondition of rule $r$ with $p$.

There is one proviso to the application of \texttt{restrictSubjectTo}: the
rules have to have the same \emph{parameter interface}: the number and types
of the parameters in the rules' \texttt{for} clause have to be the same.
Rules with different parameter interfaces can be adapted via the
\texttt{remap} rule transformer. The rule 

\begin{lstlisting}[frame=none,mathescape=true]
rule <r> for $x_1$:$T_1$ $\dots$ $x_n$:$T_n$ if Pre($x_1, \dots, x_n$) then Post($x_1, \dots, x_n$)
\end{lstlisting}
is remapped by \texttt{remap r [$y_1: S_1, \dots, y_m: S_m$] [$x_1 := e_1, \dots, x_n := e_n$]}
to 
\begin{lstlisting}[frame=none,mathescape=true]
rule <r> for $y_1$: $S_1$ $\dots$ $y_m$: $S_m$ if Pre($e_1, \dots, e_n$) then Post($e_1, \dots, e_n$)
\end{lstlisting}
Here, $e_1, \dots, e_n$ are expressions that have to be well-typed with types $E_1, \dots, E_n$ in
context $y_1: S_1, \dots, y_m: S_m$ (which means in particular that they may
contain the variables $y_i$) with $E_i \preceq T_i$,  (with the consequence that
the pre- and post-conditions of the new rule remain well-typed), where $\preceq$ is subtyping. 


\begin{example}
We come back to the running example. When processing the rules in the order of
$\prec_R$, rule \texttt{maxSpSportsCar}, defined by
\texttt{apply: \{restrictSubjectTo maxSpSportsCar'Orig maxSpCarWorkday\}},
becomes:
\begin{lstlisting}
rule <maxSpSportsCar>
   for v: Vehicle, d: Day, r: Road
   if isSportsCar v && isHighway r &&
      not (isCar v && isWorkday d)
   then maxSp v d r 320
 \end{lstlisting}

 We can now state \texttt{maxSpCarHighway}, which has been defined by
 \texttt{apply: \{restrictSubjectTo maxSpCarHighway'Orig maxSpSportsCar\}}, as:

 \begin{lstlisting}
rule <maxSpCarHighway>
   for v: Vehicle, d: Day, r: Road
   if isCar v && isHighway r &&
      not (isSportsCar v && isHighway r &&
              not (isCar v && isWorkday d))  &&
      not (isCar v && isWorkday d))
   then maxSp v d r 130
\end{lstlisting}
\end{example}

One downside of the approach of adding negated preconditions is that the
preconditions of rules can become very complex. This effect is mitigated by
the fact that conditions in \texttt{subjectTo} and \texttt{despite} clauses
express specialisation or refinement and often permit substantial
simplifications. Thus, the precondition of \texttt{maxSpSportsCar} simplifies
to \texttt{isSportsCar v \&\& isHighway r \&\& isWorkday d} and the
precondition of \texttt{maxSpCarHighway} to
\texttt{isCar v \&\& isHighway r \&\& not (isSportsCar v \&\& isWorkday d)}.


% ......................................................................
\subsubsection{Restriction via Derivability}\label{sec:restr_deriv}

We now give an alternative reading of \texttt{restrictSubjectTo}. To
illustrate the point, let us take a look at a simple propositional example.
\remms{Move into appendix altogether. Save 1.5 pages}

\begin{example}\label{ex:small_propositional} Take the definitions:
\begin{lstlisting}
rule <r1> if B1 then C1
rule <r2> {subjectTo: r1} if B2 then C2
\end{lstlisting}
\end{example}

Instead of saying: \texttt{r2} corresponds to
\texttt{\blue{if} B2 \&\& not B1 \blue{then} C2} 
as in \secref{sec:restr_precond}, we would now read it as
``if the conclusion of \texttt{r1} cannot be derived'', 
which could be written as
\texttt{\blue{if} B2 \&\& not C1 \blue{then} C2}.
The two main problems with this naive approach are the following:
\begin{itemize}
\item As mentioned in \secref{sec:resasoning_with_rules}, a \emph{subject to}
  restriction is often applied to rules with contradicting conclusions, so in
  the case that \texttt{C1} is \texttt{not C2}, the generated rule would be a
  tautology.
\item In case of the presence of a third rule
\begin{lstlisting}[frame=none]
rule <r3> if B3 then C1
\end{lstlisting}
a derivation of \texttt{C1} from \texttt{B3} would also block the application
of \texttt{r2}, and \texttt{subjectTo: r1} and \texttt{subjectTo: r1, r3}
would be indistinguishable.
\end{itemize}

We now sketch a solution for rule sets whose conclusion is always an atom (and
not a more complex formula).

\begin{enumerate}
\item In a preprocessing stage, all rules are transformed as follows:
  \begin{enumerate}
  \item We assume the existence of classes \texttt{Rulename$_P$}, one for each
    transformable predicate $P$ (see below).
  \item All the predicates $P$
    occurring in the conclusions of rules (called \emph{transformable
      predicates}) are converted into predicates $P^+$ with one additional
    argument of type \texttt{Rulename$_P$}. In the
    example, \texttt{C1$^+$: Rulename$_{C1}$ -> Boolean} and similarly for \texttt{C2}.
  \item The transformable predicates $P$ in conclusions of rules receive one
    more argument, which is the name \emph{rn} of the rule: $P$ is transformed
    into $P^+\; rn$. The informal reading is ``the predicate is derivable with
    rule \emph{rn}''.
  \item All transformable predicates in the preconditions of the rules receive
    one more argument, which is a universally quantified variable of type
    \texttt{Rulename$_P$} of the appropriate type, bound in the
    \texttt{for}-list of the rule.
  \end{enumerate}
\item In the main processing stage, \texttt{restrictSubjectTo} in the rule
  annotations generates rules according to:
  \begin{itemize}
\item $\mathtt{restrictSubjectTo}\; r_1\; [] = r_1$
\item $\mathtt{restrictSubjectTo}\; r_1\; (r' \uplus rs) =$\\
  $\mathtt{restrictSubjectTo}\; (r_1(precond := precond(r_1) \AND \NOT postcond(r')))\; rs$
  Thus, the essential difference \wrt{} the definition of
  \secref{sec:restr_precond} is that we add the negated post-condition and not
  the negated pre-condition.
\end{itemize}
\end{enumerate}

\begin{example} The rules of \exampleref{ex:small_propositional} are now
  transformed to:
\begin{lstlisting}[mathescape=true]
rule <r1> for rn:Rulename$_{B1}$ if B1$^+$ rn then C1$^+$ r1
rule <r2> for rn:Rulename$_{B2}$ if B2$^+$ rn and not C1$^+$ r1 then C2$^+$ r2
\end{lstlisting}
The derivability of another instance of \texttt{C1}, such as \texttt{C1$^+$ r3},
would not inhibit the application of \texttt{r2} any more.
\end{example}


\begin{example} The two rules of the running example become, after resolution
  of the \texttt{restrictSubjectTo} clauses:
\begin{lstlisting}[mathescape=true]
rule <maxSpSportsCar>
   for v: Vehicle, d: Day, r: Road
   if isSportsCar v && isHighway r &&
      not maxSp$^+$ maxSpCarWorkday v d r 90
   then maxSp$^+$ maxSpSportsCar v d r 320
rule <maxSpCarHighway>
   for v: Vehicle, d: Day, r: Road
   if isCar v && isHighway r &&
      not maxSp$^+$ maxSpCarWorkday v d r 90 &&
      not maxSp$^+$ maxSpSportsCar v d r 320
   then maxSp$^+$ maxSpCarHighway v d r 130
\end{lstlisting}
\end{example}

% ----------------------------------------------------------------------
\subsection{Rule Inversion}\label{sec:rule_inversion}

The purpose of this section is to derive formulas that, for a given rule set,
simulate negation as failure, but are coded in a classical first-order logic,
do not require a dedicated proof engine (such as Prolog) and can be checked
with a SAT or SMT solver. The net effect is similar to the completion
introduced by \cite{clark_NegAsFailure_1978}; however, the justification
is not operational as in \cite{clark_NegAsFailure_1978}, but takes inductive
closure as a point of departure. Some of the ideas are reminiscent of least
fixpoint semantics of logic programs, as discussed in
\cite{falaschi_etal_declarative_logic_langauges_1989,fages_consistency_clark_completion_1994}.
The discussion below applies to a considerably wider class of formulas.
\remms{Shorten or remove. Save 1 page}

In the following, we assume that our rules have an atomic predicate $P$ as
conclusion, whereas the precondition $Pre$ can be an arbitrarily complex
formula.  We furthermore assume that rules are in \emph{normalized form}: $P$
may only be applied to $n$ distinct variables $x_1, \dots, x_n$, where $n$ is
the arity of $P$, and the rule quantifies over exactly these variables.
For notational simplicity, we write normalized rules in logical format,
ignoring types:
$\forall x_1, \dots, x_n. Pre(x_1, \dots x_n) \IMPL Post(x_1, \dots, x_n)$.

Every rule can be written in normalized form, by applying the following
algorithm:
\begin{itemize}
\item Remove expressions or duplicate variables in the conclusion, by using
  the equivalences $P(\dots e \dots) = (\forall x. x = e \IMPL P(\dots x
  \dots))$ for a fresh variable x, and similarly $P(\dots y \dots y \dots) =
  (\forall x. x = y   \IMPL P(\dots x \dots y \dots))$.
\item Remove variables from the universal quantifier prefix if they do not
  occur in the conclusion, by using the equivalence
  $(\forall x. Pre(\dots x \dots) \IMPL P) = (\exists x. Pre(\dots x \dots))
  \IMPL P$.
\end{itemize}

For any rule set $\cal R$ and predicate $P$, we can form the set of
$P$-rules, ${\cal R}[P]$, as
\begin{align*}
\{ & \forall x_1, \dots, x_n. Pre_1[P](x_1, \dots x_n) \IMPL P(x_1, \dots, x_n),
     \dots,\\
  & \forall x_1, \dots, x_n.Pre_k[P](x_1, \dots x_n)\IMPL P(x_1, \dots,
x_n)\}
\end{align*}
as the subset of $\cal R$ containing all rules having $P$ as
post-condition. The notation $F[P]$ is meant to indicate that the $F$ can
contain $P$. It can also be taken as a \emph{functional}, \ie{} a higher-order
function having $P$ as parameter.

We say that a functional $F$ is \emph{semantically monotonic} if
\[
  (\forall x_1, \dots, x_n.\; P(x_1, \dots, x_n) \IMPL P'(x_1, \dots, x_n)) \IMPL 
  (\forall \overrightarrow{v}. F[P] \IMPL F[P'])
\]
A sufficient condition for semantic monotonicity is syntactic monotonicity:
$P$ does not occur under an odd number of negations in $F$. 

The inductive closure of a set of $P$-rules is the predicate $P^*$ defined by
the second-order formula
\[  P^*(x_1, \dots, x_n) = \forall P.\; (\bigwedge{\cal R}[P]) \IMPL P(x_1, \dots x_n) \]
where $\bigwedge{\cal R}[P]$ is the conjunction of all the rules in ${\cal R}[P]$.

$P^*$ can be understood as the least predicate satisfying the set of $P$-rules
and is the predicate that represents ``all that is known about $P$ and
assuming nothing else about $P$ is true'', and corresponds to the notion of
exhaustiveness prevalent in law texts. It can also be understood as the static
equivalent of the operational concept of negation as failure for predicate
$P$.  By the Knaster-Tarski theorem, $P^*$, as the least fixpoint of a
monotonic functional, is consistent (see a counterexample in
\exampleref{ex:syntactically_non_monotonic_rule}). 

Obviously, a second-order formula such as the definition of $P^*$ is unwieldy
in fully automated theorem proving, so we derive one particular consequence:

\begin{proposition}\label{lemma:p_star}
$P^*(x_1, \dots, x_n) \IMPL Pre_1[P^*](x_1, \dots x_n) \OR \dots \OR Pre_k[P^*](x_1, \dots x_n)$
\end{proposition}
% \begin{proof}
% Expand the definition of $P^*$ and instantiate the universal variable $P$ with
% $\lambda x_1 \dots x_n. Pre_1(x_1, \dots x_n) \OR \dots \OR Pre_k(x_1, \dots x_n)$.
% \end{proof}
As a consequence of the Löwenheim–Skolem theorem, there is no first-order
equivalent of $P^*$: a formula of the form $P^*$ can characterize the natural
numbers up to isomorphism, but no first-order formula can.

In the absence of such a first-order equivalent, we define the formula $Inv_P$
\[
\forall x_1, \dots, x_n.\;  P(x_1, \dots, x_n) \IMPL Pre_1(x_1, \dots x_n) \OR \dots \OR Pre_k[P](x_1, \dots x_n)
\]
called the \emph{inversion formula of  $P$}, and take it as an approximation of the
effect of $P^*$ in \propositionref{lemma:p_star}.

As usual, a disjunction over an empty set is taken to be the falsum
$\bot$. Assume there are no defining rules for a predicate $P$, then $Inv_P =
P \IMPL \bot = \NOT P$, which corresponds to a closed-world assumption for $P$.

\begin{example}\label{ex:syntactically_non_monotonic_rule}
  One motivation for the monotonicity constraint is the following: The
  simplest example of a rule that is not syntactically monotonic is
  $\NOT P \IMPL P$. Its inversion is $P \IMPL \NOT P$. The two formulas
  together, $P \IFF \NOT P$, are inconsistent.
\end{example}

Inversion formulas can be automatically derived and added to the rule set in
L4 proofs; they turn out to be essential for consistency properties. For
example, the functionality of \texttt{maxSp} stated in \figref{fig:assertions}
is not provable without the inversion formula of \texttt{maxSp}.

To avoid misunderstandings, we should emphasize that this approach is entirely
based on a classical monotonic logic, in spite of non-monotonic
effects. Adding a new $P$-rule may invalidate previously provable facts, but
this is only so because the new rule alters the inversion formula of $P$.



% ----------------------------------------------------------------------
\subsection{Comparison}\label{sec:comparison}

One may wonder whether, starting from the same set of rules, the transformations in
\secref{sec:restr_precond} and \secref{sec:restr_deriv} produce
equivalent rules. On the face of it, this is not so, because the
transformation via derivability modifies the arity of the predicates, so the
rule sets have different models.

We will however show that the two rule sets have corresponding sets of
models. This will be made more precise in the following. To fix notation,
assume ${\cal R}_M$ to be a set of rules annotated with rule modifiers. Let
${\cal R}_P$ be the set of rules obtained from ${\cal R}_M$ through the rule
translation via preconditions of \secref{sec:restr_precond}, and similarly
${\cal R}_D$ the set of rules obtained from ${\cal R}_M$ through the rule
translation via derivability of \secref{sec:restr_deriv}. From these rule
sets, we obtain formula sets ${\cal F}_P$ respectively ${\cal F}_D$ by
\begin{itemize}
\item translating rules to formulas;
\item adding inversion formulas $Inv_C$ for all
  the transformable predicates $C$ of the rule set;
% \item in the case of ${\cal F}_D$, adding exhaustivity predicates for all the
%   \texttt{Rulename$_C$} types\remms{Parameterize Rulename types with
%     predicates}, of the form $\forall x: \mathtt{Rulename}_C.\; x = rn_1 \OR
%   \dots \OR x=rn_r$ if $\{rn_1, \dots, rn_r\}$ are the rule names having $C$
%   as conclusion.\remms{Maybe not required?}
\end{itemize}


\begin{proposition}\label{lemma:mp_to_md}
  Any model ${\cal M}_P$ of ${\cal F}_P$ can be transformed into a model
  ${\cal M}_D$ of ${\cal F}_D$.
\end{proposition}

\begin{proposition}\label{lemma:md_to_mp}
  Any model ${\cal M}_D$ of ${\cal F}_D$ can be transformed into a model
  ${\cal M}_P$ of ${\cal F}_P$.
\end{proposition}

The proofs will be given in \ref{sec:comparison_proofs}, respectively after
\propositionref{lemma:mp_to_md_with_proof} and 
\propositionref{lemma:md_to_mp_with_proof}.
In the proof of \propositionref{lemma:md_to_mp}, the
inversion formulas play a decisive role.


%%% Local Variables:
%%% mode: latex
%%% TeX-master: "main"
%%% End:


\section{Defeasible Reasoning with Answer Set Programming}\label{sec:defeasible_asp}



\subsection{Introduction}
The purpose of this section is to give an account of the work we have been doing using Answer Set Programming (ASP) to formalize and reason about legal rules. This approach is complementary to the one described before using SMT solvers. Our intention is to present how some core legal reasoning tasks can be implemented in ASP while keeping the ASP representation readable and intuitive and respecting the idea of having an `isomorphism' between the rules and the encoding. Please see the appendix for a brief overview of ASP and references for further reading.

Our work in this section is inspired by \cite{DBLP:conf/iclp/WanGKFL09} and we borrow some of their notation/terminology. Readers will note that there are similarities between the use of predicates such as $according\_to$, $defeated$, $opposes$ in our ASP encoding, to reason about rules interacting with each other, and similar predicates that the authors of \cite{DBLP:conf/iclp/WanGKFL09} use in their work. However our ASP implementation is much more specific to legal reasoning whereas they seek to implement very general logic based reasoning mechanisms. We independently developed our `meta theory' for how rule modifiers interact with the rules and with each other and there are further original contributions like a proposed axiom system for what we call `legal models'. An interesting avenue of future work could be to compare our approaches within the framework of legal reasoning.

The work in this section builds on the work in \cite{morris21:_const_answer_set_progr_tool} and hence uses many of the same predicates/notation and terminology. 
% The author of \cite{morris21:_const_answer_set_progr_tool} was a member of the same research group as the authors of this paper at SMU in 2020--2021.


\subsection{Formal Setup}
Let the tuple $Config = (R,F,M,I)$ denote a $configuration$ of legal rules. The set $R$ denotes a set of rules of the form $pre\_con(r)\rightarrow concl(r)$. These are `naive' rules with no information pertaining to any of the other rules in $R$. $F$ is a set of positive atoms that describe facts of the legal scenario we wish to consider. $M$ is a set of the binary predicates $despite$, $subject\_to$ and $strong\_subject\_to$. $I$ is a collection of minimal inconsistent sets of positive atoms. Henceforth for a rule $r$, we may write $C_{r}$ for its conclusion $Concl(r)$.

Note that, throughout this section, given any rule $r$, $C_{r}$ is assumed to be a single positive atom. That is, there are no disjunctions or conjunctions in rule conclusions. Also any rule pre-condition ($pre\_con(r)$) is assumed to be a conjunction of positive and negated atoms. Here negation denotes `negation as failure'.  

Throughout this document, whenever we use an uppercase or lowercase letter (like $r$, $r_{1}$, $R$ etc.) to denote a rule that is an argument, in a binary predicate, we mean the unique integer rule \textsc{id} associated with that rule. The binary predicate $legally\_valid(r,c)$ intuitively means that the rule $r$ is `in force' and it has conclusion $c$. Here $r$ typically is an integer referring to the rule \textsc{id} and $c$ is the atomic conclusion of the rule. The unary predicate $is\_legal(c)$ intuitively means that the atom $c$ legally holds/has legal status. The predicates $despite$, $subject\_to$ and $strong\_subject\_to$ all cause some rules to override others. Their precise properties will be given next.


\subsection{Semantics}

A set $S$ of $is\_legal$ and $legally\_valid$ predicates is called a \textit{legal model} of $Config = (R,F,M,I)$, if and only if
\begin{description}
\item[(A1)]$\forall f \in F$ $is\_legal(f) \in S$.

\item[(A2)] $\forall r \in R$, if $legally\_valid(r,C_{r}) \in S$. then $S\models is\_legal(pre\_con(r))$ and $S\models is\_legal(C_{r})$ \footnote{By $S\models is\_legal(pre\_con(r))$ we mean that for each positive atom $b_{i}$ in the conjunction, $is\_legal(b_{i}) \in S$ and for each negation-as-failure body atom $not$ $b_{j}$ in the conjunction $is\_legal(b_{j})\notin S$ }

\item[(A3)] $\forall c$, if $is\_legal(c) \in S$, then either $c\in F$ or there exists $r \in R$ such that $legally\_valid(r,C_{r}) \in S$ and $c= C_{r}$.

\item[(A4)] $\forall r_{i}, r_{j} \in R$, if $despite(r_{i}, r_{j}) \in M$ and $S\models is\_legal(pre\_con(r_{j}))$, then $legally\_valid(r_{i},C_{r_{i}}) \notin S$

\item[(A5)] $\forall r_{i}, r_{j} \in R$, if $strong\_subject\_to(r_{i}, r_{j}) \in M$ and $legally\_valid(r_{i},C_{r_{i}}) \in S$, then $legally\_valid(r_{j},C_{r_{j}}) \notin S$


\item[(A6)] $\forall r_{i},r_{j} \in R$ if $subject\_to(r_{i},r_{j}) \in M$, and $legally\_valid(r_{i},C_{r_{i}}) \in S$ and there exists a minimal conflicting set $k \in I$ such that $C_{r_{i}} \in k$ and $C_{r_{j}}\in k$ and $is\_legal(k\setminus \{C_{r_{j}})\})\subseteq S $, then $legally\_valid(r_{j},C_{r_{j}}) \notin S$. Note than in our system, any minimal inconsistent set must contain at least 2 atoms. \footnote{For a set of atoms $A$, by $is\_legal(A)$, we mean the set $\{is\_legal(a)\mid a\in A\}$} 

\item[(A7)] $\forall r\in R$, if $S\models pre\_con(r)$, but $legally\_valid(r,C_{r})\notin S$, then it must be the case that at least one of A4 or A5 or A6 has caused the exclusion of $legally\_valid(r,C_{r})$. That is if $S\models pre\_con(r)$, then unless this would violate one of A5, A6 or A7, it must be the case that $legally\_valid(r,C_{r})\in S$.
\end{description}

\subsection{Some remarks on axioms A1--A7}
We now give some informal intuition behind some of the axioms and their intended effects.

A1 says that all facts in $F$ automatically gain legal status, that is, they legally hold. The set $F$ represents indisputable facts about the legal scenario we are considering.

A2 says that if a rule is `in force' then it must be the case that both the pre-condition and conclusion of the rule have legal status. Note that it is not enough if simply require that the conclusion has legal status as more than one rule may enforce the same conclusion or the conclusion may be a fact, so we want to know exactly which rules are in force as well as their conclusions.

A3 says that anything that has legal status must either be a fact or be a conclusion of some rule that is in force.

A4--A6 describe the semantics of the three modifiers. The intuition for the three modifiers will be discussed next. Firstly, it may help the reader to read the modifiers in certain ways. $despite(r_{i},r_{j})$ should be read as `despite $r_{i}$, $r_{j}$'. Thus $r_{i}$ here is the `subordinate rule' and $r_{j}$ is the `dominating' rule. The idea here is that once the precondition of the dominating rule $r_{j}$ is true, it invalidates the subordinate rule $r_{i}$ regardless of whether the dominating rule itself is then invalidated by some other rule. For \textit{strong subject to}, the intended reading for $strong\_subject\_to(r_{i},r_{j})$ is something like `(strong) subject to $r_{i}$, $r_{j}$'. Here $r_{i}$ can be considered the dominating rule and $r_{j}$ the subordinate. Once the dominating rule is in force, then it invalidates the subordinate rule. The intended reading for $subject\_to(r_{i},r_{j})$ is `subject to $r_{i}$, $r_{j}$'. For the subordinate rule $r_{j}$ to be invalidated, it has to be the case that the dominating rule $r_{i}$ is in force and there is a minimal inconsistent set $k$ in $I$ that contains the two atoms in the conclusions of the two rules and, $is\_legal(k\setminus \{C_{r_{j}})\})\subseteq S $. These minimal inconsistent sets along with the \textit{subject to} modifier give us a way to incorporate a classical-negation-like effect into our system. We are able to say which things contradict each other. Note that in our system, if say $\{a,b\}$ is a minimal inconsistent set, then it is possible for both $is\_legal(a)$ and $is\_legal(b)$ to be in a single legal model, if they are both facts or they are conclusions of rules that have no modifiers linking them. These minimal inconsistent sets only play a role where a $subject\_to$ modifier is involved. The reason for doing this is that this offers greater flexibility rather than treating $a$ and $b$ as pure logical negatives of each other that cannot be simultaneously true in a legal model. We will give examples later on to illustrate these modifiers.

A7 says essentially that A4--A6 represent the only ways in which a rule whose pre-condition is true may nevertheless be invalidated, and any rule whose precondition is satisfied and is not invalidated directly by some instance of A4--A6, must be in force. 

Note that there maybe legal rule configurations for which no legal models exist. See the appendix for a discussion of some 'pathological' rule configurations.

\subsection{ASP encoding}
Here is an ASP encoding scheme for a configuration $Config = (R,F,M,I)$ of legal rules.
\begin{lstlisting}[language=Prolog, numbers=left]
% For any f in F, we have:
is_legal(f). 
% All the modifiers get added as facts like for example:
despite(1,2).
% Any rule r in R is encoded using the general schema:
according_to(r,C_r):-is_legal(pre_con(r)).
% Given a minimal inconsistent set {a_1,a_2,...,a_n}, this corresponds to a set of rules:
opposes(a_1,a_2):-is_legal(a_2),is_legal(a_3),...,is_legal(a_n).
opposes(a_1,a_3):-is_legal(a_2),is_legal(a_4)...,is_legal(a_n).  % etc ...
opposes(a_n-1,a_n):-is_legal(a_1),...,is_legal(a_n-2).               
% Opposes is a symmetric relation
opposes(X,Y):-opposes(Y,X).
% Encoding for 'despite'
defeated(R,C,R1) :-
    according_to(R,C), according_to(R1,C1), despite(R,R1).
%Encoding for 'subject_to'
defeated(R,C,R1) :-
    according_to(R,C), legally_valid(R1,C1),
    opposes(C,C1), subject_to(R1,R).
% Encoding for 'strong_subject_to'
defeated(R,C,R1) :-
    according_to(R,C), legally_valid(R1,C1),
    strong_subject_to(R1,R).

not_legally_valid(R) :- defeated(R,C,R1).
legally_valid(R,C):-according_to(R,C),not not_legally_valid(R).
is_legal(C):-legally_valid(R,C).
\end{lstlisting}
\subsection{Proposition}

\begin{proposition}\label{lemma:legal_model_of_config}
For a configuration $Config=(R,F,M,I)$, let the above encoding be the program $ASP_{Config}$. Then given an answer set $A_{Config}$ of $ASP_{Config}$ let $S_{A_{Config}}$ be the set of $is\_legal$ and $legally\_valid$ predicates in $A_{Config}$. Then $S_{A_{Config}}$ is a legal model of $Config$. 
\end{proposition}


$Proof$ See Appendix of full paper. $\square$


\subsection{Example}
Let us see how the running example would work in the ASP setting. We have 3 rules encoded as below, there are no minimal inconsistent sets. There are 3 modifiers: $despite(2,3)$,                           $strong\_subject\_to(1,3)$, $strong\_subject\_to(1,2)$
\begin{lstlisting}[language=Prolog, numbers=left]
despite(2,3).
strong_subject_to(1,3).
strong_subject_to(1,2).
according_to(1,max_spd(v,d,r,90)):-is_legal(is_workday(d)),
is_legal(is_car(v)).
according_to(2,max_spd(v,d,r,130)):-is_legal(is_highway(r)),
   is_legal(is_car(v)).
according_to(3,max_spd(v,d,r,320)):-is_legal(is_highway(r)),
   is_legal(is_sports_car(v)).

% Encoding for 'despite'
defeated(R,C,R1) :-
    according_to(R,C), according_to(R1,C1), despite(R,R1).
%Encoding for 'subject_to'
defeated(R,C,R1) :-
    according_to(R,C), legally_valid(R1,C1),
    opposes(C,C1), subject_to(R1,R).
% Encoding for 'strong_subject_to'
defeated(R,C,R1) :-
    according_to(R,C), legally_valid(R1,C1),
    strong_subject_to(R1,R).

not_legally_valid(R) :- defeated(R,C,R1).
legally_valid(R,C):-according_to(R,C),not not_legally_valid(R).
is_legal(C):-legally_valid(R,C).
\end{lstlisting}

When the initial set of facts $F$ is the set
\begin{lstlisting}[language=Prolog, numbers=left]
is_legal(is_workday(d)).
is_legal(is_car(v)).
is_legal(is_highway(r)).
is_legal(is_sports_car(v)).
\end{lstlisting}
we get exactly one legal max speed given by 
\begin{lstlisting}[language=Prolog, numbers=left]
is_legal(max_spd(v,d,r,90)).
\end{lstlisting}
One can check this by adding the rule 
\begin{lstlisting}[language=Prolog, numbers=left]
legal_max_spd(X):- is_legal(max_spd(v,d,r,X)). 
\end{lstlisting}
and running the s(CASP) query $?- legal\_max\_spd(X).$, which returns the binding $X = 90$. This is the unique legal maximum speed which can be seen via use of the rule 
\begin{lstlisting}[language=Prolog, numbers=left]
legal_max_spd(X):- X > 90, is_legal(max_spd(v,d,r,X)). 
\end{lstlisting}
Now running the query as above we see that there is no solution. 
When removing
$is\_legal(is\_workday(d))$ from $F$, we get exactly one legal max speed of $320$, and when  $is\_legal(is\_sports\_car(v))$, and $is\_legal(is\_workday(d))$ are both removed from $F$ we get exactly one exactly legal max speed of $130$.

% We note here that if all 4 facts as in the first initial fact set are passed
% to the SMT solver following the approach presented in
% \secref{sec:defeasible_classical} then the SMT solver will also return exactly
% one max speed of $90$. If instead we choose to modify the initial fact set by
% negating $is\_work\_day(d)$, then the SMT solver will return exactly one max
% speed of $320$ and if modify the initial fact set by negating both
% $is\_work\_day(d)$ and $is\_sports\_car(v)$, we get exactly one max speed of
% $130$.







%%% Local Variables:
%%% mode: latex
%%% TeX-master: "main"
%%% End:


\section{Conclusions}\label{sec:conclusions}

This paper has discussed different approaches for representing defeasibility
as used in law texts, by annotating rules with modifiers that explicate their
relation to other rules. We have notably presented two encodings in classical
logic (\secrefs{sec:restr_precond} and \ref{sec:restr_deriv}) and explained
how they are related (\secref{sec:comparison}). Quite a different approach,
based on Answer Set Programming, is presented in
\secref{sec:defeasible_asp}. \emph{Added by Avishkar} All the encodings are motivated by the need to explore the implementation of various forms of defeasibility in logics for which well developed and powerful solvers already exist.  

An experimental implementation of the L4 ecosystem is under
way\footnote{\url{https://github.com/smucclaw/baby-l4}}, but it has not yet
reached a stable, user-friendly status. It is implemented in Haskell and features
an IDE based on VS Code and natural language processing via Grammatical
Framework\cite{ranta_grammatical_2004}. Currently, only the coding of
\secrefs{sec:restr_precond} has been implemented---the coding of
\secref{sec:restr_deriv} would require a much more profound transformation of
rules and assertions. Likewise, the ASP coding of \secref{sec:defeasible_asp}
has only been done manually. The interaction with SMT solvers is done through
an SMT-LIB \cite{BarFT_SMTLIB} interface, thus opening the possibility to
interact with a wide range of solvers. As our rules typically contain
quantification, reasoning with quantifiers is crucial, and the best support
currently seems to be provided by Z3 \cite{demoura_bjorner_z3_2008}.

We are still at the beginning of the journey. A theoretical comparison of the
classical and ASP approaches presented here still has to be carried out, and
it has to be propped up by an empirical evaluation. For this purpose, we are
currently in the process of coding some real-life law texts in L4. We are
fully aware of shortcomings of the current L4, which will strongly evolve in
the next months, to include reasoning about deontics and about temporal
relations. Integrating these aspects will not be easy, and this is one reason
for not committing prematurely to one particular logical framework.


%%% Local Variables:
%%% mode: latex
%%% TeX-master: "main"
%%% End:


\begin{acknowledgments}
The contributions of the members of the L4 team to this common effort are
thankfully acknowledged, in particular of Jason Morris who contributed his
experience with Answer Set Programming; of Jacob Tan and Ruslan Khafizov who
have participated in discussions about its contents and commented on the
paper; and of Liyana Muthalib who has proof-read a previous version.

This research is supported by the National Research Foundation (NRF),
Singapore, under its Industry Alignment Fund –- Pre-Positioning Programme, as
the Research Programme in Computational Law. Any opinions, findings and
conclusions or recommendations expressed in this material are those of the
authors and do not reflect the views of National Research Foundation,
Singapore.
\end{acknowledgments}


%----------------------------------------------------------------------
\bibliography{gde22}


\end{document}


%%% Local Variables:
%%% mode: latex
%%% TeX-master: t
%%% End:
