\documentclass[]{ceurart}
%%
%% One can fix some overfulls
\sloppy



% Language setting
% Replace `english' with e.g. `spanish' to change the document language
% \usepackage[english]{babel}

% Set page size and margins
% Replace `letterpaper' with `a4paper' for UK/EU standard size
%\usepackage[letterpaper,top=2cm,bottom=2cm,left=3cm,right=3cm,marginparwidth=1.75cm]{geometry}

% Useful packages
\usepackage{amsmath}
\usepackage{tikz}
\usetikzlibrary{trees}
\usetikzlibrary{positioning} 
\usetikzlibrary{arrows}
\usetikzlibrary{decorations.pathmorphing}
\usetikzlibrary{shapes.multipart}
\usetikzlibrary{shapes.geometric}
\usetikzlibrary{calc}
\usetikzlibrary{positioning} 
\usetikzlibrary{fit}
\usetikzlibrary{backgrounds}
\usetikzlibrary{automata}
\usepgflibrary{shapes.geometric}
\usetikzlibrary{shapes.geometric}
\usepackage{mathpartir}
\usepackage{listings}
\usepackage{proof}
\newtheorem{theorem}{Theorem}
\newtheorem{lemma}{Lemma}

\usepackage{graphicx}
%\usepackage[colorlinks=true, allcolors=blue]{hyperref}


\lstdefinelanguage{L4}
{morekeywords={
      assert 
    , class  
    , decl   
    , defn   
    , extends
    , lexicon
    , fact   
    , rule   
    , derivable
    , let   
    , in    
    , not   
    , forall
    , exists
    , if   
    , then 
    , else 
    , for  
    , true 
    , false
},    
sensitive=false,
morecomment=[l]{\#},
morestring=[b]",
}

\lstset{frame=tb,
  language=L4,
  aboveskip=3mm,
  belowskip=3mm,
  showstringspaces=false,
  columns=flexible,
  basicstyle={\footnotesize\ttfamily},
  numbers=none,
  numberstyle=\tiny\color{gray},
  keywordstyle=\color{blue},
  commentstyle=\color{green},
  stringstyle=\color{mauve},
  breaklines=true,
  breakatwhitespace=true,
  tabsize=2
}

%%% Local Variables: 
%%% mode: latex
%%% TeX-master: "main"
%%% End: 

% Theorems and definitions

% \newtheorem{definition}{Definition}
% \newtheorem{theorem}{Theorem}
% \newtheorem{lemma}{Lemma}
% \newtheorem{proposition}{Proposition}


% Definition of colors
\newcommand{\blue}[1]{{\color{blue}#1}}
\newcommand{\green}[1]{{\color{green}#1}}
\newcommand{\red}[1]{{\color{red}#1}}
\newcommand{\gray}[1]{{\color{gray}#1}}

% From MSCS file
\newcommand{\eg}{\textit{e.g.\ }}
\newcommand{\etal}{\textit{et al.\ }}
\newcommand{\etc}{\textit{etc}}
\newcommand{\ie}{\textit{i.e. }}
\newcommand{\viz}{\textit{viz.\ }}
\newcommand{\wrt}{\textit{w.r.t.\ }}
\newcommand{\lex}{\langle}
\newcommand{\rex}{\rangle}

% Own abbreviations
\newcommand{\secref}[1]{Section~\ref{#1}}
\newcommand{\secrefs}[1]{Sections~\ref{#1}}
\newcommand{\figref}[1]{Figure~\ref{#1}}
\newcommand{\figrefs}[1]{Figures~\ref{#1}}
\newcommand{\pgref}[1]{page~\pageref{#1}}
\newcommand{\theoremref}[1]{Theorem~\ref{#1}}
\newcommand{\theoremrefs}[1]{Theorems~\ref{#1}}
\newcommand{\lemmaref}[1]{Lemma~\ref{#1}}
\newcommand{\exampleref}[1]{Example~\ref{#1}}
\newcommand{\defref}[1]{Definition~\ref{#1}}

\newcommand{\figline}{\rule{\textwidth}{0.5pt}}


% Logique

\newcommand{\IMPL}[0]{\longrightarrow}
\newcommand{\IMPLL}[0]{\Longrightarrow} % another implication, to make
                                % a difference with reduction relations
\newcommand{\AND}[0]{\land}
\newcommand{\OR}[0]{\lor}
\newcommand{\NOT}[0]{\lnot}
\newcommand{\FALSE}[0]{\perp}
\newcommand{\TRUE}[0]{\top}
\newcommand{\IFF}[0]{\leftrightarrow}
\newcommand{\BIGAND}[1]{\bigwedge_{#1}}
\newcommand{\BIGOR}[1]{\bigvee_{#1}}
\newcommand{\BIGANDC}[2]{\bigwedge_{#1|#2}} % bigand with constraint
\newcommand{\BIGORC}[2]{\bigvee_{#1|#2}}    % bigor with constraint

\newcommand{\exgeq}[1]{\exists^{{\geq #1}}}
\newcommand{\exeq}[1]{\exists^{{= #1}}}
\newcommand{\exle}[1]{\exists^{{< #1}}}

% Remark macros for the authors

\newcommand{\remms}[2][]{\todo[color=green!40,#1]{MS: #2}}
\newcommand{\remre}[2][]{\todo[color=blue!40,#1]{RE: #2}}
\newcommand{\remjhb}[2][]{\todo[color=blue!20,#1]{JHB: #2}}


% Other

\newcommand{\smalltalcq}[0]{{\small small}-t{$\cal ALCQ$}}
\newcommand{\smalltalcqe}[0]{{\small small}-t{$\cal ALCQ$e}}
\newcommand{\trule}[0]{\xhookrightarrow}
\newcommand{\tableaurule}[1]{{\xhookrightarrow[]{#1}}}
\newcommand{\nodes}[1]{{\cal N}({#1})}
\newcommand{\trans}[1]{{\cal T}({#1})}
\newcommand{\transm}[1]{{\cal T'}({#1})}
\newcommand{\rconts}[1]{\llparenthesis #1 \rrparenthesis} %record contents
\newcommand{\rupd}[2]{{#1}\llparenthesis #2 \rrparenthesis} %record update

\newcommand{\eform}[0]{\mathit{eform}}
\newcommand{\form}[0]{\mathit{form}}
\newcommand{\free}[0]{\mathit{free}}
\newcommand{\exclprop}[0]{\stackrel{\times}{\longrightarrow}}

%%% Local Variables: 
%%% mode: latex
%%% TeX-master: "main"
%%% End: 

\newtheorem{example}{Example}
\newtheorem{proposition}{Proposition}

\usepackage{amsmath}
\usepackage{graphicx}
\usepackage{multirow}


%%
%% Minted listings support 
%% Need pygment <http://pygments.org/> <http://pypi.python.org/pypi/Pygments>
\usepackage{listings}
%% auto break lines
\lstset{breaklines=true}

%%
%% end of the preamble, start of the body of the document source.

\begin{document}


%% Rights management information.
%% CC-BY is default license.
\copyrightyear{2022}
\copyrightclause{Copyright for this paper by its authors.
  Use permitted under Creative Commons License Attribution 4.0
  International (CC BY 4.0).}

%% This command is for the conference information
\conference{2nd Workshop on Goal-directed Execution of Answer Set
  Programs (GDE'22), August 1, 2022}

\title{Automating Defeasible Reasoning in Law with Answer Set Programming}

\author{Lim How Khang}[%
style=chinese,
orcid=0000-0002-9333-1364
%email=name@org.com,
%url=http://www. ,
]
%
\author{Avishkar Mahajan}[%
orcid=0000-0002-9925-1533
%email=name@org.com,
]
%
\author{Martin Strecker}[%
orcid=0000-0001-9953-9871
%email=name@org.com,
]
%
\author{Meng Weng Wong}[%
orcid=0000-0003-0419-9443
%email=name@org.com,
]
\address{Singapore Management University}

\begin{abstract}
The paper studies defeasible reasoning in rule-based systems, in particular
about legal norms and contracts. We identify rule modifiers that specify how
rules interact and how they can be overridden. We then define rule
transformations that eliminate these modifiers, leading in the end to a
translation of rules to formulas. For reasoning with and about rules, we
contrast two approaches, one in a classical logic with SMT solvers, which is
only briefly sketched, and one using non-monotonic logic with Answer Set
Programming solvers, described in more detail.
\end{abstract}

\begin{keywords}
  Knowledge representation and reasoning
  \sep
  Argumentation and law
  \sep
  Computational Law
  \sep
  Defeasible reasoning
\end{keywords}

\maketitle




%----------------------------------------------------------------------

\section{Introduction}\label{sec:introduction}

The goal of this paper is to show how a bottom up ASP reasoner like Clingo can be used for Abductive reasoning over First Order Horn clauses. As mentioned in the abstract previous work in abductive reasoning has mostly focused on implementing abduction in a top-down manner with Prolog as the underlying engine. CIFF \cite{mancarella09:_ciff} is a prominent example of this. More recently sCASP \cite{arias19:_const_answer_set_progr_groun_applic,arias_phd_2019} has been developed as a goal directed ASP implementation that can be used for abduction but this too uses a top down method for query evaluation. However there may be use cases where one wants to know all the resulting consequences of an abductive solution to a query with respect to a rule-set. Also, as mentioned in the abstract, top-down methods can sometimes result in solvers that are not truly declarative. Therefore an abductive reasoner that uses a solver like Clingo \cite{gebser12:_answer_set_solvin_pract} can
complement the abilities of goal directed reasoners like sCASP, CIFF \etc.

This paper shows how, given an input ASP rule set, one can write a new ASP program based on that rule set which will yield abductive solutions to queries, with the input ASP rule set as the background theory. The user does not have to explicitly specify the space of abducibles. This translation from the input ASP rule set to the derived ASP program is a purely mechanical one. The key idea is to encode backward chaining over the input rules through the use of meta predicates which incorporate a notion of 'reversing' the input rules to recursively generate pre-conditions from post conditions thereby generating a maximal space of abducibles. Then having generated this maximal space of abducibles, this 'feeds into' another part of the program where we have a representation of the input rules in the normal 'forward' direction. Entailment of the specified query is then checked via an integrity constraint and a minimal set of abduced facts is returned.

The main technical challenges are dealing with situations where input rules have existential variables in pre-conditions or when the query itself has existential variables. The other challenge is to control the depth of the abducibles generation process. The work that seems to come closest to ours is \cite{schueller16:_model_variat_first_order_horn}. It too uses some similar meta predicates to encode backward chaining, and a forward representation of the rules to check for query entailment via integrity constraints.

However there are several novel features in our work.  Firstly, depth  control for abducible generation is done in a purely declarative way as part of the encoding itself without needing to call external functions or other pieces of software. Furthermore, adding facts to the program automatically gives an implicit form of term substitution where Skolem terms or
other 'place-holder' terms occurring in abducibles are replaced away so that
the resulting proof is simplified, without any need for an explicit representation of equality between terms. Past work on this topic such as \cite{schueller16:_model_variat_first_order_horn} models equality between terms via an explicit equality predicate which may become unwieldy. Another approach to dealing with existential variables encountered during the abductive proof search is to simply ground all the rules over the entire domain of constants. However, this can often lead to too many choices for what an existential variable may be substituted for which may result in unexpected/unintuitive solutions. Our method avoids both of these techniques. We present three main sets of abductive proof generation encodings. One of the encodings only supports partial term substitution whereas the other two support full term substitution. Lastly, we also present an encoding which generates a set of directed edges representing a justification 
graph  for the generated proof, where the graph can be of any desired depth.

The rest of the paper is organised as
follows. First we give a brief introduction to Answer Set Programming and Abductive reasoning then, Section~\ref{sec:abductive_proof} defines the problem being tackled more formally. Section~\ref{sec:derived_asp} presents the encodings that facilitate the abductive proof generation and directed edge generation. The sections that follow discuss some formal results regarding completeness, finiteness of abductive proof generation. We also discuss a formal result regarding term substitution. Finally Section~\ref{sec:conclusion} discusses
future work and concludes.

This is an extended version of a paper presented at PPDP~2022 \cite{ppdp_version}.

\subsection{Answer Set Programming}
Answer Set Programming (ASP) is a declarative language from the logic programming family. It is widely used and studied by knowledge representation and reasoning and symbolic AI researchers for its ability to model common sense reasoning, model combinatorial search problems etc. It incorporates the $\textit{negation-as-failure}$ operator as interpreted under the $\textit{stable model semantics}$. Clingo is a well established implementation of ASP, incorporating additional features such as $\textit{choice rules}$ and optimization statements. We shall only briefly touch upon various aspects of ASP and Clingo here. The reader may consult \cite{gebser12:_answer_set_solvin_pract} for a more thorough description. Each rule in an ASP program consists of a set of body atoms. Some of these body atoms maybe negated via the negation as failure operator $not$. Rules with no pre-conditions are called facts. Given a set of rules $R$ and a set of facts $F$, the Clingo solver computes all stable models of the ASP program $F\cup R$. For example given the fact $r(alpha)$ and the rules:
\begin{lstlisting}[frame=none]
p(X):-r(X),not q(X).
q(X):-r(X), not p(X).
\end{lstlisting}
The solver will show us 2 models or answer sets given by\\ $\{r(alpha),p(alpha)\}$ and $\{r(alpha),q(alpha)\}$. Note that as opposed to Prolog, Clingo is a bottom up solver meaning that it computes complete stable models (also known as answer sets) given any ASP program. An integrity constraint is formally speaking a rule whose post-condition is the boolean $false$. In ASP, integrity constraints are written as rules with no post-conditions and are used to eliminate some computed answer sets. For example given in the following ASP program
\begin{lstlisting}[frame=none]
r(alpha).
p(X):-r(X),not q(X).
q(X):-r(X), not p(X).
:-q(X).
\end{lstlisting}
any answer set where some instantiation of $q$ is true is eliminated. Hence we get just one answer set. $\{r(alpha),p(alpha)\}$.\\ We will now give a quick introduction to two features of Clingo that we will use throughout this paper. Namely $\textit{choice rules}$ and $\textit{weak constraints}$. Weak constraints are also often known as optimization statements. Intuitively a choice rule is a rule where if the pre-conditions are satisfied then the post-condition may or may not be made true. The post-condition of a choice rule is enclosed in curly brackets. So given the following ASP program:\begin{lstlisting}[frame=none]
r(alpha).
{q(X)}:-r(X).
\end{lstlisting}, where the rule is a choice rule the solver will give us 2 models namely $\{r(alpha)\}$, $\{r(alpha),q(alpha)\}$. If we modify the program by adding an integrity constraint like so:
\begin{lstlisting}[frame=none]
r(alpha).
{q(X)}:-r(X).
:-q(X).
\end{lstlisting}
then we get just one model $\{r(alpha)\}$.\\
Weak constraints are used in Clingo to order answer sets by preference according to the atoms that appear in them. Without going into too much detail let us just explain the meaning of one kind of weak constraint which is the only kind that we will use in the paper namely:
\begin{lstlisting}[frame=none]
:~a(X). [1@1,X]
\end{lstlisting}
Adding this to an ASP program, orders the answer sets of the program according to the number of distinct instantiations of the predicate $a$ in the answer set. The answer set with the least number of instantiations of $a$ is called the most $optimal$ answer set. 
\subsection{Abductive Reasoning}
Briefly, abduction is a reasoning process where given a background theory $T$, we wish to find a set of facts $F$ such that $F\cup T$ is consistent and $F\cup T$ entails some goal $g$ for some given entailment relation. Usually we also want $F$ to be minimal in some well defined sense. Traditional Abductive Logic Programming has a long history, but we have our own definitions of what it means to formulate and solve an abductive reasoning problem and we will make all the relevant concepts/notions precise in the sections that follow.  


%%% Local Variables:
%%% mode: latex
%%% TeX-master: "main"
%%% End:


% file intentionally commented out, text now integrated into l4_language_app.tex
% \section{An Overview of the L4 Language}\label{sec:l4_language}

This section gives an account of the L4 language as it is currently defined --
as an experimental language, L4 will evolve over the next months. In our
discussion, we will ignore some features such as a natural language interface
\cite{} which is not relevant for the topic of this paper but may be [desirable/explored/necessary in the future for some reason].

As a language intended for representing legal texts and reasoning about them,
an L4 module is essentially composed of four sections:
\begin{itemize}
\item a terminology in the form of \emph{class definitions};
\item \emph{declarations} of functions and predicates;
\item \emph{rules} representing the core of a law text, specifying what is
  considered as legal behaviour;
\item \emph{assertions} for stating and proving properties about the rules.
\end{itemize}

In the following, these elements will be presented in more detail and
illustrated with an example, a (fictitious) regulation of speed
limits for different types of vehicles.


% ----------------------------------------------------------------------
\subsection{Terminology and Class Definitions}\label{sec:classdefs}

The definition in \figref{fig:classdefs} introduces classes for vehicles, days
and roads. 

\begin{figure}[h!]
%\begin{mdframed}
\begin{lstlisting}
class Vehicle {
   weight: Integer
}
class Car extends Vehicle {
   doors: Integer
}
class Truck extends Vehicle
class SportsCar extends Car

class Day
class Workday extends Day
class Holiday extends Day

class Road
class Highway extends Road
\end{lstlisting}
%\end{mdframed}
  \caption{Class definitions of speedlimit example}\label{fig:classdefs}
\end{figure}

Classes are arranged in a tree-shaped hierarchy, having a class named
\texttt{Class} as its top element. Classes that are not explicitly derived
from another class via \texttt{extends} are implicitly derived from
\texttt{Class}. A class $S$ derived from a class $C$ by \texttt{extends} will
be called a subclass of $C$, and the immediate subclasses of \texttt{Class}
will be called \emph{sorts} in the following. Intuitively, classes are meant
to be sets of entitities, with subclasses being interpreted as
subsets. Different subclasses of a class are not meant to be disjoint.

Class definitions can come with attributes, in braces. These attributes can be
of simple type, as in the given example, or of higher type (the notion of type
will be explained in \secref {sec:fundecls}). In a declarative reading,
attributes can be understood as a shorthand for function declarations that
have the class they are defined in as additional domain. Thus, the attribute
\texttt{weight} corresponds to a top-level declaration \texttt{weight: Vehicle
  -> Integer}. In a more operational reading, L4 classes can be understood as
prototypes of classes in object-oriented programming languages, and an
alternative field selection syntax can be used: For \texttt{v: Vehicle}, the
expression \texttt{v.weight} is equivalent to \texttt{weight(v)}, at least
logically, even though the operational interpretations may differ.


% ----------------------------------------------------------------------
\subsection{Types and Function Declarations}\label{sec:fundecls}

L4 is an \emph{explicitly} and \emph{strongly typed} language: all entities
such as functions, predicates and variables have to be declared before being
used. One purpose of this measure is to ensure that the executable sublanguage
of L4, based on the simply-typed lambda calculus with subtyping, enjoys a type
soundness property: evaluation of a function cannot produce a dynamic type
error.

\figref{fig:fundecls} shows two function declarations. Functions with
\texttt{Boolean} result type will sometimes be called \emph{predicates} in the
following, even though there is no syntactic difference. All the declared
classes are considered as elementary types, as well as \texttt{Integer},
\texttt{Float}, \texttt{String} and \texttt{Boolean} (which are internally also treated as
classes). If $T_1, T_2, \dots T_n$ are types, then so are function types
\texttt{$T_1$ -> $T_2$} and tuple types \texttt{($T_1$, $\dots$ ,$T_n$)}. The
type system and the expression language, to be presented later, are
higher-order, but extraction to some solvers (\secref{}) will be limited to
(restricted) first-order theories.

\begin{figure}[h]
%\begin{mdframed}
\begin{lstlisting}
decl isCar : Vehicle -> Boolean
decl maxSp : Vehicle -> Day -> Road -> Integer -> Boolean
\end{lstlisting}
%\end{mdframed}
  \caption{Declarations of speedlimit example}\label{fig:fundecls}
\end{figure}

The nexus between the terminological and the logical level is established with
the aid of \emph{characteristic predicates}. Each class $C$ which is a
subclass of sort $S$ gives rise to a declaration \texttt{is$C$: $S$ ->
  Boolean}. An example is the declaration of \texttt{isCar} in
\figref{fig:fundecls}. In the L4 system, this declaration, as well as the
corresponding class inclusion axiom (\secref{sec:rules}), are generated
automatically.\remms{Promise}

Two classes derived from the same base class (thus: \texttt{$C_1$ extends $B$}
and \texttt{$C_2$ extends $B$}) are not necessarily disjoint. 

From the subclass relation, a \emph{subtype} relation $\preceq$ can be defined
inductively as follows: if \texttt{$C$ extends $B$}, then $C \preceq B$, and
for types $T_1, \dots, T_n, T_1', \dots, T_n'$,
if $T_1 \preceq T_1', \dots, T_n \preceq T_n'$, 
then \texttt{$T_1'$ -> $T_2 \; \preceq \; T_1$ -> $T_2'$} 
and \texttt{($T_1$, $\dots$, $T_n$) $\preceq$ ($T_1'$, $\dots$, $T_n'$)}.

Without going into details of the type system, let us remark that it has been
designed to be compatible with subtyping: if an element of a type is
acceptable in a given context, then so is an element of a subtype. In
particular,
\begin{itemize}
\item for field selection, if $C'$ is a class having field $f$ of type $T$,
  and $C \preceq C'$, and $c : C$, then field selection is well-typed with $c.f : T$.
\item for function application, if $f: A' \mbox{\texttt{->}} B$ and $a:A$ and
  $A \preceq A'$, then function application is well-typed with $f\; a : B$.
\end{itemize}


% ----------------------------------------------------------------------
\subsection{Rules}\label{sec:rules}

Before discussing rules, a few remarks about \emph{expressions} which are
their main constituents: L4 supports a simple functional language featuring
typical arithmetic, Boolean and comparison operators, an \texttt{if .. then
  .. else} expression, function application, anonymous functions (\ie, lambda
abstraction) written in the form \texttt{$\backslash$x : T -> e}, class
instances and field access (as mentioned before). A \emph{formula} is just a
Boolean expression, and, consequently, so are quantified formulas
\texttt{forall x:T. form} and \texttt{exists x:T. form}.

In its most complete form, a \emph{rule} is composed by a list of variable
declarations introduced by the keyword \texttt{for}, a precondition introduced
by \texttt{if} and a postcondition introduced by
\texttt{then}. \figref{fig:rules} gives an example of rules of our speed limit
scenario, stating, respectively, that the maximal speed of cars is 90 km/h on a
workday,\remms{scenario changed wrt repo}
and that they may drive at 130 km/h if the road is a highway.  Note that in
general, both pre- and postconditions are Boolean formulas that can be
arbitrarily complex, thus are not limited to conjunctions of literals in the
preconditions or atomic formulas in the postconditions.

\begin{figure}[h!]
%\begin{mdframed}

  \begin{lstlisting}
rule <maxSpCarWorkday> 
   for v: Vehicle, d: Day, r: Road
   if isCar v && isWorkday d
   then maxSp v d r 90

rule <maxSpCarHighway>
   for v: Vehicle, d: Day, r: Road
   if isCar v && isHighway r
   then maxSp v d r 130
\end{lstlisting}

%\end{mdframed}
  \caption{Rules of speedlimit example}\label{fig:rules}
\end{figure}

Rules whose precondition is \texttt{true} can be written as

\begin{lstlisting}[frame=none,mathescape=true]
fact <name> for $v_1$: $T_1$ $\dots$ $v_n$: $T_n$ P $v_1 \dots v_n$
\end{lstlisting}

Rules may not contain free variables, so all variables occuring in the body of
the rule have to be declared in the \texttt{for} clause. In the absence of
variables to be declared, the \texttt{for} clause can be omitted.

Intuitively, a rule

\begin{lstlisting}[frame=none,mathescape=true]
  rule <r> for $\overrightarrow{v}$: $\overrightarrow{T}$ if Pre $\overrightarrow{v}$ then Post $\overrightarrow{v}$
\end{lstlisting}
corresponds to a universally quantified formula
$\forall \overrightarrow{v} : \overrightarrow{T}.\; Pre \overrightarrow{v}
\IMPL Post \overrightarrow{v}$ that could directly be written as a fact,
and it may seem that a separate rule syntax is redundant. This is not so,
because the specific structure of rules makes them amenable to transformations
that are useful for defeasible reasoning, as will be seen in \secref{sec:defeasible}.

Apart from user-defined rules and rules obtained by transformation, there are
system generated rules: For each subclass relation \texttt{$C$ extends $B$}, a
class inclusion axiom of the form \texttt{for x: $S$ if is$C$ x then is$B$ x}
is generated, where \texttt{is$C$} and \texttt{is$B$} are the characteristic
predicates and $S$ is the common supersort of $C$ and $B$.\remms{Promise}


% ----------------------------------------------------------------------
\subsection{Assertions}\label{sec:assertions}


Assertions are statements that the L4 system is meant to verify or to
reject -- differently said, they are proof obligations. These assertions are verified
relative to a rule set comprising some or all of the rules and facts stated
before. 

The statement in \figref{fig:assertions} claims that the predicate
\texttt{maxSp} behaves like a function, \ie{} given the same car, day and
road, the speed will be the same. Instead of a universal quantification, we
here use variables \texttt{inst...} that have been declared globally, because they
produce more readable (counter-)models. 

\begin{figure}[h]
%\begin{mdframed}

\begin{lstlisting}
assert <maxSpFunctional> {SMT: {valid}}
   maxSp instCar instDay instRoad instSpeed1 &&
   maxSp instCar instDay instRoad instSpeed2
   --> instSpeed1 == instSpeed2
\end{lstlisting}

%\end{mdframed}
  \caption{Assertions of speedlimit example}\label{fig:assertions}
\end{figure}

The active rule set used for verification can be configured, by adding rules
to or deleting rules from a default set. Assume the active rule set consists
of $n$ rules whose logical representation is $R_1 \dots R_n$, and assume the
formula of the assertion is $A$. The proof obligation can then be checked for
\begin{itemize}
\item  \emph{satisfiability}: in this case, $R_1 \AND \dots \AND R_n \AND A$
  is checked for satisfiability.
\item \emph{validity}: in this case, $R_1 \AND \dots \AND R_n \IMPL A$ is
  checked for validity.
\end{itemize}
In either case, if the proof fails, a model resp.{} countermodel is produced.
In the given example, the SMT solver checks the validity of the formula and
indeed returns a countermodel that leads to contradictory prescriptions of the
maximal speed: if the vehicle is a car, the day a workday and the road a
highway, the maximal speed can be 90 or 130, depending on the rule applied.
\remms{Put output of model checker here?}

The assertion \texttt{maxSpFunctional} can be considered as an essential
consistency requirement and a rule system violating it as inconsistent. One
remedial action is to declare one of the rules as default and the other rule
as overrriding it. In \secref{sec:defeasible}, we will discuss different solutions
implemented in L4.

After this repair action, \texttt{maxSpFunctional} will be provable (under
additional natural conditions described in \secref{sec:rule_inversion}). We can now
continue to probe other consistency requirements, such as exhaustiveness
stating that a maximal speed is defined for every combination of vehicle:

\begin{lstlisting}
assert <maxSpExhaustive>
   exists sp: Integer. maxSp instVeh instDay instRoad sp
\end{lstlisting}

The intended usage scenario of L4 is that by an interplay of proving
assertions and repairing broken rules, one arrives at a rule set satisfying
general principles of coherence, completeness and other, more elusive
properties such as fairness (at most temporary exclusion from essential
resources or rights).\remms{maybe put this into the intro}

%%% Local Variables:
%%% mode: latex
%%% TeX-master: "main"
%%% End:


% ----------------------------------------------------------------------
\section{Reasoning with and about Rules}\label{sec:resasoning_with_rules}
% ----------------------------------------------------------------------

%%% Local Variables:
%%% mode: latex
%%% TeX-master: "main"
%%% End:


%\section{Defeasible Reasoning}\label{sec:defeasible}

% ----------------------------------------------------------------------
\subsection{Facets of Defeasible Reasoning}\label{sec:facets}

\begin{tcolorbox}[title=To be done]
Discussion on existing notions of defeasibility in particular in a legal context
\begin{itemize}
\item \cite{hart_concept_of_law_1997}
\item \cite{alchourron_makinson_hierarchies_of_regulations_1981}
\item \cite{hage_law_and_defeasibility_2003}
\item More recent: \cite{governatori21:_unrav_legal_refer_defeas_deont_logic}
\item Using SMT solvers for CASP:
  \cite{shen_lierler_smt_answer_set_kr_2018}. Some of the techniques are
  similar (use of Clark completion), but our purpose is not to simulate answer
  set programming in SMT solvers, but to represent legal reasoning. In this
  context, maybe also look at \cite{fages_consistency_clark_completion_1994}.
\end{itemize}
\end{tcolorbox}

Given a plethora of different notions of ``defeasibility'', we had to make a
choice as to which notions to support, and which semantics to give to them. We
will here concentrate on two concepts, which we call \emph{rule modifiers},
that limit the applicability of rules and make them ``defeasible''. They will
we be presented in \secref{sec:intro_rule_modifiers}, and we will see how to give
them a semantics in a classical first-order logic, but also in a non-monotonic
logic in a system based on Answer Set Programming
(\secref{sec:answer_set_programming}). An orthogonal question is that of
arriving at conclusion in absence of complete information, which, via the
mechanism of negation as \remms{as/by??} failure, often prones the use of
non-monotonic logics. In \secref{sec:rule_inversion}, we will see how similar
effects can be achieved by means of \emph{rule inversion}. An Answer Set
Programming approach to defeasible reasoning, with an emphasis on rule
modifiers, will be presented in \secref{sec:answer_set_programming}. We finish
with a comparison of the two strands of reasoning in \secref{sec:comparison}.


% ----------------------------------------------------------------------
\subsection{Introducing Rule Modifiers}\label{sec:intro_rule_modifiers}

We will concentrate on two rule modifiers that restrict the applicability of
rules and that frequently occur in law texts: \emph{subject to} and
\emph{despite}. To illustrate their use, we consider an excerpt of Singapore's
Professional Conduct Rules \S~34 \cite{professional_conduct_rules} (also see
\cite{morris21:_const_answer_set_progr_tool} for a more detailed treatment of
this case study):

\begin{description}
\item[(1)] A legal practitioner must not accept any executive appointment
  associated with any of the following businesses: 
  \begin{description}
  \item[(a)] any business which detracts from, is incompatible with, or
    derogates from the dignity of, the legal profession;
  \item[(b)] any business which materially interferes with the legal
    practitioner’s primary occupation of practising as a lawyer; (\dots)
  \end{description}
\item[(5)] Despite paragraph (1)(b), but subject to paragraph (1)(a) and (c)
  to (f), a locum solicitor may accept an executive appointment in a business
  entity which does not provide any legal services or law-related services, if
  all of the conditions set out in the Second Schedule are satisfied.
\end{description}

The two main notions developed in the Conduct Rules are which executive appointments a legal
practictioner \emph{may} or \emph{must not} accept under which
circumstances. As there is currently no direct support for deontic logics in
L4, these notions are defined as two predicates \texttt{MayAccept} and
\texttt{MustNotAccept}, with the intended meaning that these two notions are
contradictory, and this is indeed what will be provable after a complete
formalization.

Let us here concentrate on the modifiers \emph{despite} and \emph{subject
  to}. A synonym of ``despite'' that is often used in legal texts is
``notwithstanding'',  and a synonym of
``subject to'' is ``except as provided in'', see \cite{adams_contract_drafting_2004}.

The reading of rule (5) is the following:
\begin{itemize}
\item ``subject to paragraph (1)(a) and (c) to (f)'' means: rule (5) applies
  as far as (1)(a) and (c) to (f) is not established. Differently said, rules
  (1)(a) and (c) to (f) undercut or defeat rule (5).

  One way of explicitating the ``subject to'' clause would be to rewrite (5)
  to: ``Despite paragraph (1)(b), provided the business does not detract from,
  is incompatible with, or derogate from the dignity of, the legal profession;
  and provided that not [clauses (1)(c) to (f)]; then a locum
  sollicitor\footnote{in our class-based terminology, a subclass of legal
    practitioner} may accept an executive appointment.''

\item ``despite paragraph (1)(b)'' expresses that rule (5) overrides rule
  (1)(b). In a similar spirit as the ``subject to'' clause, this can be made
  explicit by introducing a proviso, however not locally in  rule (5), but
  remotely in rule (1)(b).

  One way of explicitating the ``despite'' clause of rule (5) would be to
  rewrite (1)(b) to: ``Provided (5) is not applicable, a legal practitioner
  must not accept any executive appointment associated with any business which
  materially interferes with the legal practitioner’s primary occupation of
  practising as a lawyer.''
\end{itemize}

The astute reader will have remarked that the treatment in both cases is
slightly different, and this is not related to the particular semantics of
\emph{subject to} and \emph{despite}: we can state defeasibility
\begin{itemize}
\item either in the form of (negated) preconditions of rules: ``rule $r_1$ is
  applicable if the preconditions of $r_2$ do not hold'';
\item or in the form of (negated) derivability of the postcondition of rules: ``rule $r_1$ is
  applicable if the postcondition of $r_2$ does not hold''.
\end{itemize}
We will subsequently come back to this difference\remms{where?}.

Before looking at a formalization, let us summarize this informal exposition
of defeasibility rule modifiers as follows:
\remms{Make the writing of the modifiers more homogenous: in italics or in quotes}
\begin{itemize}
\item ``$r_1$ subject to $r_2$'' and ``$r_1$ despite $r_2$'' are complementary
  ways of expressing that one rule may override the other rule. They have in
  common that $r_1$ and $r_2$ have contradicting conclusions. The conjunction
  of the conclusions can either be directly unsatisfiable (may accept vs.{}
  must not accept) or unsatisfiable \wrt{} an intended background theory
  (obtaining different maximal speeds is inconsistent when expecting
  \texttt{maxSp} to be functional in its fourth argument).
\item Both modifiers differ in that ``subject to'' modifies the rule to which
  it is attached, whereas ``despite'' has a remote effect.
\item They permit to structure a legal text, favoring conciseness and
  modularity: In the case of \emph{despite}, the overridden, typically more
  general rule need not be aware of the overriding, typically subordinate rules.
\item Even though these modifiers appear to be mechanisms on the meta-level in
  that they reasoning about rules, they can directly be reflected on the
  object-level.
\end{itemize}
Making the meaning of the modifiers explicit can therefore be understood as a
\emph{compilation} problem, which will be described in the following.

In L4, rule modifiers are introduced with the aid of \emph{rule annotations}, with a
list of rule names following the keywords \texttt{subjectTo} and
\texttt{despite}. We return to our running example and modify rule
\texttt{maxSpCarHighway} of \figref{fig:rules} with

\begin{lstlisting}
rule <maxSpCarHighway>
  {restrict: {despite: maxSpCarWorkday}}
# rest of rule unchanged
\end{lstlisting}

For the delight of the public of the country with the highest density of
sports cars, we also introduce a new rule:\remms{Problem with spaces in lstlisting}

\begin{lstlisting}
rule <maxSpSportsCar>
  {restrict: {subjectTo: maxSpCarWorkday, 
              despite: maxSpCarHighway}}
   for v: Vehicle, d: Day, r: Road
   if isSportsCar v && isHighway r
   then maxSp v d r 320
 \end{lstlisting}

In the following, we will examine the interaction of these rules.

% ----------------------------------------------------------------------
\subsection{Rule Modifiers in Classical Logic}\label{sec:rule_modifiers_in_classical_logic}

In this section, we will describe how to rewrite rules, progressively
eliminating the instructions appearing in the rule annotations so that in the
end, only purely logical rules remain. Whereas the first preprocessing steps
(\secref{sec:preprocessing}) are generic, we will discuss two variants of
conversion into logical format (\secrefs{sec:restr_precond} and
\ref{sec:restr_deriv}).



% ......................................................................
\subsubsection{Preprocessing}\label{sec:preprocessing}

Preprocessing consists of several elimination steps that are carried out in a
fixed order.

\paragraph{\textbf{``Despite''  elimination}}

As can be concluded from the discussion in \secref{sec:facets}, a
$\mathtt{despite}\; r_2$ clause appearing in rule $r_1$ is equivalent to a
$\mathtt{subjectTo}\; r_1$ clause in rule $r_2$. The first rule transformation
consists in applying exhaustively the following \emph{despite elimination}
rule transformer:
\remms[inline]{In the whole discussion (and the implementation), make a
  clearer distinction between rule set transformer \texttt{restrict} and rule
  generator / transformer \texttt{derived}.}

\noindent
\emph{despiteElim:}\\
$
\{r_1 \{\mathtt{restrict}: \{\mathtt{despite}\; r_2\} \uplus a_1\},\;\;
r_2\{\mathtt{restrict}: a_2\}, \dots\} \longrightarrow$\\
$\{r_1 \{\mathtt{restrict}: a_1\},\;\;
r_2\{\mathtt{restrict}:  \{\mathtt{subjectTo}\; r_1\} \uplus a_2\}, \dots\}
$

\begin{example}\label{ex:rewrite_despite}\mbox{}\\
Application of this rewrite rule to the three example rules \texttt{maxSpCarWorkday},
\texttt{maxSpCarHighway} and  \texttt{maxSpSportsCar} changes them to:

\begin{lstlisting}
rule <maxSpCarWorkday>
  {restrict: {subjectTo: maxSpCarHighway}}
rule <maxSpCarHighway>
  {restrict: {subjectTo: maxSpSportsCar}}
rule <maxSpSportsCar>
  {restrict: {subjectTo: maxSpCarWorkday}}
\end{lstlisting}
Here, only the headings are shown, the bodies of the rules are
unchanged. 
\end{example}

One defect of the rule set already becomes apparent to the human reader at
this point: the circular dependency of the rules. We will however continue
with our algorithm, applying the next step which will be to rewrite the
\texttt{\{restrict: \{subjectTo: \dots\}\}} clauses.  Please note that each
rule can be \texttt{subjectTo} several other rules, each of which may have a
complex structure as a result of transformations that are applied to it.


\paragraph{\textbf{``Subject'' To elimination}}

The rule transformer \emph{subjectToElim} does the following: it splits up the
rule into two rules, (1) its source (the rule body as originally given), and
(2) its definition as the result of applying a rule transformation function to
several rules.

\begin{example}\label{ex:rewrite_subject_to}\
Before stating the rule transformer, we show its effect on rule
\texttt{maxSpCarWorkday} of \exampleref{ex:rewrite_despite}. On rewriting
with \emph{subjectToElim}, the rule is transformed into two rules:

\begin{lstlisting}
# new rule name, body of rule unchanged
rule <maxSpCarWorkday'Orig>
   {source}
   for v: Vehicle, d: Day, r: Road
   if isCar v && isWorkday d
   then maxSp v d r 90

# rule with header and without body
rule <maxSpCarWorkday>
 {derived: {apply: 
 {restrictSubjectTo maxSpCarWorkday'Orig  maxSpSportsCar}}}
\end{lstlisting}
\remms[inline]{check syntax of apply}
\end{example}

We can now state the transformation (after grouping the
\texttt{subjectTo $r_2$}, \dots, \texttt{subjectTo $r_n$} into
\texttt{subjectTo $[r_2 \dots r_n]$}):


\noindent
\emph{subjectToElim:}\\
$
\{r_1 \{\mathtt{restrict}: \{\mathtt{subjectTo}\; [r_2, \dots, r_n]\}\}, \dots \} \longrightarrow$\\
$\{r_1^o \{\mathtt{source}\}, r_1 \{\mathtt{derived:}\; \{\mathtt{apply:}\; \{
\mathtt{restrictSubjectTo}\;\; r_1^o\; [r_2 \dots r_n] \}\}\}, \dots \}
$



\paragraph{\textbf{Computation of derived rule}}
The last step consists in generating the derived rules, by evaluating the
value of the rule transformer expression marked by \texttt{apply}. The rules
appearing in these expressions may themselves be defined by complex
expressions. However, direct or indirect recursion is not allowed. For
simplifying the expressions in a rule set, we compute a rule dependency order
$\preceq_R$ defined by: $r \preceq_R r'$ iff $r$ appears in the defining
expression of $r'$. If $\preceq_R$ is not a partial order (in particular, if
it is not cycle-free), then evaluation fails. Otherwise, we order the rules
topologically by $\preceq_R$ and evaluate the expressions starting from the
minimal elements.

\begin{example}
It is at this point that the cyclic dependence already remarked after
\exampleref{ex:rewrite_despite} will be discovered. We have:

\noindent
\texttt{maxSpCarWorkday'Orig}, \texttt{maxSpCarHighway} $\preceq_R$ \texttt{maxSpCarWorkday}\\
\texttt{maxSpCarHighway'Orig}, \texttt{maxSpSportsCar} $\preceq_R$ \texttt{maxSpCarHighway}\\
\texttt{maxSpSportsCar'Orig}, \texttt{maxSpCarWorkday} $\preceq_R$  \texttt{maxSpSportsCar}\\
\noindent
which cannot be totally ordered.

Let us fix the problem by changing the heading of rule
\texttt{maxSpCarHighway} from \texttt{despite} to \texttt{subjectTo}:
\begin{lstlisting}
rule <maxSpCarHighway>
  {restrict: {subjectTo: maxSpCarWorkday}}
\end{lstlisting}

After rerunning \emph{despiteElim} and \emph{subjectToElim}, we can now order
the rules:

% \begin{lstlisting}
% rule <maxSpCarWorkday>
% rule <maxSpCarHighway>
%   {restrict: {subjectTo: maxSpCarWorkday, maxSpSportsCar}}
% rule <maxSpSportsCar>
%   {restrict: {subjectTo: maxSpCarWorkday}}
% \end{lstlisting}

\noindent
\{ \texttt{maxSpSportsCar'Orig}
\texttt{maxSpCarHighway'Orig},
\texttt{maxSpCarWorkday} \} $\preceq_R$
\texttt{maxSpSportsCar} $\preceq_R$
\texttt{maxSpCarHighway}\\
and will use this order for rule elaboration.
\end{example}


% ......................................................................
\subsubsection{Restriction via Preconditions}\label{sec:restr_precond}

Here, we propose one possible implementation of the rule transformer
\texttt{restrictSubjectTo} introduced in \secref{sec:preprocessing} that takes
a rule $r_1$ and a list of rules $[r_2 \dots r_n]$ and produces a new rule, by
adding the negation of the preconditions of $[r_2 \dots r_n]$ to $r_1$. More
formally:

\begin{itemize}
\item $\mathtt{restrictSubjectTo}\; r_1\; [] = r_1$
\item $\mathtt{restrictSubjectTo}\; r_1\; (r' \uplus rs) =$\\
  $\mathtt{restrictSubjectTo}\; (r_1(precond := precond(r_1) \AND \NOT precond(r')))\; rs$
\end{itemize}
where $precond(r)$ selects the precondition of rule $r$ and $r(precond:=p)$
updates the precondition of rule $r$ with $p$.

There is one proviso to the application of \texttt{restrictSubjectTo}: the
rules have to have the same \emph{parameter interface}: the number and types
of the parameters in the rules' \texttt{for} clause have to be the same.
Rules with different parameter interfaces can be adapted via the
\texttt{remap} rule transformer.\remms{Promise} The rule 

\begin{lstlisting}[frame=none,mathescape=true]
rule <r> for $x_1$:$T_1$ $\dots$ $x_n$:$T_n$ if Pre($x_1, \dots, x_n$) then Post($x_1, \dots, x_n$)
\end{lstlisting}
is remapped with 
\begin{lstlisting}[frame=none,mathescape=true]
remap r [$y_1: S_1, \dots, y_m: S_m$] [$x_1 := e_1, \dots, x_n := e_n$]
\end{lstlisting}
to the rule
\begin{lstlisting}[frame=none,mathescape=true]
rule <r> for $y_1$: $S_1$ $\dots$ $y_m$: $S_m$ if Pre($e_1, \dots, e_n$) then Post($e_1, \dots, e_n$)
\end{lstlisting}
Here, $e_1, \dots, e_n$ are expressions that have to be well-typed with types $E_1, \dots, E_n$ in
context $y_1: S_1, \dots, y_m: S_m$ (which means in particular that they may
contain the variables $y_i$) with $E_i \preceq T_i$ (with the consequence that
the pre- and post-conditions of the new rule remain well-typed). 


\begin{example}\mbox{}\\
We come back to the running example. When processing the rules in the order of
$\preceq_R$, rule \texttt{maxSpSportsCar}, defined by
\texttt{apply: \{restrictSubjectTo maxSpSportsCar'Orig maxSpCarWorkday\}},
becomes:
\begin{lstlisting}
rule <maxSpSportsCar>
   for v: Vehicle, d: Day, r: Road
   if isSportsCar v && isHighway r &&
      not (isCar v && isWorkday d)
   then maxSp v d r 320
 \end{lstlisting}

 We can now state \texttt{maxSpCarHighway}, which has been defined by
 \texttt{apply: \{restrictSubjectTo maxSpCarHighway'Orig maxSpSportsCar\}}, as:

 \begin{lstlisting}
rule <maxSpCarHighway>
   for v: Vehicle, d: Day, r: Road
   if isCar v && isHighway r &&
      not (isSportsCar v && isHighway r &&
           not (isCar v && isWorkday d))
   then maxSp v d r 130
\end{lstlisting}
\end{example}

One downside of the approach of adding negated preconditions is that the
preconditions of rules can become very complex. This effect is mitigated by
the fact that conditions in \texttt{subjectTo} and \texttt{despite} clauses
express specialization or refinement and often permit substantial
simplifications. Thus, the precondition of \texttt{maxSpSportsCar} simplifies
to \texttt{isSportsCar v \&\& isHighway r \&\& isWorkday d} and the
precondition of \texttt{maxSpCarHighway} to
\texttt{isCar v \&\& isHighway r \&\& not (isSportsCar v \&\& isWorkday d)}.


% ......................................................................
\subsubsection{Restriction via Derivability}\label{sec:restr_deriv}

We now give an alternative reading of \texttt{restrictSubjectTo}. To
illustrate the point, let us take a look at a simple propositional example.

\begin{example}\label{ex:small_propositional} Take the definitions:
\begin{lstlisting}
rule <r1> if B1 then C1
rule <r2> {subjectTo: r1} if B2 then C2
\end{lstlisting}
\end{example}

Instead of saying: \texttt{r2} corresponds to
\texttt{\blue{if} B2 \&\& not B1 \blue{then} C2} 
as in \secref{sec:restr_precond}, we would now read it as
``if the conclusion of \texttt{r1} cannot be derived'', 
which could be written as
\texttt{\blue{if} B2 \&\& not C1 \blue{then} C2}.
The two main problems with this naive approach are the following:
\begin{itemize}
\item As mentioned in \secref{sec:intro_rule_modifiers}, a \emph{subject to}
  restriction is often applied to rules with contradicting conclusions, so in
  the case that \texttt{C1} is \texttt{not C2}, the generated rule would be a
  tautology.
\item In case of the presence of a third rule
\begin{lstlisting}[frame=none]
rule <r3> if B3 then C1
\end{lstlisting}
a derivation of \texttt{C1} from \texttt{B3} would also block the application
of \texttt{r2}, and \texttt{subjectTo: r1} and \texttt{subjectTo: r1, r3}
would be indistinguishable.
\end{itemize}

We now sketch a solution for rule sets whose conclusion is always an atom (and
not a more complex formula).

\begin{enumerate}
\item In a preprocessing stage, all rules are transformed as follows:
  \begin{enumerate}
  \item We assume the existence of a class \texttt{Rulename}, which we will
    take to be \texttt{String} in the following.
  \item All the predicates $p$
    occurring in the conclusions of rules (called \emph{transformable
      predicates}) are converted into predicates $p^+$ with one additional
    argument of type \texttt{Rulename}. In the
    example, \texttt{C1$^+$: Rulename -> Boolean} and similarly for \texttt{C2}.
  \item The transformable predicates $p$ in conclusions of rules receive one
    more argument, which is the name \emph{rn} of the rule: $p$ is transformed
    into $p^+\; rn$. The informal reading is ``the predicate is derivable with
    rule \emph{rn}''.
  \item All transformable predicates in the preconditions of the rules receive
    one more argument, which is a universally quantified variable of type
    \texttt{Rulename} bound in the \texttt{for}-list of the rule.
  \end{enumerate}
\item In the main processing stage, \texttt{restrictSubjectTo} in the rule
  annotations generates rules according to:
  \begin{itemize}
\item $\mathtt{restrictSubjectTo}\; r_1\; [] = r_1$
\item $\mathtt{restrictSubjectTo}\; r_1\; (r' \uplus rs) =$\\
  $\mathtt{restrictSubjectTo}\; (r_1(precond := precond(r_1) \AND \NOT postcond(r')))\; rs$
  Thus, the essential difference \wrt{} the definition of
  \secref{sec:restr_precond} is that we add the negated postcondition.
\end{itemize}
\end{enumerate}

\begin{example} The rules of \exampleref{ex:small_propositional} are now
  transformed to:
\begin{lstlisting}[mathescape=true]
rule <r1> for rn: Rulename if B1$^+$ rn then C1$^+$ r1
rule <r2> for rn: Rulename if B2$^+$ rn and not C1$^+$ r1 then C2$^+$ r2
\end{lstlisting}
The derivability of another instance of \texttt{C1}, such as \texttt{C1$^+$ r3},
would not inhibit the application of \texttt{r2} any more.
\end{example}


\begin{example} The two rules of the running example become, after resolution
  of the \texttt{restrictSubjectTo} clauses:
\begin{lstlisting}[mathescape=true]
rule <maxSpSportsCar>
   for v: Vehicle, d: Day, r: Road
   if isSportsCar v && isHighway r &&
      not maxSp$^+$ maxSpCarWorkday v d r 90
   then maxSp v d r 320
rule <maxSpCarHighway>
   for v: Vehicle, d: Day, r: Road
   if isCar v && isHighway r &&
      not maxSp$^+$ maxSpCarWorkday v d r 90 &&
      not maxSp$^+$ maxSpSportsCar v d r 320
   then maxSp v d r 130
\end{lstlisting}
\end{example}

% ----------------------------------------------------------------------
\subsection{Rule Inversion}\label{sec:rule_inversion}

The purpose of this section is to derive formulas that, for a given rule set,
simulate negation as failure, but are coded in a classical first-order logic,
do not require a dedicated proof engine (such as Prolog) and can be checked
with a SAT or SMT solver. The net effect is similar to the completion
introduced by Clark \cite{clark_NegAsFailure_1978}; however, the justification
is not operational as in \cite{clark_NegAsFailure_1978}, but takes 
inductive closure as a point of departure. Apart from that, our technique
applies to a considerably wider class of formulas.

In the following, we assume that our rules have an atomic predicate $P$ as
conclusion, whereas the precondition $Pre$ can be an arbitrarily complex
formula.  We furthermore assume that rules are in \emph{normalized form}: $P$
may only be applied to $n$ distinct variables $x_1, \dots, x_n$, where $n$ is
the arity of $P$, and the rule quantifies over exactly these variables.
For notational simplicity, we write normalized rules in logical format,
ignoring types:
$\forall x_1, \dots, x_n. Pre(x_1, \dots x_n) \IMPL Post(x_1, \dots, x_n)$.

Every rule can be written in normalized form, by applying the following
algorithm:
\begin{itemize}
\item Remove expressions or duplicate variables in the conclusion, by using
  the equivalences $P(\dots e \dots) = (\forall x. x = e \IMPL P(\dots x
  \dots))$ for a fresh variable x, and similarly $P(\dots y \dots y \dots) =
  (\forall x. x = y   \IMPL P(\dots x \dots y \dots))$.
\item Remove variables from the universal quantifier prefix if they do not
  occur in the conclusion, by using the equivalence
  $(\forall x. Pre(\dots x \dots) \IMPL P) = (\exists x. Pre(\dots x \dots))
  \IMPL P$.
\end{itemize}

For any rule set $R$ and any predicate $P$, we can form the set of $P$-rules
$\{\forall x_1, \dots, x_n. Pre_1(x_1, \dots x_n) \IMPL P(x_1, \dots, x_n),
\dots, \forall x_1, \dots, x_n. Pre_k(x_1, \dots x_n)$ $\IMPL P(x_1, \dots,
x_n)\}$ as the subset of $R$ containing all rules having $P$ as
post-condition. 

The inductive closure of a set of $P$-rules is the predicate $P^*$ defined by
the second-order formula
\begin{align*}
  P^*(x_1, \dots, x_n) = \forall P.\;  & (Pre_1(x_1, \dots x_n) \IMPL P(x_1, \dots, x_n)) \IMPL \dots \\
                         & (Pre_k(x_1, \dots x_n) \IMPL P(x_1, \dots, x_n)) \IMPL\\
                         & P(x_1, \dots, x_n)
\end{align*}

$P^*$ can be understood as the least predicate satisfying the set of $P$-rules
and is the predicate that represents ``all that is known about $P$ and
assuming nothing else about $P$ is true'', and corresponds to the notion of
exhaustiveness prevalent in law texts. It can also be understood as the static
equivalent of the operational concept of negation as failure for predicate $P$.

As an illustration, let us note that in case the $P$-rule set is empty, \ie there
are no rules establishing $P$, we have $P^*(x_1, \dots, x_n) = \forall
P.\;  P(x_1, \dots, x_n) = \bot$.

Obviously, a second-order formula such as the definition of $P^*$ is unwieldy
in fully automated theorem proving. We derive one consequence:

\begin{lemma}
$P^*(x_1, \dots, x_n) \IMPL Pre_1(x_1, \dots x_n) \OR \dots \OR Pre_k(x_1, \dots x_n)$
\end{lemma}
\begin{proof}
Expand the definition of $P^*$ and instantiate the universal variable $P$ with
$\lambda x_1 \dots x_n. Pre_1(x_1, \dots x_n) \OR \dots \OR Pre_k(x_1, \dots x_n)$.
\end{proof}
As a consequence of the Löwenheim–Skolem theorem, there is no first-order
equivalent of $P^*$: a formula of the form $P^*$ can characterize the natural
numbers up to isomorphism, but no first-order formula can. In the absence of
such a first-order equivalent, we define the formula
\[
\forall x_1, \dots, x_n.\;  P(x_1, \dots, x_n) \IMPL Pre_1(x_1, \dots x_n) \OR \dots \OR Pre_k(x_1, \dots x_n)
\]
called the \emph{inversion formula of  $P$}, as an approximation of the
effect of $P^*$.

Inversion formulas can be automatically derived and added to the rule set in
L4 proofs; they turn out to be essential for consistency properties. For
example, the functionality of \texttt{maxSp} stated in \figref{fig:assertions}
is not provable without the inversion formula of \texttt{maxSp}.

To avoid misunderstandings, we should emphasize that this approach is entirely
based on a classical monotonic logic, in spite of non-monotonic
effects. Adding a new $P$-rule may invalidate previously provable facts, but
this is only so because the new rule alters the inversion formula of $P$.




% ----------------------------------------------------------------------
\subsection{Answer Set Programming}\label{sec:answer_set_programming}

\remms[inline]{section probably does not have an adequate name}

% ----------------------------------------------------------------------
\subsection{Comparison}\label{sec:comparison}





%%% Local Variables:
%%% mode: latex
%%% TeX-master: "main"
%%% End:


\section{Defeasible Reasoning with Answer Set Programming}\label{sec:defeasible_asp}



\subsection{Introduction}
The purpose of this section is to give an account of the work we have been doing using Answer Set Programming (ASP) to formalize and reason about legal rules. This approach is complementary to the one described before using SMT solvers. Our intention is to present how some core legal reasoning tasks can be implemented in ASP while keeping the ASP representation readable and intuitive and respecting the idea of having an `isomorphism' between the rules and the encoding. Please see the appendix for a brief overview of ASP and references for further reading.

Our work in this section is inspired by \cite{DBLP:conf/iclp/WanGKFL09}. Readers will note that there are similarities between the use of predicates such as $according\_to$, $defeated$, $opposes$ in our ASP encoding, to reason about rules interacting with each other, and similar predicates that the authors of \cite{DBLP:conf/iclp/WanGKFL09} use in their work. However our ASP implementation is much more specific to legal reasoning whereas they seek to implement very general logic based reasoning mechanisms. We independently developed our `meta theory' for how rule modifiers interact with the rules and with each other and there are further original contributions like a proposed axiom system for what we call `legal models'. An interesting avenue of future work could be to compare our approaches within the framework of legal reasoning.

The work in this section builds on the work in \cite{morris21:_const_answer_set_progr_tool} and hence uses some of the same notation and terminology. 
% The author of \cite{morris21:_const_answer_set_progr_tool} was a member of the same research group as the authors of this paper at SMU in 2020--2021.


\subsection{Formal Setup}
Let the tuple $Config = (R,F,M,I)$ denote a $configuration$ of legal rules. The set $R$ denotes a set of rules of the form $pre\_con(r)\rightarrow concl(r)$. These are `naive' rules with no information pertaining to any of the other rules in $R$. $F$ is a set of positive atoms that describe facts of the legal scenario we wish to consider. $M$ is a set of the binary predicates $despite$, $subject\_to$ and $strong\_subject\_to$. $I$ is a collection of minimal inconsistent sets of positive atoms. Henceforth for a rule $r$, we may write $C_{r}$ for its conclusion $Concl(r)$.

Note that, throughout this section, given any rule $r$, $C_{r}$ is assumed to be a single positive atom. That is, there are no disjunctions or conjunctions in rule conclusions. Also any rule pre-condition ($pre\_con(r)$) is assumed to be a conjunction of positive and negated atoms. Here negation denotes `negation as failure'.  

Throughout this document, whenever we use an uppercase or lowercase letter (like $r$, $r_{1}$, $R$ etc.) to denote a rule that is an argument, in a binary predicate, we mean the unique integer rule \textsc{id} associated with that rule. The binary predicate $legally\_valid(r,c)$ intuitively means that the rule $r$ is `in force' and it has conclusion $c$. Here $r$ typically is an integer referring to the rule \textsc{id} and $c$ is the atomic conclusion of the rule. The unary predicate $is\_legal(c)$ intuitively means that the atom $c$ legally holds/has legal status. The predicates $despite$, $subject\_to$ and $strong\_subject\_to$ all cause some rules to override others. Their precise properties will be given next.


\subsection{Semantics}

A set $S$ of $is\_legal$ and $legally\_valid$ predicates is called a \textit{legal model} of $Config = (R,F,M,I)$, if and only if
\begin{description}
\item[(A1)]$\forall f \in F$ $is\_legal(f) \in S$.

\item[(A2)] $\forall r \in R$, if $legally\_valid(r,C_{r}) \in S$. then $S\models is\_legal(pre\_con(r))$ and $S\models is\_legal(C_{r})$ \footnote{By $S\models is\_legal(pre\_con(r))$ we mean that for each positive atom $b_{i}$ in the conjunction, $is\_legal(b_{i}) \in S$ and for each negation-as-failure body atom $not$ $b_{j}$ in the conjunction $is\_legal(b_{j})\notin S$ }

\item[(A3)] $\forall c$, if $is\_legal(c) \in S$, then either $c\in F$ or there exists $r \in R$ such that $legally\_valid(r,C_{r}) \in S$ and $c= C_{r}$.

\item[(A4)] $\forall r_{i}, r_{j} \in R$, if $despite(r_{i}, r_{j}) \in M$ and $S\models is\_legal(pre\_con(r_{j}))$, then $legally\_valid(r_{i},C_{r_{i}}) \notin S$

\item[(A5)] $\forall r_{i}, r_{j} \in R$, if $strong\_subject\_to(r_{i}, r_{j}) \in M$ and $legally\_valid(r_{i},C_{r_{i}}) \in S$, then $legally\_valid(r_{j},C_{r_{j}}) \notin S$


\item[(A6)] $\forall r_{i},r_{j} \in R$ if $subject\_to(r_{i},r_{j}) \in M$, and $legally\_valid(r_{i},C_{r_{i}}) \in S$ and there exists a minimal conflicting set $k \in I$ such that $C_{r_{i}} \in k$ and $C_{r_{j}}\in k$ and $is\_legal(k\setminus \{C_{r_{j}})\})\subseteq S $, then $legally\_valid(r_{j},C_{r_{j}}) \notin S$. Note than in our system, any minimal inconsistent set must contain at least 2 atoms. \footnote{For a set of atoms $A$, by $is\_legal(A)$, we mean the set $\{is\_legal(a)\mid a\in A\}$} 

\item[(A7)] $\forall r\in R$, if $S\models pre\_con(r)$, but $legally\_valid(r,C_{r})\notin S$, then it must be the case that at least one of A4 or A5 or A6 has caused the exclusion of $legally\_valid(r,C_{r})$. That is if $S\models pre\_con(r)$, then unless this would violate one of A5, A6 or A7, it must be the case that $legally\_valid(r,C_{r})\in S$.
\end{description}

\subsection{Some remarks on axioms A1--A7}
We now give some informal intuition behind some of the axioms and their intended effects.

A1 says that all facts in $F$ automatically gain legal status, that is, they legally hold. The set $F$ represents indisputable facts about the legal scenario we are considering.

A2 says that if a rule is `in force' then it must be the case that both the pre-condition and conclusion of the rule have legal status. Note that it is not enough if simply require that the conclusion has legal status as more than one rule may enforce the same conclusion or the conclusion may be a fact, so we want to know exactly which rules are in force as well as their conclusions.

A3 says that anything that has legal status must either be a fact or be a conclusion of some rule that is in force.

A4--A6 describe the semantics of the three modifiers. The intuition for the three modifiers will be discussed next. Firstly, it may help the reader to read the modifiers in certain ways. $despite(r_{i},r_{j})$ should be read as `despite $r_{i}$, $r_{j}$'. Thus $r_{i}$ here is the `subordinate rule' and $r_{j}$ is the `dominating' rule. The idea here is that once the precondition of the dominating rule $r_{j}$ is true, it invalidates the subordinate rule $r_{i}$ regardless of whether the dominating rule itself is then invalidated by some other rule. For \textit{strong subject to}, the intended reading for $strong\_subject\_to(r_{i},r_{j})$ is something like `(strong) subject to $r_{i}$, $r_{j}$'. Here $r_{i}$ can be considered the dominating rule and $r_{j}$ the subordinate. Once the dominating rule is in force, then it invalidates the subordinate rule. The intended reading for $subject\_to(r_{i},r_{j})$ is `subject to $r_{i}$, $r_{j}$'. For the subordinate rule $r_{j}$ to be invalidated, it has to be the case that the dominating rule $r_{i}$ is in force and there is a minimal inconsistent set $k$ in $I$ that contains the two atoms in the conclusions of the two rules and, $is\_legal(k\setminus \{C_{r_{j}})\})\subseteq S $. These minimal inconsistent sets along with the \textit{subject to} modifier give us a way to incorporate a classical-negation-like effect into our system. We are able to say which things contradict each other. Note that in our system, if say $\{a,b\}$ is a minimal inconsistent set, then it is possible for both $is\_legal(a)$ and $is\_legal(b)$ to be in a single legal model, if they are both facts or they are conclusions of rules that have no modifiers linking them. These minimal inconsistent sets only play a role where a $subject\_to$ modifier is involved. The reason for doing this is that this offers greater flexibility rather than treating $a$ and $b$ as pure logical negatives of each other that cannot be simultaneously true in a legal model. We will give examples later on to illustrate these modifiers.

A7 says essentially that A4--A6 represent the only ways in which a rule whose pre-condition is true may nevertheless be invalidated, and any rule whose precondition is satisfied and is not invalidated directly by some instance of A4--A6, must be in force. 

Note that there maybe legal rule configurations for which no legal models exist. See the appendix for a discussion of some 'pathological' rule configurations.

\subsection{ASP encoding}
Here is an ASP encoding scheme for a configuration $Config = (R,F,M,I)$ of legal rules.
\begin{lstlisting}[language=Prolog, numbers=left]
% For any f in F, we have:
is_legal(f). 
% All the modifiers get added as facts like for example:
despite(1,2).
% Any rule r in R is encoded using the general schema:
according_to(r,C_r):-is_legal(pre_con(r)).
% Given a minimal inconsistent set {a_1,a_2,...,a_n}, this corresponds to a set of rules:
opposes(a_1,a_2):-is_legal(a_2),is_legal(a_3),...,is_legal(a_n).
opposes(a_1,a_3):-is_legal(a_2),is_legal(a_4)...,is_legal(a_n).  % etc ...
opposes(a_n-1,a_n):-is_legal(a_1),...,is_legal(a_n-2).               
% Opposes is a symmetric relation
opposes(X,Y):-opposes(Y,X).
% Encoding for 'despite'
defeated(R,C,R1) :-
    according_to(R,C), according_to(R1,C1), despite(R,R1).
%Encoding for 'subject_to'
defeated(R,C,R1) :-
    according_to(R,C), legally_valid(R1,C1),
    opposes(C,C1), subject_to(R1,R).
% Encoding for 'strong_subject_to'
defeated(R,C,R1) :-
    according_to(R,C), legally_valid(R1,C1),
    strong_subject_to(R1,R).

not_legally_valid(R) :- defeated(R,C,R1).
legally_valid(R,C):-according_to(R,C),not not_legally_valid(R).
is_legal(C):-legally_valid(R,C).
\end{lstlisting}
\subsection{Proposition}

\begin{proposition}\label{lemma:legal_model_of_config}
For a configuration $Config=(R,F,M,I)$, let the above encoding be the program $ASP_{Config}$. Then given an answer set $A_{Config}$ of $ASP_{Config}$ let $S_{A_{Config}}$ be the set of $is\_legal$ and $legally\_valid$ predicates in $A_{Config}$. Then $S_{A_{Config}}$ is a legal model of $Config$. 
\end{proposition}


$Proof$ See Appendix. $\square$


\subsection{Example}
Let us see how the running example would work in the ASP setting. We have 3 rules encoded as below, there are no minimal inconsistent sets. There are 3 modifiers: $despite(2,3)$,                           $strong\_subject\_to(1,3)$, $strong\_subject\_to(1,2)$
\begin{lstlisting}[language=Prolog, numbers=left]

despite(2,3).
strong_subject_to(1,3).
strong_subject_to(1,2).

according_to(1,max_spd(v,d,r,90)):-is_legal(is_workday(d)),
is_legal(is_car(v)).
according_to(2,max_spd(v,d,r,130)):-is_legal(is_highway(r)),is_legal(is_car(v)).
according_to(3,max_spd(v,d,r,320)):-is_legal(is_highway(r)),is_legal(is_sports_car(v)).

% Encoding for 'despite'
defeated(R,C,R1) :-
    according_to(R,C), according_to(R1,C1), despite(R,R1).
%Encoding for 'subject_to'
defeated(R,C,R1) :-
    according_to(R,C), legally_valid(R1,C1),
    opposes(C,C1), subject_to(R1,R).
% Encoding for 'strong_subject_to'
defeated(R,C,R1) :-
    according_to(R,C), legally_valid(R1,C1),
    strong_subject_to(R1,R).

not_legally_valid(R) :- defeated(R,C,R1).
legally_valid(R,C):-according_to(R,C),not not_legally_valid(R).
is_legal(C):-legally_valid(R,C).
\end{lstlisting}

When the initial set of facts $F$ is the set
\begin{lstlisting}[language=Prolog, numbers=left]
is_legal(is_workday(d)).
is_legal(is_car(v)).
is_legal(is_highway(r)).
is_legal(is_sports_car(v)).
\end{lstlisting}
we get exactly one legal max speed given by 
\begin{lstlisting}[language=Prolog, numbers=left]
is_legal(max_spd(v,d,r,90)).
\end{lstlisting}
One can check this by adding the rule 
\begin{lstlisting}[language=Prolog, numbers=left]
legal_max_spd(X):- is_legal(max_spd(v,d,r,X)). 
\end{lstlisting}
and running the s(CASP) query $?- legal\_max\_spd(X).$, which returns the binding $X = 90$. This is the unique legal maximum speed which can be seen via use of the rule 
\begin{lstlisting}[language=Prolog, numbers=left]
legal_max_spd(X):- X > 90, is_legal(max_spd(v,d,r,X)). 
\end{lstlisting}
Now running the query as above we see that there is no solution. 
When $is\_legal(is\_workday(d))$ is removed from $F$ we get exactly one legal max speed of $320$, and when  $is\_legal(is\_sports\_car(v))$, and $is\_legal(is\_workday(d))$ are both removed from $F$ we get exactly one exactly legal max speed of $130$.\\
We note here that if all 4 facts as in the first initial fact set are passed to the SMT solver following the approach presented in \secref{sec:defeasible_classical} then the SMT solver will also return exactly one max speed of $90$. If instead we choose to modify the initial fact set by negating $is\_work\_day(d)$, then the SMT solver will return exactly one max speed of $320$ and if modify the initial fact set by negating both  $is\_work\_day(d)$ and  $is\_sports\_car(v)$, we get exactly one max speed of $130$.







%%% Local Variables:
%%% mode: latex
%%% TeX-master: "main"
%%% End:


\section{Conclusions}\label{sec:conclusions}

This paper has discussed different approaches for representing defeasibility
as used in law texts, by annotating rules with modifiers that explicate their
relation to other rules. We have notably presented encodings in classical
logic (\secrefs{sec:restr_precond}) and explained
how they are related. Quite a different approach,
based on Answer Set Programming, is presented in
\secref{sec:defeasible_asp}. All the encodings are motivated by the need to
explore the implementation of various forms of defeasibility in logics for
which well developed and powerful solvers such as SMT solvers and ASP solvers
already exist. The ASP based and SMT based approaches are complimentary to
each other. ASP's closed-world-assumption is useful when a legal verdict must
be reached with potentially incomplete information, when the truth value of
every atom is not explicitly known. For example in the speed limit example,
unless it is known explicitly that the car in question was a sports car, the
ASP solver would always return a speed limit lower than $320$. The SMT solver
would generate two models, one in which the car is a sports car and one in
which it is not. This could lead to two different speed limits, hence
requiring human intervention to determine the true speed limit in the scenario
being considered. On the other hand, when one does model checking or wants to
reason about meta-properties of the rules such as soundness of the rule-set,
the SMT solver approach is more suited than the ASP approach. Intuitively,
here we want to reason over all possible scenarios rather than restricting the
reasoning to a particular scenario being considered.

However, we are still at the beginning of the journey. To allow the two
approaches to work together coherently and seamlessly, a theoretical
comparison of the classical and ASP semantics presented here still has to be
carried out, and it has to be propped up by an empirical evaluation. For this
purpose, we are currently in the process of coding some real-life law texts in
L4, such as Singapore's Personal Data Protection
Act\footnote{\url{https://sso.agc.gov.sg/Act/PDPA2012}}.  An implementation of
the L4 ecosystem is under
way\footnote{\url{https://github.com/smucclaw/baby-l4}}, providing a
transpilation of L4 rules to both the SMT and the ASP world. The interaction
with SMT solvers is done through an SMT-LIB \cite{BarFT_SMTLIB}
interface. Advanced solvers, such as Z3 \cite{demoura_bjorner_z3_2008},
provide good support for quantification.

% , but it has not yet
% reached a stable, user-friendly status. It is implemented in Haskell and features
% an IDE based on VS Code and natural language processing via Grammatical
% Framework \citep{ranta_grammatical_2004}. Currently, only the coding of
% \secrefs{sec:restr_precond} has been implemented---the coding of
% \secref{sec:restr_deriv} would require a much more profound transformation of
% rules and assertions. Likewise, the ASP coding of \secref{sec:defeasible_asp}
% has only been done manually. The interaction with SMT solvers is done through
% an SMT-LIB \citep{BarFT_SMTLIB} interface, thus opening the possibility to
% interact with a wide range of solvers. As our rules typically contain
% quantification, reasoning with quantifiers is crucial, and the best support
% currently seems to be provided by Z3 \citep{demoura_bjorner_z3_2008}.



% We are
% fully aware of shortcomings of the current L4, which will strongly evolve in
% the next months, to include reasoning about deontics and about temporal
% relations. Integrating these aspects will not be easy, and this is one reason
% for not committing prematurely to one particular logical framework.


%%% Local Variables:
%%% mode: latex
%%% TeX-master: "main"
%%% End:


\begin{acknowledgments}
The contributions of the members of the L4 team to this common effort are
thankfully acknowledged, in particular of Jason Morris who contributed his
experience with Answer Set Programming; of Jacob Tan and Ruslan Khafizov who
have participated in discussions about its contents and commented on the
paper; and of Liyana Muthalib who has proof-read a previous version.

This research is supported by the National Research Foundation (NRF),
Singapore, under its Industry Alignment Fund –- Pre-Positioning Programme, as
the Research Programme in Computational Law. Any opinions, findings and
conclusions or recommendations expressed in this material are those of the
authors and do not reflect the views of National Research Foundation,
Singapore.
\end{acknowledgments}


%----------------------------------------------------------------------
\bibliography{gde22}


\end{document}


%%% Local Variables:
%%% mode: latex
%%% TeX-master: t
%%% End:
