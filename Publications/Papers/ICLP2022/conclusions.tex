\section{Conclusions}\label{sec:conclusions}

This paper has discussed different approaches for representing defeasibility
as used in law texts, by annotating rules with modifiers that explicate their
relation to other rules. We have notably presented two encodings in classical
logic (\secrefs{sec:restr_precond} and \ref{sec:restr_deriv}) and explained
how they are related (\secref{sec:comparison}). Quite a different approach,
based on Answer Set Programming, is presented in
\secref{sec:defeasible_asp}. \remam{Added by Avishkar} All the encodings are motivated by the need to explore the implementation of various forms of defeasibility in logics for which well developed and powerful solvers such as SMT solvers and ASP solvers already exist. The ASP based and SMT based approaches are complimentary to each other. ASP's closed-world-assumption is useful when a legal verdict must be reached with potentially incomplete information, when the truth value of every atom is not explicitly known. For example in the speed limit example, unless it is known explicitly that the car in question was a sports car, the ASP solver would always return a speed limit lower than $320$. The SMT solver would generate two models, one in which the car is a sports car and one in which it is not. This could lead to two different speed limits, hence requiring human intervention to determine the true speed limit in the scenario being considered. On the other hand, when one does model checking or wants to reason about meta-properties of the rules such as soundness of the rule-set, the SMT solver approach is more suited than the ASP approach. Intuitively, here we want to reason over all possible scenarios rather than restricting the reasoning to a particular scenario being considered. However, we are still at the beginning of the journey. To allow the two approaches to work together coherently and seamlessly, a theoretical comparison of the classical and ASP semantics presented here still has to be carried out, and it has to be propped up by an empirical evaluation. For this purpose, we are currently in the process of coding some real-life law texts in L4, such as Singapore's Personal Data Protection Act \citep{ppda}.
An experimental implementation of the L4 ecosystem is under way\footnote{\url{https://github.com/smucclaw/baby-l4}}. The interaction with SMT solvers is done through an SMT-LIB \citep{BarFT_SMTLIB} interface. Advanced solvers, such as Z3 \citep{demoura_bjorner_z3_2008}, provide good support for quantification. \remms{Say something more about the ASP part}


% , but it has not yet
% reached a stable, user-friendly status. It is implemented in Haskell and features
% an IDE based on VS Code and natural language processing via Grammatical
% Framework \citep{ranta_grammatical_2004}. Currently, only the coding of
% \secrefs{sec:restr_precond} has been implemented---the coding of
% \secref{sec:restr_deriv} would require a much more profound transformation of
% rules and assertions. Likewise, the ASP coding of \secref{sec:defeasible_asp}
% has only been done manually. The interaction with SMT solvers is done through
% an SMT-LIB \citep{BarFT_SMTLIB} interface, thus opening the possibility to
% interact with a wide range of solvers. As our rules typically contain
% quantification, reasoning with quantifiers is crucial, and the best support
% currently seems to be provided by Z3 \citep{demoura_bjorner_z3_2008}.



% We are
% fully aware of shortcomings of the current L4, which will strongly evolve in
% the next months, to include reasoning about deontics and about temporal
% relations. Integrating these aspects will not be easy, and this is one reason
% for not committing prematurely to one particular logical framework.


%%% Local Variables:
%%% mode: latex
%%% TeX-master: "main"
%%% End:
