\section{Rule Transformation Strategies}\label{sec:rule_transformation_strategies}

In \secref{sec:rule_modifiers_in_classical_logic}, we have discussed one
approach for coding defeasibility in a classical, monotonic logic. We will now
present a second approach (\secref{sec:restr_deriv}) that can potentially lead to more compact
preconditions, but that is more complex because it requires a change of the
signature of predicates. It has so far not been implemented. We compare the
two approaches in \secref{sec:comparison}, showing that they produce
essentially the same derivable models.

% ......................................................................
\subsection{Restriction via Derivability}\label{sec:restr_deriv}

We now give an alternative reading of \texttt{restrictSubjectTo} introduced in
\secref{sec:preprocessing}. To
illustrate the point, let us take a look at a simple propositional example.

\begin{example}\label{ex:small_propositional} Take the definitions:
\begin{lstlisting}
rule <r1> if B1 then C1
rule <r2> {subjectTo: r1} if B2 then C2
\end{lstlisting}
\end{example}

Instead of saying: \texttt{r2} corresponds to
\texttt{\blue{if} B2 \&\& not B1 \blue{then} C2} 
as in \secref{sec:restr_precond}, we would now read it as
``if the conclusion of \texttt{r1} cannot be derived'', 
which could be written as
\texttt{\blue{if} B2 \&\& not C1 \blue{then} C2}.
The two main problems with this naive approach are the following:
\begin{itemize}
\item As mentioned in \secref{sec:resasoning_with_rules}, a \emph{subject to}
  restriction is often applied to rules with contradicting conclusions, so in
  the case that \texttt{C1} is \texttt{not C2}, the generated rule would be a
  tautology.
\item In case of the presence of a third rule
\begin{lstlisting}[frame=none]
rule <r3> if B3 then C1
\end{lstlisting}
a derivation of \texttt{C1} from \texttt{B3} would also block the application
of \texttt{r2}, and \texttt{subjectTo: r1} and \texttt{subjectTo: r1, r3}
would be indistinguishable.
\end{itemize}

We now sketch a solution for rule sets whose conclusion is always an atom (and
not a more complex formula).

\begin{enumerate}
\item In a preprocessing stage, all rules are transformed as follows:
  \begin{enumerate}
  \item We assume the existence of classes \texttt{Rulename$_P$}, one for each
    transformable predicate $P$ (see below).
  \item All the predicates $P$
    occurring in the conclusions of rules (called \emph{transformable
      predicates}) are converted into predicates $P^+$ with one additional
    argument of type \texttt{Rulename$_P$}. In the
    example, \texttt{C1$^+$: Rulename$_{C1}$ -> Boolean} and similarly for \texttt{C2}.
  \item The transformable predicates $P$ in conclusions of rules receive one
    more argument, which is the name \emph{rn} of the rule: $P$ is transformed
    into $P^+\; rn$. The informal reading is ``the predicate is derivable with
    rule \emph{rn}''.
  \item All transformable predicates in the preconditions of the rules receive
    one more argument, which is a universally quantified variable of type
    \texttt{Rulename$_P$} of the appropriate type, bound in the
    \texttt{for}-list of the rule.
  \end{enumerate}
\item In the main processing stage, \texttt{restrictSubjectTo} in the rule
  annotations generates rules according to:
  \begin{itemize}
\item $\mathtt{restrictSubjectTo}\; r_1\; [] = r_1$
\item $\mathtt{restrictSubjectTo}\; r_1\; (r' \uplus rs) =$\\
  $\mathtt{restrictSubjectTo}\; (r_1(precond := precond(r_1) \AND \NOT postcond(r')))\; rs$
  Thus, the essential difference \wrt{} the definition of
  \secref{sec:restr_precond} is that we add the negated post-condition and not
  the negated pre-condition.
\end{itemize}
\end{enumerate}

\begin{example} The rules of \exampleref{ex:small_propositional} are now
  transformed to:
\begin{lstlisting}[mathescape=true]
rule <r1> for rn:Rulename$_{B1}$ if B1$^+$ rn then C1$^+$ r1
rule <r2> for rn:Rulename$_{B2}$ if B2$^+$ rn and not C1$^+$ r1 then C2$^+$ r2
\end{lstlisting}
The derivability of another instance of \texttt{C1}, such as \texttt{C1$^+$ r3},
would not inhibit the application of \texttt{r2} any more.
\end{example}


\begin{example} The two rules of the running example become, after resolution
  of the \texttt{restrictSubjectTo} clauses:
\begin{lstlisting}[mathescape=true]
rule <maxSpSportsCar>
   for v: Vehicle, d: Day, r: Road
   if isSportsCar v && isHighway r &&
      not maxSp$^+$ maxSpCarWorkday v d r 90
   then maxSp$^+$ maxSpSportsCar v d r 320
rule <maxSpCarHighway>
   for v: Vehicle, d: Day, r: Road
   if isCar v && isHighway r &&
      not maxSp$^+$ maxSpCarWorkday v d r 90 &&
      not maxSp$^+$ maxSpSportsCar v d r 320
   then maxSp$^+$ maxSpCarHighway v d r 130
\end{lstlisting}
\end{example}



% ----------------------------------------------------------------------
\subsection{Comparison}\label{sec:comparison}

One may wonder whether, starting from the same set of rules, the transformations in
\secref{sec:restr_precond} and \secref{sec:restr_deriv} produce
equivalent rules. On the face of it, this is not so, because the
transformation via derivability modifies the arity of the predicates, so the
rule sets have different models.

We will however show that the two rule sets have corresponding sets of
models. This will be made more precise in the following. To fix notation,
assume ${\cal R}_M$ to be a set of rules annotated with rule modifiers. Let
${\cal R}_P$ be the set of rules obtained from ${\cal R}_M$ through the rule
translation via preconditions of \secref{sec:restr_precond}, and similarly
${\cal R}_D$ the set of rules obtained from ${\cal R}_M$ through the rule
translation via derivability of \secref{sec:restr_deriv}. From these rule
sets, we obtain formula sets ${\cal F}_P$ respectively ${\cal F}_D$ by
\begin{itemize}
\item translating rules to formulas;
\item adding inversion formulas $Inv_C$ for all
  the transformable predicates $C$ of the rule set;
% \item in the case of ${\cal F}_D$, adding exhaustivity predicates for all the
%   \texttt{Rulename$_C$} types\remms{Parameterize Rulename types with
%     predicates}, of the form $\forall x: \mathtt{Rulename}_C.\; x = rn_1 \OR
%   \dots \OR x=rn_r$ if $\{rn_1, \dots, rn_r\}$ are the rule names having $C$
%   as conclusion.\remms{Maybe not required?}
\end{itemize}


\begin{proposition}\label{lemma:mp_to_md}
  Any model ${\cal M}_P$ of ${\cal F}_P$ can be transformed into a model
  ${\cal M}_D$ of ${\cal F}_D$.
\end{proposition}

\begin{proof}
  We consider the transformation of a model ${\cal M}_P$ to a model
  ${\cal M}_D$, and assume ${\cal M}_P$ is a model of ${\cal F}_P$. 
  We now construct an interpretation ${\cal M}_D$ for the formulas with the
  signature over ${\cal F}_D$.

  The interpretation ${\cal M}_D$ will be the same as ${\cal M}_P$,
  except for (1) the interpretation of the new types \texttt{Rulename$_C$},
  each of which will be chosen to be the set of all rule names having $C$ as
  conclusion, and (2) the interpretation of the new predicates $C^+$ on which
  we will focus now: 
  For each rule  $\forall x_1, \dots, x_n.\; Pre(x_1,
  \dots, x_n) \IMPL C(x_1, \dots, x_n)$ with name  $rn$, whenever the $n$-tuple
  $(a_1, \dots, a_n)$ satsifies the precondition $Pre$ under ${\cal M}_P$ and, consequently,
  $(a_1, \dots, a_n) \in C^{{\cal M}_P}$, we will have   $(rn, a_1, \dots, a_n) \in (C^+)^{{\cal M}_D}$.

  It remains to be shown that ${\cal M}_D$ is indeed a model of ${\cal
    M}_D$. We show that related formulas in ${\cal F}_P$ and ${\cal F}_D$
  are interpreted as true in ${\cal M}_P$ resp.{} ${\cal M}_D$, where two
  formulas are \emph{related} if they are rules originating from the same rule
  of ${\cal R}_M$, or if they are related inversion predicates $Inv_C$ and $Inv_{C^+}$.

  We first address related rules. The proof is by well-founded induction over
  the rule order $\prec_R$. Consider a rule $r_P \in {\cal F}_P$ with rule
  name $rn_P$ which by construction has the form
  $r_p = \forall x_1, \dots x_n.\; pre_P^o \AND \NOT pre_P^1 \AND \NOT pre_P^k
  \IMPL C(x_1, \dots, x_n)$.
  We make a case distinction:
  \begin{itemize}
  \item Assume that for arguments $(a_1, \dots, a_n)$, interpretation
    ${\cal M}_P$ satisfies the precondition
    $pre_P^o \AND \NOT pre_P^1 \AND \NOT pre_P^k$ and thus also the
    conclusion. In this case, $(rn_P, a_1, \dots, a_n)\in (C^+)^{{\cal M}_D}$, thus
    satisfying the related rule $r_D \in {\cal F}_D$.
  \item Assume that for arguments $(a_1, \dots, a_n)$, interpretation
    ${\cal M}_P$ does not satisfy the precondition. Either $pre_P^o$ is not
    satisfied, leading again to a satisfying assignment of the related rule
    $r_D$, or one of the $pre_P^i$ is satisfied.

    In this case, as the rule $r_P^i$ with precondition $pre_P^i$ is strictly
    smaller than $r_P$ \wrt{} $\prec_R$, by induction hypothesis, also the
    postcondition of $r_P^i$ will be satisfied, so that in ${\cal M}_D$, one
    negated precondition of the related rule $r_D$ is not satisfied, so $r_D$
    is satisfied.
  \end{itemize}

  Once the equi-satisfiability of related rules has been established, it is
  easy to do so for related inversion predicates $Inv_C$ and $Inv_{C^+}$.
\end{proof}


\begin{proposition}\label{lemma:md_to_mp}
  Any model ${\cal M}_D$ of ${\cal F}_D$ can be transformed into a model
  ${\cal M}_P$ of ${\cal F}_P$.
\end{proposition}

\begin{proof} (Sketch)
  In analogy to \propositionref{lemma:mp_to_md}, we start from a model ${\cal M}_D$
  of ${\cal F}_D$ and construct a model ${\cal M}_P$ of ${\cal F}_P$. 

  As in \propositionref{lemma:mp_to_md}, the proof is by induction on $\prec_R$.
  Consider a rule $r_D \in {\cal F}_D$ with rule
  name $rn_D$ which by construction has the form
  $r_D = \forall x_1, \dots x_n.\; pre_D^o \AND \NOT post_D^1(rn_1) \AND \NOT post_D^k(rn_k)
  \IMPL C^+(rn_D, x_1, \dots, x_n)$. Again, we make a case distinction:
  \begin{itemize}
  \item Assume that for arguments $(a_1, \dots, a_n)$, interpretation
    ${\cal M}_D$ satisfies the precondition and thus also the conclusion. In
    this case, $(a_1, \dots, a_n)\in C^{{\cal M}_P}$, thus satisfying the
    related rule $r_P \in {\cal F}_P$.
  \item Assume that for arguments $(a_1, \dots, a_n)$, interpretation
    ${\cal M}_D$ does not satisfy the precondition. The interesting situation
    is if one $post_D^i(rn_i)$ is satisfied. At this point, we need the
    inversion formula of $post_D^i$, of the form
    $\forall r.\; post_D^i(r) \IMPL P_1(r) \OR \dots \OR P_p(r)$. The rule
    name $rn_i$ permits to select precisely the precondition $P_j$ of the
    related formula
    $r_P = \forall x_1, \dots x_n.\; pre_P^o \AND \NOT pre_P^1 \AND \NOT
    pre_P^k \IMPL C(x_1, \dots, x_n)$.
  \end{itemize}
\end{proof}
%%% Local Variables:
%%% mode: latex
%%% TeX-master: "main"
%%% End:
