% ----------------------------------------------------------------------
\section{Reasoning with and about Rules}\label{sec:resasoning_with_rules}
% ----------------------------------------------------------------------


Given a plethora of different notions of ``defeasibility'', we had to make a
choice as to which notions to support, and which semantics to give to them. We
will here concentrate on two concepts, which we call \emph{rule modifiers},
that limit the applicability of rules and make them ``defeasible''. They will
be presented informally in the following. Giving them a precise semantics in
classical, monotonic logic is the topic of \secref{sec:defeasible_classical};
a semantics based on Answer Set Programming will be provided in
\secref{sec:defeasible_asp}.

We will concentrate on two rule modifiers that restrict the applicability of
rules and that frequently occur in law texts: \emph{subject to} and
\emph{despite}. A motivating discussion justifying their informal semantics is
given in \appref{sec:resasoning_with_rules_app}, drawn from a detailed
analysis of Singapore's Professional Conduct Rules. In the following, however,
we return to our running example.

\begin{example}
  The rules \texttt{maxSpCarHighway} and \texttt{maxSpCarWorkday} are not
  mutually exclusive and contradict another because they postulate different
  maximal speeds. For disambiguation, we would like to say:
  \texttt{maxSpCarHighway} holds \emph{despite} rule
  \texttt{maxSpCarWorkday}. In L4, rule modifiers are introduced with the aid
  of \emph{rule annotations}, with a list of rule names following the keywords
  \texttt{subjectTo} and \texttt{despite}. Thus, we modify rule
  \texttt{maxSpCarHighway} of \figref{fig:rules} with
\begin{lstlisting}
rule <maxSpCarHighway>
  {restrict: {despite: maxSpCarWorkday}}
# rest of rule unchanged
\end{lstlisting}
Furthermore, to the delight of the public of the country with the highest
density of sports cars, we also introduce a new rule \texttt{maxSpSportsCar}
that holds \emph{subject to} \texttt{maxSpCarWorkday} and \emph{despite}
\texttt{maxSpCarHighway}:
\begin{lstlisting}
rule <maxSpSportsCar>
  {restrict: {subjectTo: maxSpCarWorkday, 
              despite: maxSpCarHighway}}
   for v: Vehicle, d: Day, r: Road
   if isSportsCar v && isHighway r
   then maxSp v d r 320
 \end{lstlisting}
\end{example}

We will now give an informal characterization of these modifiers:
\begin{itemize}
\item $r_1$ \emph{subject to} $r_2$ and $r_1$ \emph{despite} $r_2$ are complementary
  ways of expressing that one rule may override the other rule. They have in
  common that $r_1$ and $r_2$ have contradicting conclusions. The conjunction
  of the conclusions can either be directly unsatisfiable (such as: ``may hold'' vs.{}
  ``must not hold'') or unsatisfiable \wrt{} an intended background theory
  (obtaining different maximal speeds is inconsistent when expecting
  \texttt{maxSp} to be functional in its fourth argument).
\item Both modifiers differ in that \emph{subject to} modifies the rule to which
  it is attached, whereas \emph{despite} has a remote effect on the rule given
  as argument.
\item They permit to structure a legal text, favouring conciseness and
  modularity: In the case of \emph{despite}, the overridden, typically more
  general rule need not be aware of the overriding, typically subordinate rules.
\item Even though these modifiers appear to be mechanisms on the meta-level in
  that they reasoning about rules, they can directly be reflected on the
  object-level.
\end{itemize}

%%% Local Variables:
%%% mode: latex
%%% TeX-master: "main"
%%% End:
